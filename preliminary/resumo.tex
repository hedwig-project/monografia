% TODO - same as abstract but in portuguese

\begin{resumo}
%
O crescente desenvolvimento nas tecnologias de Internet das Coisas (IoT) trazem inúmeras oportunidades para a reinvenção dos nossos arredores, sendo a inteligência e eficiência peças chaves na concepção de novos produtos. Desde os mais básicos, aparelhos que hoje operam isoladamente, passarão a fazer parte de um sistema complexo, integrado, onde a troca de informações é requisito integral para o funcionamento. A elaboração de uma sólida infraestrutura de comunicação é vital ao processo e, juntamente com a aplicação de tais tecnologias, surgem conceitos atuais, como o de Casas e Cidades Inteligentes (Smart Homes e Smart Cities, respectivamente), que oferecem um paradigma novo, responsável por modernizar a vivência urbana.

Com base nesse cenário, o presente trabalho tem o objetivo de desenvolver um sistema completo para casas inteligentes, onde serão exploradas as tecnologias de comunicação e conectividade entre dispositivos, e fornecer uma plataforma acessível e expansível para a automatização de residências, com baixo custo envolvido.

Circuitos microcontrolados, atuadores, sensores e radiotransmissores são vastamente utilizados nos módulos físicos, instalados na residência. A criação de um protocolo para troca de mensagens, em conjunto com o sistema de mensageria, do tipo publicação e subscrição, e coordenado por um servidor, formam a infraestrutura local de troca de informações. O usuário final interage com o sistema por meio de clientes web, controlados por serviços na nuvem, com conexão por WebSocket com a casa.

Todo o protótipo desenvolvido mostrou-se viável e funcional, atendendo aos requisitos propostos e aos testes realizados. Espera-se que esta iniciativa possa ser continuada, em cima da fundação atual.

%
\textbf{Palavras-Chave} -- IoT, Smart Houses, Smart Cities, Infraestrutura de comunicação, Mensageria.
\end{resumo}
