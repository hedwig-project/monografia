\begin{resumo}
% TODO: peças-chaves ou peças-chave? :emoji pensativo:
O crescente desenvolvimento das tecnologias de Internet das Coisas (\emph{IoT}) traz inúmeras oportunidades para a reinvenção dos nossos arredores, sendo a inteligência e eficiência peças chaves na concepção de novos produtos. Mesmo os mais básicos aparelhos, que hoje operam isoladamente, passarão a fazer parte de um sistema complexo, integrado, no qual a troca de informações é requisito básico para o funcionamento. Dessa forma, a elaboração de uma sólida infraestrutura de comunicação é vital ao processo. Juntamente com tais tecnologias, surgem conceitos atuais para suas aplicações, como os de Casas e Cidades Inteligentes - \textit{Smart Homes} e \textit{Smart Cities}, respectivamente -, que oferecem um novo paradigma responsável por modernizar a vivência urbana.

Com base nesse cenário, o presente trabalho tem o objetivo de desenvolver um sistema completo para casas inteligentes. Nele, são exploradas as tecnologias de comunicação e conectividade entre dispositivos, criando assim uma plataforma acessível e expansível para a automatização e monitoração de residências - tudo isso com baixo custo envolvido.

Circuitos microcontrolados, atuadores, sensores e radiotransmissores são vastamente utilizados nos módulos físicos, que ficam instalados na residência. A infraestrutura local de comunicação é formada por um protocolo para troca de mensagens e um sistema de mensageria do tipo publicação e subscrição coordenado por um servidor. O usuário final interage com o sistema por meio de aplicativos web, que utilizam-se de serviços na nuvem e conectam-se com as casas por meio de WebSockets.

Todo o protótipo desenvolvido mostrou-se viável e funcional, atendendo aos requisitos propostos e aos testes realizados. Espera-se que esta iniciativa possa ser continuada em cima da fundação atual.

%
\textbf{Palavras-Chave} -- Internet das Coisas, Casas Inteligentes, Cidades Inteligentes, Infraestrutura de comunicação, Mensageria.
\end{resumo}
