\begin{abstract}
%
The increasing development of the Internet of Things (\wiot{}) technologies offers countless opportunities to reinvent our surroundings, with intelligence and efficiency being key concepts during the creation of new products. Even the most basic devices, that currently work in isolation, will become part of a complex integrated system, in which information exchange will be a basic requirement for operation. Thus, the establishment of a solid communication infrastructure is a vital part of the process. In conjunction with such technologies, modern concepts for their application are devised, such as Smart Homes and Smart Cities, offering a new paradigm responsible for the modernization of urban living.

Based on this scenario, this work aims to provide a complete system for Smart Houses. It explores communication technologies and connectivity between devices, creating an accessible and expandable platform for residency automation --- all of that with low production cost.

Microcontroller circuits, in addition to actuators, sensors  and radio transmitters, are vastly used on the hardware modules. The local communication infrastructure is composed by a messaging exchange protocol and a publisher/subscriber messaging broker controlled by a server. The end user interacts with the system through a web client, which uses cloud services and are connected to the home via WebSockets.

All of the prototypes developed proved to be viable and functional, fulfilling the specified requirements and passing performed tests. Hopefully, this initiative will be continued and further improved on top of this underlying foundation.

%
\textbf{Keywords} -- \wiot{}, Smart Houses, Smart Cities, Communication infrastructure, Messaging Systems.
\end{abstract}
