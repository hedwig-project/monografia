\documentclass[twosideprint]{politex}

% REMOVER LINHA ABAIXO PARA VOLTAR À FONTE NORMAL DO LATEX
% \renewcommand{\familydefault}{\sfdefault}
% ========== Opções ==========
% pnumromarab - Numeração de páginas usando algarismos romanos na parte pré-textual e arábicos na parte textual
% abnttoc - Forçar paginação no sumário conforme ABNT (inclui "p." na frente das páginas)
% normalnum - Numeração contínua de figuras e tabelas
%	(caso contrário, a numeração é reiniciada a cada capítulo)
% draftprint - Ajusta as margens para impressão de rascunhos
%	(reduz a margem interna)
% twosideprint - Ajusta as margens para impressão frente e verso
% capsec - Forçar letras maiúsculas no título das seções
% espacosimples - Documento usando espaçamento simples
% espacoduplo - Documento usando espaçamento duplo
%	(o padrão é usar espaçamento 1.5)
% times - Tenta usar a fonte Times New Roman para o corpo do texto
% noindentfirst - Não indenta o primeiro parágrafo dos capítulos/seções


% ========== Packages ==========
\usepackage[utf8]{inputenc}
\usepackage{amsmath,amsthm,amsfonts,amssymb}
\usepackage{graphicx,cite,enumerate}
\usepackage{blkarray}
\usepackage{float}
\usepackage{caption}
\usepackage{ctable}
\usepackage{textcomp}
\graphicspath{ {images/} }

% ========= Check Marks =========
\usepackage{amssymb}% http://ctan.org/pkg/amssymb
\usepackage{pifont}% http://ctan.org/pkg/pifont
\newcommand{\cmark}{\ding{51}}%
\newcommand{\xmark}{\ding{55}}%

% ========== Language options ==========
\usepackage[brazil]{babel}
%\usepackage[english]{babel}

% ========== ABNT (requer ABNTeX 2) ==========
%	http://www.ctan.org/tex-archive/macros/latex/contrib/abntex2
\usepackage[alf]{abntex2cite}

% ========== Packages to display code ==========
\usepackage{listings}
\usepackage{color}
\lstset {
    basicstyle=\ttfamily,
    columns=fullflexible,
    breaklines=true,
    postbreak=\mbox{\textcolor{red}{$\hookrightarrow$}\space},
}

% Forçar o abntex2 a usar [ ] nas referências ao invés de ( )
%\citebrackets{[}{]}

% ========== Opções do documento ==========
% Título
\titulo{Hedwig - Casa Conectada}

% Autor
%\autor{Nome Sobrenome}

% Para múltiplos autores (TCC)
\autor{Daniela Sayuri Yassuda\\
Gabriela Souza de Melo\\
Hugo da Silva Possani\\[0.15cm]
Victor Takashi Hayashi}

% Orientador / Coorientador
\orientador{Prof. Dr. Reginaldo Arakaki}
\coorientador{Eng. Marcelo Pita}


% Tipo de documento
\tcc{Eletricista com ênfase em Computação}

% Departamento e área de concentração
\departamento{PCS}
\areaConcentracao{Engenharia de Computação}

% Local
\local{São Paulo}

% Ano
\data{2017}

\begin{document}

% ========== Palavras usadas - comecam com w, para evitar colisao com comandos ==========
\newcommand{\wwifi}{\emph{WiFi}}
\newcommand{\wmqtt}{\emph{MQTT}}
\newcommand{\wid}{\emph{ID}}
\newcommand{\wiot}{\emph{IoT}}

% ========== Capa e folhas de rosto ==========
\capa
\newpage\mbox{}

\falsafolhaderosto
\newpage\mbox{}

\folhaderosto
% \newpage\mbox{}


% ========== Folha de assinaturas (opcional) ==========
%\begin{folhadeaprovacao}
%	\assinatura{Prof.\ X}
%	\assinatura{Prof.\ Y}
%	\assinatura{Prof.\ Z}
%\end{folhadeaprovacao}


% ========== Ficha catalográfica ==========
% Fazer solicitação no site:
%	http://www.poli.usp.br/en/bibliotecas/servicos/catalogacao-na-publicacao.html


% ========== Dedicatória (opcional) ==========
%\dedicatoria{Dedicatória}
% TODO

% ========== Agradecimentos ==========
\begin{agradecimentos}
A Fabio Hirotsugu Hayashi, principal responsável pela montagem e circuito eletrônico dos módulos. Sem sua ajuda, esse trabalho não seria possível.
% TODO

\end{agradecimentos}


% ========== Epígrafe (opcional) ==========
% \epigrafe{

% TODO

% 	\emph{``Anything one man can imagine, other men can make real''}
% 	\begin{flushright}
%		-{}- Jules Verne
%	\end{flushright}

%	\hfill \break

% \newpage\mbox{}\newpage

% ========== Resumo ==========
\begin{resumo}

O crescente desenvolvimento das tecnologias de Internet das Coisas (\wiot) traz inúmeras oportunidades para a reinvenção dos nossos arredores, sendo a inteligência e eficiência peças-chave na concepção de novos produtos. Mesmo os mais básicos aparelhos, que hoje operam isoladamente, passarão a fazer parte de um sistema complexo, integrado, no qual a troca de informações é requisito básico para o funcionamento. Dessa forma, a elaboração de uma sólida infraestrutura de comunicação é vital ao processo. Juntamente com tais tecnologias, surgem conceitos atuais para suas aplicações, como os de Casas e Cidades Inteligentes --- \textit{Smart Homes} e \textit{Smart Cities}, respectivamente ---, que oferecem um novo paradigma responsável por modernizar a vivência urbana.

Com base nesse cenário, o presente trabalho tem por objetivo o estudo da arquitetura de um sistema completo para automação residencial, com a utilização do conceito de Internet das Coisas, e de contingências no caso de falhas na comunicação. Nele, são exploradas as tecnologias de comunicação e conectividade entre dispositivos, criando assim uma plataforma acessível e expansível para a automatização e monitoramento de residências --- tudo isso com baixo custo envolvido.

Circuitos microcontrolados, atuadores, sensores e radiotransmissores são vastamente utilizados nos módulos físicos, que ficam instalados na residência. A infraestrutura local de comunicação é formada por um protocolo para troca de mensagens e um sistema de mensageria do tipo publicação e subscrição coordenado por um servidor. O usuário final interage com o sistema por meio de aplicativos híbridos, que utilizam-se de serviços na nuvem e conectam-se com as casas por meio de WebSockets.

Todo o protótipo desenvolvido mostrou-se viável e funcional, atendendo aos requisitos propostos e aos testes realizados. Espera-se que esta iniciativa possa ser continuada em cima da fundação atual.

%
\textbf{Palavras-Chave} -- Internet das Coisas, Casas Inteligentes, Cidades Inteligentes, Infraestrutura de comunicação, Mensageria.
\end{resumo}

\newpage\mbox{}\newpage
% ========== Abstract ==========
\begin{abstract}
%
An  increasing development in the Internet of Things (IoT) technologies offer countless opportunities to reinvent our surroundings, where intelligence and efficiency are key concepts in the creation of new products. Even the most basic devices, that currently work in isolation, will become part of a complex integrated system, and information exchange will be an integral requirement for operation. The establishment of a solid communication infrastructure is a vital part of the process and, in conjunction with the application of such technologies, it brings modern concepts, such as Smart Homes and Smart Cities, and offers a new paradigm responsible for the modernization of urban living.

Based on the previous scenario, this work aims to provide a complete system for Smart Houses. It explores communication technologies and connectivity between devices, with an accessible and expandable platform for residencies automation, and low cost production.

Microcontroller circuits, in addition to actuators, sensors  and radio transmitters, are vastly used on the hardware modules. The creation of a messaging exchange protocol, and the use of a publisher/subscriber messaging broker, all controlled by a server, composes the local communication infrastructure. The user interacts with the system through a web client, powered by cloud microservices, which are connected via WebSockets to the home.

All of the prototypes developed proved to be viable and functional, fulfilling the specified requirements and passing performed tests. With hope, this initiative will be continued and further improved on top of this underlying foundation.

%
\textbf{Keywords} -- IoT, Smart Houses, Smart Cities, Communication infrastructure, Messaging Systems.
\end{abstract}

\newpage\mbox{}\newpage
% ========== Listas (opcional) ==========
\listadefiguras
\listadetabelas

% ========== Listas definidas pelo usuário (opcional) ==========
\begin{pretextualsection}{Lista de siglas}
\begin{table}[H]
\centering
\label{my-label}
\begin{tabular}{lll}
ACME  &  & \textit{Automatic Certificate Management Environment}     \\
API   &  & \textit{Appliaction Programming Interface}                \\
CAGR  &  & \textit{Compound Annual Growth Rate}                      \\
CDN   &  & \textit{Content Delivery Network}                         \\
CDMA  &  & \textit{Code Division Multiple Access}                    \\
CGNAT &  & \textit{Carrier-grade NAT}                                \\
CORS  &  & \textit{Cross-Origin Resource Sharing}                    \\
DoS   &  & \textit{Denial of Service}                                \\
DDoS   &  & \textit{Distributed Denial of Service}                                \\
E/S   &  & Entrada / Saída                                           \\
EEPROM & & \textit{Electrically-Erasable Programmable Read-Only Memory} \\
HTTP  &  & \textit{Hypertext Transfer Protocol}                      \\
I2C   &  & \textit{Inter-Integrated Circuit}                         \\
IDE   &  & \textit{Integrated Development Environment}               \\
IoC   &  & \textit{Inversion of Control}                             \\
IoT   &  & \textit{Internet of Things}                               \\
IP    &  & \textit{Internet Protocol}                                \\
JSON  &  & \textit{JavaScript Object Notation}                       \\
JVM   &  & \textit{Java Virtual Machine}                             \\
LCD   &  & \textit{Liquid Crystal Display}                           \\
LDR   &  & \textit{Light Dependent Resistor}                         \\
M2M   &  & \textit{Machine to Machine}                               \\
MOOC  &  & \textit{Massive Open Online Course}                       \\
MQTT  &  & \textit{Message Queuing Telemetry Transport}              \\
NAT   &  & \textit{Network Address Translation}                      \\
NoSQL &  & \textit{Not Only SQL}                                     \\
PIR   &  & \textit{Passive Infrared Sensor}                          \\
PWA   &  & \textit{Progressive Web App}                              \\
QoS   &  & \textit{Quality of Service}                               \\
REST  &  & \textit{Representational State Transfer}                  \\
RF    &  & Radiofrequência                                           \\
SOA   &  & \textit{Service-Oriented Architecture}                    \\
SQL   &  & \textit{Structured Query Language}                        \\
SSID  &  & \textit{Service Set Identifier}                           \\
TCP   &  & \textit{Transmission Control Protocol}                    \\
TLS   &  & \textit{Transport Layer Security}                         \\
UI    &  & \textit{User Interface}                                   \\
URL   &  & \textit{Uniform Resource Locator}                         \\
UX    &  & \textit{User Experience}                                  \\
WLAN  &  & \textit{Wireless LAN}                                     \\
WPA   &  & \textit{Wi-Fi Protected Access}                           \\
XML   &  & \textit{eXtensible Markup Language}                       \\
YAML  &  & \textit{YAML Ain't Markup Language}
\end{tabular}
\end{table}

\end{pretextualsection}

\newpage\mbox{}
% ========== Sumário ==========
\sumario


% ========== Elementos textuais ==========

\chapter{Introdução}

\section{Motivação}
Há uma expectativa de que o número de casas inteligentes aumente cerca de 17\% nos Estados Unidos no ano de 2017 \cite{mckinseyReport}, onde já se tem investimentos de grandes empresas, como Google, Amazon e Apple, mostrando a relevância do tema no momento atual. O interesse nessa área é tamanho que a Google investiu cerca de 5 milhões de dólares em um comercial de seu produto Google Home no Super Bowl 2017 (final de futebol americano nos EUA) \cite{kennemer}.

Assim, as oportunidades trazidas pelo conceito de Internet das Coisas (IoT) à área de automação residencial são uma grande motivação para esse projeto. Também destacam-se as possibilidades de promover tais tecnologias de casas inteligentes ao mercado nacional, personalizando produtos e adequando-as às necessidades dos potenciais consumidores brasileiros. Mesmo nos Estados Unidos, ainda é necessário algum tempo até que as casas conectadas se consolidem, de modo que há grandes oportunidade de pioneirismo no mercado brasileiro, com o lançamento de produtos de IoT a preços acessíveis e focando nas necessidades dos consumidores locais.

\section{Projeto Hedwig}

\subsection{Objetivo}
A contribuição do projeto será um sistema baseado em arquitetura local modularizada, e em camadas, com funcionalidades local e em nuvem, e provedor de uma \textit{API} que permita seu acesso por diversos clientes - como \textit{websites} ou aplicativos para \textit{smartphones} - que seja capaz de monitorar e agir em diversos módulos presentes na residência do usuário final do sistema. O projeto irá disponibilizar módulos físicos, prontos para serem instalados e configurados na residência, sem que seja necessário conhecimentos avançados de eletrônica ou computação.

Desta forma, os principais pontos do projeto são:

\begin{itemize}
\item \textbf{Robustez}

3 níveis de funcionamento: Online, Local e Offline, para garantir a disponibilidade mesmo com problemas (queda do servidor, internet indisponível, falha no roteador), com medidas para a tentativa automática de reconexão, monitoramento e manutenções preventivas e corretivas do sistema.

\item \textbf{Modularidade}

Garante a independência de funcionamento dos módulos que atendem às várias necessidades, contribuindo para a robustez. Diminui o custo e personaliza o produto, de acordo com as necessidades do cliente.

\item \textbf{Camadas}

O funcionamento da aplicação decorre em diversos níveis e camadas, de responsabilidades independentes, permitindo maior separação de responsabilidades.

\item \textbf{Machine Learning}

Levantamento de rotinas para gerar conhecimento, que se mostra como notificações, alertas e acionamentos automáticos de funções para o cliente.

\item \textbf{Segurança}

Utilização de criptografia assimétrica para comunicação entre servidor local e serviços de nuvem, juntamente com conexão por \textit{WebSocket}. Autenticação e autorização de usuários por métodos \textit{WebToken}. Uso de canais \textit{Publisher/Subscriber} protegidos para troca de mensagens.

\end{itemize}

\subsection{Nome do Projeto}
O nome do projeto foi escolhido em homenagem a Hedy Lamarr. Nascida Hedwig Eva Maria Kiesler \cite{shearer}, a atriz e inventora desenvolveu, durante a Segunda Guerra Mundial, um aparelho de interferência em rádio para despistar radares nazistas, cujos princípios estão incorporados nas tecnologias atuais de Wi-fi, CDMA e Bluetooth \cite{electronicFrontier}. Baseado na ideia de um sistema de comunicação seguro, e como reconhecimento de seu trabalho, foi dado esse nome ao projeto aqui descrito.

\subsection{Logo}
O logo do projeto é uma coruja, também em referência à coruja \textit{Hedwig}, pertencente ao personagem \textit{Harry Potter}, da série de livros de mesmo nome.
\begin{figure}[H]
	\centering
	\caption{Projeto Hedwig}
  \includegraphics[width=0.2\textwidth]{hedwigLogo}
\label{fig:hedwigLogo}
\end{figure}

\section{Aplicações}
Como aplicações do projeto Hedwig, destacam-se a automação no uso de eletrodomésticos e iluminação, segurança no acesso à casa, economia nas contas de água e energia elétrica, além de um monitoramento remoto de pessoas que moram sozinhas (como é o caso de idosos), garantindo a tranquilidade de seus familiares e mantendo a segurança do indivíduo.

Exemplos de módulos que podem ser incluídos no sistema são: quarto (despertador, iluminação, monitoramento de temperatura e umidade); cozinha (\textit{timer}, iluminação, monitoramento de presença e gás); acesso (controle de abertura, monitoramento de estado); externo (monitoramento de temperatura, umidade, energia elétrica e consumo de água); corredor (monitoramento de presença, iluminação), chuveiro (controle de temperatura\slash potência a partir do perfil de usuário e temperatura externa) e ar condicionado (controle da potência a partir do monitoramento das temperaturas internas e externas da casa).

\subsection{Aplicações de Machine Learning}
\textbf{REVISAR TODA ESSA PARTE}\\
Como possíveis perguntas a serem respondidas pelo módulo de Machine Learning do projeto e os dados a serem coletados (em diferentes lugares da casa), temos:

\begin{itemize}
	
\item Quando notificar a chegada de pessoas ou ambiente vazio? - presença e sensor de abertura do portão
\item Quando enviar alertas de atividade suspeita? - presença
\item Quanto o sistema é usado? (Por funcionalidade) - sensor de abertura e log de aberturas pelo módulo
\item Quando notificar condições insalubres, como temperatura e umidade altas persistentes? - temperatura e umidade
\item Quando notificar falta de atividades rotineiras (como acordar, almoçar) - presença
\item Melhor horário para despertar? - presença
\item Notificar mudança brusca de temperatura, principalmente esfriamento? - temperaturas interna e externa, umidade (para sensação térmica)
\item Quanto o sistema está indisponível na instalação do cliente? - log de qualquer dado periódico
\item Quando acender ou apagar a luz? - presença, acionamento manual (horário e módulo)
\item Quantas vezes notificar? Prioridades? - respostas do cliente (log), para notificar o mínimo necessário, e classificação de notificações (email, somente quando usuário abre o aplicativo, notificação no celular e até módulo de painel externo com buzzer, no caso de comunicação de situação de perigo entre residências fisicamente separadas).
\end{itemize}

Respondendo a essas perguntas, esperamos contribuir para a construção de um sistema autônomo, que aprende com feedbacks do usuário seja pelo monitoramento por módulos ou respostas dadas pelo aplicativo, atuando em segurança (safety), saúde e A
automação da residência do cliente.

\subsection{Módulo de Acesso}

\textbf{Essa definição não deve estar na introdução. Mudar para onde fizer sentido}

Buscando garantir mais segurança e comodidade para o acesso à residência, além dO controle de abertura, o módulo de acesso atua em paralelo com uma fechadura eletrônica, que é acionada por meio do controle, que utiliza ondas de rádio. Assim, mesmo com falha total do sistema, o usuário pode abrir o portão diretamente, sem a necessidade de acesso à internet.

\begin{figure}[H]
	\centering
	\caption{Diagrama ilustrativo do módulo de Acesso ao Portão}
  \includegraphics[width=0.8\textwidth]{diagramaModuloAcesso}
\label{fig:diagramaModuloAcesso}
\end{figure}

O diagrama ilustra, em vermelho, os sistemas já existentes. Sensor e sirene sem fio adicionais são mostrados em verde (dispositivos externos ao módulo, que se comunicam por rádio); o próprio módulo de acesso, com um buzzer embutido, e sua conexão com a rede local Wi-Fi ou sua conexão direta com o celular (quando o módulo opera como um ponto de acesso de rede) em amarelo, além de funcionalidades adicionais, em roxo.

A comodidade, no exemplo em questão, está em abrir o portão por meio do celular, ao utilizar o aplicativo web ou o aplicativo local (de emergência), sem a necessidade de carregar uma chave ou controle.

Entretanto, é necessário realizar que a realização do controle de acesso seja feita de maneira segura. Assim, a funcionalidade é apenas local (o usuário deve estar com o celular conectado à rede da casa para acessar a página local), e um algoritmo de rotação de teclas é utilizado, para evitar que pessoas mal intencionadas possam (1) olhar e copiar a senha que o usuário digita em seu celular e (2) copiar os dados de abertura e usá-los mais tarde (“middle man”). Na última alternativa, a cada acesso de um usuário, um novo mapeamento de teclas é gerado e enviado ao usuário. Mesmo que haja cópia, ela não funcionará devido ao mapeamento ter mudado. Observe ainda que a fechadura eletrônica, por si só, já estava vulnerável a este tipo de ataque (há, inclusive, dispositivos copiadores de senhas).

Outro aspecto de segurança é a preocupação dos usuários em esquecer a porta, ou portão, abertos. Para mitigar esse perigo, o módulo deve monitorar, por meio de um sensor, o estado da vigente (aberto/fechado), e alertar localmente (por meio de “buzzer”) e remotamente (e.g. por email ou notificação no \textit{smartphone}) o usuário. Essa e outras configurações (como de rede) são acessadas por uma senha diferente daquela de abertura, de modo que a interface básica seja simples para uso.

Para o caso de falha de envio de notificação (e.g. servidor fora do ar, ou indisponibilidade na conexão), há um algoritmo de novas tentativas com tempos progressivamente maiores conforme as falhas ocorrerem, buscando deixar o módulo disponível para outras funções. Tratamento análogo é realizado no servidor local, e no sistema de mensageria, de modo a evitar perdas de mensagens, mesmo em situações desfavoráveis. Para o caso de falta de conexão à internet, o módulo não seria controlável pela nuvem, com o aplicativo web, mas sim com o aplicativo emergencial, com a ativação do \textit{Access Point}, desenvolvido para operar diretamente com os módulos, sem intermédio do servidor local e dos serviços remotos.

% TODO (victor) checar esse parágrafo abaixo. Acho que não ficou claro o que é esse "travamento"
Para garantir que o módulo está ativo, utiliza-se um sinal de “keep alive” monitorado, e um circuito anti-travamento deve ativar um “hard reset” (reset por hardware), ou então uma rotina de “soft reset” deve ser acionada. No entanto, observe que a segunda alternativa é a mais fácil de implementar, mas é menos robusta, já que ainda pode não funcionar em casos de loop infinito.

% TODO (victor) não entendi a parte do "executa algoritmo análogo ao do envio de emails"
Outra situação que poderia gerar indisponibilidade do sistema é um ataque de DoS local (“Evil Twin”), no qual uma rede mal intencionada usa o mesmo SSID (\textit{Service Set Identifier}, o nome associado à rede WLAN) da rede original, tentando obter a senha na ocasião de reconexão de módulos. Muitas vezes, é também acompanhado de rádio interferência e outros procedimentos para fazer os módulos se desconectarem. Para mitigar o risco, cada módulo tenta inicialmente se conectar usando uma senha falsa no SSID fornecido. Caso obtenha sucesso (se a rede for aberta, como é o caso na maioria desses ataques), ele executa algoritmo análogo ao envio de emails (observe que enquanto não está conectado à rede o módulo atua como ponto de acesso e disponibiliza funcionalidades básicas). Caso ele não obtenha sucesso usando a senha errada (portanto, não detectou a situação de “Evil Twin”), o módulo envia a senha correta. Para proteger a rede, um controlador local do sistema pode atuar junto ao roteador e desligar a conexão sem fio enquanto a situação se mantiver.

O controle de acesso pode ser implementado por meio de persistência de dados de login e senha, e o uso de diversas senhas para uma residência (uma para cada morador - isso torna possível o conhecimento dos usuários que abriram o portão sem a necessidade de login prévio, facilitando o uso). O log destes acessos pode ser analisado (utilizando técnicas de Machine Learning) para determinar perfis de acesso, e evoluir até o sistema saber quando houver um acesso em horário inesperado e notificar o usuário remotamente. O aprendizado de máquina é fundamental aqui para descobrir comportamentos que podem ser entendidos como suspeitos. Para um usuário que costuma chegar em um horário aproximado todos os dias, e acionar funções semelhantes da casa, uma tentativa de acesso que não se enquadre em tais padrões pode ser produto de atividade criminosa, a qual pode ser informada pela casa para uma central, que acionará a polícia caso não seja um falso positivo.

\chapter{Projetos Relacionados}

\section{Sistemas Existentes no Mercado}

\subsection{Sistemas Comerciais}
Atualmente, já existem sistemas comerciais de automação residencial --- a maioria deles atuando de maneira mais forte no mercado norte-americano. Alguns dos sistemas mais populares nessa linha são o Amazon Echo e o Google Home.

O Amazon Echo\footnote{http://www.amazon.com/oc/echo/} consiste em um \textit{smart speaker} (alto-falante inteligente) conectado ao assistente pessoal Alexa, também da Amazon, que é capaz de entender comandos de voz. Inicialmente, funcionava como uma maneira de encomendar produtos, mas, atualmente, além de ser assistente pessoal, também é capaz de controlar diversos \textit{smart devices} da casa, como um \textit{hub} de automação residencial. Uma limitação deste produto é dependência de conexão \textit{wireless} de Internet, não sendo capaz de operar em nenhum nível sem ela.

Uma característica interessante do Alexa é a possibilidade de adição de novas \textit{skills} (habilidades) por desenvolvedores, que possuem acesso a uma API pública, documentada e disponibilizada online. Dessa forma, seu \textit{skillset} é passível de grande expansão e personalização. Além disso, o serviço de voz desse sistema, conhecido como Alexa Voice Service, pode ser utilizado por qualquer dispositivo que contenha microfone e alto falante, e que consiga conectar-se a ele pela Internet.

O Google Home\footnote{https://madeby.google.com/home/} é similar ao Amazon Echo em alguns aspectos, sendo também um \textit{smart speaker}, que surgiu como expansão do aplicativo para smartphones Google Now, um assistente pessoal. Atualmente existe também como aplicativo para smartphones. Não é possível o desenvolvimento de módulos e expansões ao Google Home por desenvolvedores desvinculados à Google, porém ela trabalha diretamente com outras marcas e produtos para o estabelecimento de parcerias, de forma que o Google Home também consiga funcionar como \textit{hub} de automação residencial.

\subsection{Sistemas \emph{Open-source}}
Também existem diversos projetos \emph{open-source} sobre o tema, cujas documentações estão disponíveis publicamente na Internet. Alguns desses projetos, analisados para o desenvolvimento do Hedwig, foram o OpenHAB e o Home Assistant.

O OpenHAB\footnote{http://www.openhab.org/} possui como objetivo principal o estabelecimento de uma plataforma de integração, capaz de solucionar o problema atual de haver diversos dispositivos em uma residência que não são capazes de se comunicar, devido à falta de uma linguagem comum para a troca informações. Por ser independente de hardware específico, é extremamente flexível e personalizável, porém isso implica em maior complexidade para o usuário no momento de sua instalação. O OpenHAB apresenta interface para o usuário em cliente web e aplicativos nativos para iOS e Android.

O Home Assistant\footnote{https://github.com/home-assistant/home-assistant} é uma plataforma de automação residencial capaz de controlar e monitorar os diversos dispositivos em uma casa, oferecendo uma plataforma web para o controle do sistema pelo usuário. O controlador local é implementado em Python, e recomenda-se instalá-lo em um Raspberry Pi. Possui diversas integrações já estabelecidas, com sistemas e serviços como o próprio Amazon Echo, Google Cast, IFTTT, Digital Ocean, entre outros, mas possibilita também a criação de novos componentes pelos próprios usuários. A personalização pelos usuários é feita por meio de um arquivo de configuração no formato YAML.

Os dois projetos apresentam como maior dificuldade a necessidade do usuário possuir conhecimentos técnicos para utilizá-los.

\section{Projeto HomeSky}

O Projeto HomeSky \cite{homeSky} é um Trabalho de Conclusão de Curso desenvolvido por alunos de Engenharia de Computação na Escola Politécnica da Universidade de São Paulo. Com o objetivo de fomentar iniciativas de desenvolvimento na área de casas inteligentes, o trabalho focou-se na criação do Rainfall, um protocolo em código aberto a nível de aplicação para ser usado na coordenação de uma rede de sensores. Isso permitiria aos desenvolvedores ter uma maior flexibilidade em seus projetos, visto que muitas das soluções existentes são proprietárias. Por fim, também foi realizada a implementação de um algoritmo de aprendizado de máquina capaz de controlar a iluminação.

\begin{figure}[H]
	\centering
	\caption{Camadas da arquitetura usada no Projeto HomeSky. As camadas em verde correspondem às bibliotecas desenvolvidas no trabalho.}
  \includegraphics[width=0.8\textwidth]{arquiteturaHomeSky}
	\caption*{Fonte: \cite{homeSky}}
\label{fig:arquiteturaHomeSky}
\end{figure}

No desenvolvimento do protocolo Rainfall, foram consideradas algumas hipóteses simplificadoras a respeito da conectividade e da segurança. O protocolo não trata de forma especial a fase de conexão à rede, considerando que todos os nós já estão conectados a ela, e também considera que todos os protocolos adjacentes são confiáveis, delegando as implementações de mecanismos de reconhecimento de entrega e retransmissão ao desenvolvedor. Quanto à segurança, assume-se que a infraestrutura seja segura e que nenhum nó conectado à rede tenha comportamento mal-intencionado, como por exemplo espionar mensagens destinadas a outros nós ou fingir ser o controlador.

\chapter{Especificação}

\section{Componentes}
% TODO

\section{Stakeholders}
Com as funcionalidades e os módulos apresentados, podemos destacar os seguintes grupos dentre os potenciais consumidores:

\begin{itemize}
\item Pessoas que moram sozinhas e suas famílias, que podem estar interessadas em monitoramento;
\item Pessoas que desejam comodidade de controlar seus aparelhos numa interface única, pelo celular, e/ou conforto maior em casa;
\item Pessoas preocupadas com o consumo de água e energia elétrica.
\end{itemize}

Considerando o Censo de 2010 \cite{ibge}, podemos estimar grosseiramente as classes de consumidores para a cidade de São Paulo:

\begin{itemize}
\item Considerando que 1/10 da população com mais de 60 anos more sozinha e que 1/4 deles adquiriria o produto, temos uma estimativa de 33 mil consumidores. Como essa população está envelhecendo em taxas cada vez maiores (8,96\% em 2000 contra 13,6\% em 2016) \cite{bibliotecaVirtual}, a tendência é que essa classe aumente;
\item Considerando que 1/100 dos domicílios ocupados tenha uma pessoa com esse perfil, temos uma estimativa de 35 mil consumidores em potencial;
\item Considerando que cerca de 70\% das residências reduziram o consumo com campanhas de redução de uso de água em 2015 \cite{g1}, supondo que 5\% ficariam preocupados/interessados ao nível de se tornarem consumidores, temos uma estimativa de 71 mil consumidores em potencial.
\end{itemize}

\section{Requisitos}

\subsection{Requisitos Funcionais}
\begin{itemize}
\item O sistema deve permitir o monitoramento de aparelhos do dia a dia, dentro de uma residência, em módulos independentes;
\item O sistema deve ser capaz de enviar notificações aos usuários, seja por meio de um serviço no cliente utilizado pelo usuário (web ou aplicativo \textit{mobile});
\item O sistema deve poder ser personalizável pelo usuário, o qual pode adquirir novos módulos ou retirar algum já existente;
\item O sistema deve ser capaz de aprender a respeito de cada usuário, utilizando conceitos de Machine Learning. O aprendizado de máquina é responsável por detectar padrões no comportamento do usuário, os quais podem ser utilizados para a segurança da casa. Assim, se o usuário, por padrão, chega em casa em uma janela de horário constante, e interage com certos módulos, caso haja uma atividade que não se enquadra no padrão, o comportamento pode ser considerado suspeito, e providências tomadas (como notificações para outros usuários, como alguém da família);
\item O sistema deve manter backup de dados do controlador local na nuvem;
\item O sistema deve permitir ao usuário o seu cadastro na plataforma, pela plataforma que melhor lhe convier;
\item O usuário poderá cadastrar sua casa na plataforma, podendo ter uma ou mais casas cadastradas;
\item O usuário poderá cadastrar os módulos dentro de uma casa, sendo que uma casa pode ter vários módulos, e cada módulo só poderá existir em uma casa;
\item O usuário pode efetuar as operações de remoção e modificação nos seus módulos e casas;
% TODO rever item abaixo, acho que talvez essa funcao de reset nao tenha ficado muito claro. Tipo, o que ela faz, qual parte do sistema ela afeta, etc. Dado que nesse contexto, o "sistema" é tudo, ou seja, API, etc tb. E esse reset nao afetaria API. Entao acho que vale a reescrita desse requisito
\item O sistema deve possuir uma função de reset de fácil utilização.
\end{itemize}

\subsection{Requisitos Não-Funcionais}
O levantamento de requisitos não-funcionais foi realizado com base na norma ISO25010:2011 \cite{iso25010}.

\begin{itemize}
\item Os módulos que compõem o sistema dentro de uma residência devem ser independentes entre si, devendo obedecer a uma interface comum de integração com o core do projeto, para que seja facilitada a ampliação e a inserção de novos módulos, com outras funcionalidades. Haverá validação com o desligamento de um módulo e verificação do comportamento dos demais;
\item O sistema deve garantir segurança dos dados por meio de protocolo de comunicação seguro, tanto para o controle de acesso à API por usuários autenticados quanto para impedir que dados sejam interceptados em sua transmissão;
\item O banco de dados deve possuir acesso restrito e estar hospedado em servidor de alta segurança;
% TODO como assim "em menos de 10 minutos"? podemos mover para uma seção específica do módulo de acesso
\item O sistema deve ser robusto, de modo a continuar operando, mesmo com menor nível de funcionalidades, quando da ocorrência de falhas na comunicação com a nuvem (indisponibilidade parcial devido a problemas com os servidores remotos, ou total com perda da conexão com a Internet) ou falhas na rede local (indisponibilidade da conexão com a rede local). Também deve se recuperar em caso de travamento total do módulo e continuar funcionando em caso de DoS Local. Para validação, haverão testes de indisponibilidade de servidor, conexão com a internet, rede local e DoS local, e observação da continuidade de serviço de atuação na iluminação da casa e abertura do portão em menos de 10 minutos;
% TODO era bom falar algo sobre os dois itens abaixo (tipo, se a gente fez algo pra tentar seguir essa meta)
\item O sistema deve apresentar disponibilidade de 99,9\% - cerca de 8 horas de indisponibilidade por ano -, não levando em consideração problemas com a conexão de internet da residência;
\item O sistema deve ser escalável a até 10 mil usuários, sem perdas de desempenho consideráveis, ou aumento na latência para as requisições serem atendidas;
\item O sistema deve possuir instalação intuitiva e simplificada.
\end{itemize}

\subsection{Requisitos por Nível de Conectividade}

% TODO melhorar legibilidade da tabela
\begin{figure}[H]
	\centering
	\caption{Tabela de requisitos por nível de conectividade}
  \includegraphics[width=0.95\textwidth]{tabelaRequisitos}
\label{fig:tabelaRequisitos}
\end{figure}

\chapter{Arquitetura}
% TODO rever tudo nesse arquivo

\section{Visão geral}
% TODO

\section{Evolução arquitetural}
O processo de escolha para a arquitetura utilizada foi iterativo, e foram analisados os pontos fracos e as vantagens de cada nova sugestão.

A primeira versão proposta baseava-se unicamente em microsserviços, responsáveis por toda a inteligência do projeto, o que a fazia interessante do ponto de vista da escalabilidade para um número muito grande de casas. Com uma arquitetura fundamentalmente desenvolvida assim, também é possível utilizar quantas tecnologias forem necessárias ou desejáveis para cada um dos serviços sem efeitos colaterais nos outros, ou seja, transparentemente. Por outro lado, cria-se grande uma complexidade na integração entre todos os serviços disponíveis, que pode ser gerenciada por técnicas conhecidas e também explicadas aqui, como a coreografia e a orquestração. No entanto, há um aumento do \textit{overhead} para a comunicação, e os serviços do Hedwig necessitam de um meio rápido e robusto, que implemente qualidade de serviço para padrões diferentes de mensagens. Foi proposto um \textit{gateway} para os serviços da nuvem, por onde passaria toda a comunicação com a casa. A inserção do \textit{gateway}, no entanto, cria um ponto único de falha.

\begin{figure}[H]
	\centering
	\caption{Primeira versão da arquitetura do projeto Hedwig}
  \includegraphics[width=0.8\textwidth]{arquiteturaV1}
\label{fig:arquiteturaV1}
\end{figure}

É possível observar que alguns microsserviços são classificados como sensitivos, os quais dependem de nova consulta ao serviço de autenticação e autorização para garantir a segurança. Esses serviços são todos aqueles responsáveis por tomar uma ação em relação à casa que envolva riscos. Os microsserviços não-sensitivos utilizam a autenticação já realizada pelo \textit{gateway} na chegada da requisição.

Quando uma requisição chega à nuvem, ela deve ser autenticada, e caso passe nos critérios de autenticação e autorização, é retornado um JWT (\textit{JSON Web Token}), necessário para os passos seguintes. O JWT é discutido aqui, na seção MARCAR SEÇÃO AQUI. % TODO marcar seção

De extrema importância, e não cobertos pela arquitetura anterior, são os requisitos de disponibilidade do projeto. Se o \textit{gateway} estiver inacessível em determinado momento, a casa não terá mais nenhuma forma de comunicação com os meios externos, mesmo para os serviços mais básicos. Para resolver este problema, foi proposta uma segunda versão, conforme ilustra a imagem seguinte.

\begin{figure}[H]
	\centering
	\caption{Segunda versão da arquitetura do projeto Hedwig}
  \includegraphics[width=0.8\textwidth]{arquiteturaV2}
\label{fig:arquiteturaV2}
\end{figure}

Nesta versão, serviços essenciais seriam duplicados dentro da casa, e, no caso de haver qualquer forma de impedimento na comunicação com a nuvem, esses serviços seriam responsáveis por controlar diretamente os atuadores desejados. Entretanto, cria-se mais uma complexidade ao manter serviços duplicados na casa, e, no caso destes serviços não estarem online no momento necessário, também não seriam alcançados requisitos mais fortes de disponibilidade. Contudo, é uma versão que chega mais próximo de obedecer às necessidades do projeto.

Essa arquitetura provê módulos sem inteligência, e todo o controle é feito pelo serviço correspondente. Ao mesmo tempo, essa escolha tem benefícios como a escalabilidade, a manutenção (já que é muito mais simples atualizar o software de um ponto único, sempre que necessário) e a facilidade para prover correções ou possíveis aumentos de funcionalidade. Porém, não ficaríamos livres, mais uma vez, do ponto único de falha. Outro ponto é que alguns módulos ficariam em lugares de difícil acesso, ou mesmo fora da casa, onde a comunicação poderia ser perdida ou ser intermitente. Assim, em caso de falha de comunicação, um atuador não receberia os sinais necessários do serviço, acarretando em sérios problemas de segurança. No caso de uma garagem, por exemplo, o portão permaneceria aberto indeterminadamente, ou poderia não ser aberto o morador chegasse em csa.

Assim, começamos o desenvolvimento de um modelo arquitetural modularizado, onde cada módulo teria inteligência para realizar as tarefas necessárias e, ao mesmo tempo, podendo enviar dados à nuvem e ser avisado quando deve realizar uma tarefa. Com isso, em um aspecto também comercial, módulos inteiros poderiam ser vendidos, substituídos e aumentados.

A arquitetura escolhida também faz uso de microsserviços no lado da nuvem, e, no lado da casa, os componentes de hardware passam a ser agrupados em módulos independentes, com responsabilidades bem estabelecidas, inteligência para fazer todas as atividades necessárias, e com comunicação a um servidor local, que realizará, por último, a comunicação direta com os serviços não locais. Esse servidor se comunicaria com módulos por meio de mensagens enviadas em tópicos, as quais seriam interpretadas e enviadas aos servidores remotos. Se a casa perder comunicação com a nuvem, o servidor local armazenará as mensagens, que serão enviadas posteriormente. Essas mensagens, no caso, seriam de dados, advindas de sensores em módulos. Como não há urgência para o processamento de tais dados, os quais serão utilizados para análise de comportamento e Machine Learning, não há prejuízo com o eventual envio tardio.

Quando a comunicação entre o servidor local e a nuvem for perdida, os aplicativos web ou \textit{mobile}, poderão se comunicar diretamente com o servidor local da casa, para acessar uma quantidade mais restrita e essencial de ações - como, por exemplo, a liberação de acesso à casa. Além disso, no caso de perda de comunicação tanto com a nuvem como com o servidor local, os aplicativos poderão se comunicar diretamente com os módulos para terem acesso aos serviços de extrema importância.

Por ser escolhida, essa arquitetura será extensivamente detalhada e discutida aqui, com seus benefícios e limitações.


\section{Criação de Módulos}
Para a criação dos módulos de hardware, foram escolhidos componentes de \wiot{} comerciais, que possuem preços acessíveis, ampla documentação disponível e uma comunidade de desenvolvedores crescente.

A interconexão dos componentes, bem como a comunicação com o mundo externo pela internet será intermediada por um servidor local, instalado e disponível na plataforma \emph{Raspberry Pi}, rodando um sistema operacional Linux (\emph{Raspbian}, baseado em \emph{Debian}) e que dispõe da interface de hardware necessária para conexão com a rede.

Os sensores e atuadores devem ser conectados fisicamente com um módulo controlador, e para que essa limitação fosse contornada, foram utilizados dispositivos \emph{ESP8266} --- subseção \ref{subsec:esp8266} --- para transmissão sem fio por meio de \emph{Wi-Fi}. Esses módulos serão responsáveis pela transmissão das informações recebidas para o servidor local. Toda a arquitetura para essa transmissão será detalhada mais à frente. Os outros dispositivos a serem utilizados, como sensores \emph{DHT11}, \emph{LM555}, etc. podem ser vistos em uma lista completa no Anexo \ref{listamateriais}.

Em geral, os módulos consistem do microcontrolador, relés, sensores e fontes\slash{}conversores de tensão a depender do módulo, além de um circuito para manutenção corretiva baseado no astável 555, conectados à rede \emph{Wi-Fi} ou trabalhando como pontos de acesso. Para casos de falha de conexão, há um algoritmo de novas tentativas com tempos progressivamente maiores conforme as falhas ocorrerem, que busca deixar o módulo disponível para outras funções enquanto o serviço não está disponível. Para mitigar o travamento, um sinal de \textit{keep alive} é monitorado, e um circuito anti-travamento deve ativar o \textit{hard reset} (reset por hardware), ou então uma rotina de \textit{soft reset} deve ser acionada, de modo que os requisitos \emph{RNF-5} e \emph{RF-9} sejam cumpridos. No entanto, observa-se que a segunda alternativa é a mais natural de se implementar, mas menos robusta, já que ainda pode não funcionar em casos de loop infinito.

\subsection{Módulos Base}
\subsubsection{ESP8266 \label{subsec:esp8266}}
O ESP8266 é um microprocessador com baixo consumo e radiotransmisor com conexão \emph{Wi-Fi 802.11} integrada \cite{espressif}. Pode ser programado usando a \emph{Arduino IDE}, vastamente utilizada \cite{thomsen}. Opera com uma tensão de 3.3 V, suporta \emph{WPA} e possui modo de interrupção somente por software. É amplamente usado como \textit{shield} para conexão \emph{Wi-Fi} de placas de desenvolvimento da plataforma Arduino. Contudo, no projeto Hedwig, o dispositivo será utilizado em modo \textit{StandAlone} como principal processador e responsável pela conexão dos diferentes módulos de automação. Suas duas principais plataformas de desenvolvimento são \emph{Wemos}\footnote{https://www.wemos.cc/} e \emph{NodeMCU}\footnote{http://nodemcu.com/}. O projeto utilizará o \emph{Wemos D1 Mini}, versão compacta da \emph{Wemos D1 R2}.

O \emph{ESP8266} possui um modo de operação de baixa potência (\textit{sleep mode}) em que o consumo de bateria fica muito menor --- em contrapartida, o número de funcionalidades é limitado. Pode-se utilizar 7 portas de E\slash{}S digitais e uma porta de entrada analógica. Duas portas não são acessíveis, pois são utilizadas para programação e outras tarefas do sistema integrado do \emph{ESP8266}. Alternativas para extensão de portas são:

\begin{enumerate}
	\item Utilização três níveis de sinal análogico para detectar três tipos de acionamento, através de um circuito dedicado, com priorização de entrada;

	\item Utilização interface I2C, como o usado para o display;

	\item Utilização de radiofrequência, por meio de um par receptor-transmissor integrado no módulo, controles, atuadores e sensores sem fio.
\end{enumerate}

\subsection{Módulo de Acesso}

Buscando garantir mais segurança e comodidade para o acesso à residência, além do controle de abertura, o módulo de acesso atua em paralelo com uma fechadura eletrônica acionada por meio de controle remoto, que utiliza ondas de rádio para envio de dados. Assim, mesmo com falha total do sistema, o usuário poderá abrir o portão diretamente, sem a necessidade de acesso à Internet.

\begin{figure}[H]
	\centering
	\caption{Diagrama ilustrativo do módulo de Acesso ao Portão}
  \includegraphics[width=0.8\textwidth]{diagramaModuloAcesso}
\label{fig:diagramaModuloAcesso}
\end{figure}

A Figura \ref{fig:diagramaModuloAcesso} ilustra, em vermelho, dispositivos já existentes no mercado, como o controle remoto. O sensor e a sirene sem fio adicionais são mostrados em verde --- dispositivos externos ao módulo, que se comunicam por ondas de rádio. O próprio módulo de acesso, com um buzzer embutido, e sua conexão com a rede local Wi-Fi ou sua conexão direta com o celular (quando o módulo opera como um ponto de acesso de rede), em amarelo. As funcionalidades adicionais são marcadas em roxo.

A comodidade, no exemplo em questão, está em abrir o portão por meio do celular, ao utilizar o aplicativo web ou o aplicativo local (de emergência), sem a necessidade de carregar uma chave ou controle.

Entretanto, é necessário que a realização do controle de acesso seja feita de maneira segura. Assim, é empregado um algoritmo de rotação de teclas, para evitar que pessoas mal intencionadas possam:

\begin{enumerate}
	\item Olhar e copiar a senha que o usuário digita em seu celular;
	\item \label{alt:manInTheMiddle} Copiar os dados de abertura e usá-los mais tarde (\textit{middle man}).
\end{enumerate}

Na alternativa alternativa \ref{alt:manInTheMiddle}, a cada acesso, um novo mapeamento de teclas é gerado e enviado ao usuário. Mesmo que haja cópia das credenciais, ela não funcionará devido ao mapeamento ter mudado. Observe ainda que a fechadura eletrônica, por si só, já estava vulnerável a este tipo de ataque --- há, inclusive, dispositivos copiadores de senhas comercializados.

Outro aspecto de segurança é a preocupação dos usuários em esquecer a porta ou portão abertos. Para mitigar esse perigo, o módulo deve monitorar, por meio de um sensor, o estado vigente (aberto/fechado), conforme o requisito \emph{RF-1}, e alertar localmente (por meio de \textit{buzzer}) e remotamente (e.g. por email ou notificação no \textit{smartphone}) o usuário, conforme o requisito \emph{RF-2}. Essa e outras configurações (como de rede) são acessadas por uma senha diferente daquela de abertura, de modo que a interface básica seja simples para uso.

Para o caso de falha de envio de notificação (e.g. servidor fora do ar, ou indisponibilidade na conexão), há um algoritmo de novas tentativas com tempos progressivamente maiores conforme as falhas ocorrerem, buscando deixar o módulo disponível para outras funções. Tratamento análogo é realizado no servidor local, e no sistema de mensageria, de modo a evitar perdas de mensagens, mesmo em situações desfavoráveis. Para o caso de falta de conexão à internet, o módulo não seria controlável pela nuvem, com o aplicativo web, mas sim com o aplicativo emergencial, com a ativação do \textit{Access Point}, desenvolvido para operar diretamente com os módulos, sem intermédio do servidor local e dos serviços remotos.

Por meio das credenciais disponíveis no sistema, é possível saber qual dos usuários que solicitou a abertura do portão. A persistência destes acessos pode ser analisada e, utilizando-se técnicas de \emph{Machine Learning}, perfis de acesso podem ser determinados, e evoluir até o sistema saber quando houver um acesso em horário inesperado e notificar o usuário remotamente, confome o requisito \emph{RF-5}. O aprendizado de máquina é fundamental aqui para descobrir comportamentos que podem ser entendidos como suspeitos. Um exemplo prático de caso de uso seria um usuário que costuma chegar em um horário semelhante todos os dias, e realizar certo conjunto de tarefas na casa. Uma tentativa de acesso que não se enquadre em tais padrões pode ser produto de atividade suspeita, a qual pode ser informada pela casa para uma central, que acionaria a polícia caso não seja um falso positivo.

\subsection{Módulo de Quarto/Sala/Cozinha}
Um dos módulos com muitas opções de implementação e uso é o módulo de quarto, pois também pode ser usado no controle de iluminação para corredores, salas e ambientes externos.

\begin{figure}[H]
	\centering
	\caption{Diagrama ilustrativo do módulo de Quarto/Sala/Cozinha}
	\includegraphics[width=0.8\textwidth]{diagramaModuloQuarto}
	\label{fig:diagramaModuloQuarto}
\end{figure}

O diagrama ilustra equipamentos externos ao sistema em vermelho (lâmpada e abajur), enquanto o controle e sensor de abertura possuem comunicação sem fio. Em roxo, representa-se os equipamentos opcionais.

Como principais funcionalidades, tem-se o despertador (configurado pelo usuário, que também pode receber recomendações baseadas na informação gerada pelo monitoramento de seus ciclos de sono); monitoramento de temperatura e umidade do ambiente (que podem ser notificadas ao usuário, caso informem valores fora de determinados intervalos); controle de iluminação (da luz direta, que é a lâmpada central do ambiente, com maior potência, e da luz indireta, que é usualmente um abajur ou uma lâmpada com menor potência, usada para leitura); e estado da janela, para verificar remotamente se a janela está fechada ou não (por notificação ou visualização no aplicativo, útil para dias chuvosos). Além disso, se o módulo for instalado em ambientes internos e externos, o usuário pode usufruir de dados de temperatura e umidade, que podem ser usados para escolha de vestimenta, uso de guarda-chuva na ída para o trabalho ou se é mais vantajoso deixar roupas secando dentro ou fora de casa, e em que períodos.

O módulo de quarto pode ser acoplado ao sistema existente (fisicamente, é instalado no mesmo lugar do interruptor), e possui estados para o despertador. No estado inicial, somente a luz indireta é ligada. Após determinado tempo (programável pelo usuário), há avisos sonoros periódicos. No terceiro estado, os períodos são menores. Finalmente, no quarto estado a luz direta é ligada e os avisos sonoros são ininterruptos. Até o terceiro estado, o alarme pode ser desarmado (apertar duas vezes) ou entrar em estado soneca (apertar única vez) diretamente no módulo. Já no estado 4, a critério anterior do usuário, o alarme pode ser desarmado somente fisicamente em outro módulo presente em um segundo aposento --- por exemplo, na sala. Esse módulo pode variar de dia para dia, caso o usuário assim desejar. O sistema desarma o alarme após 40 minutos.

O display possui iluminação automática para não aparentar brilho muito intenso quando todas as luzes estiverem desligadas (por meio de circuito baseado em \emph{LDR}). Para o controle da iluminação, há diferentes tempos para desligamento. Por exemplo, quando ocorre controle manual (pelo botão presente no módulo), o tempo pode ser maior. Já pelo modo automático, quando a luz já foi ligada pelo próprio módulo, o tempo para desligamento pode ser menor (por exemplo, 4 minutos).

Com o monitoramento da presença, há um \emph{preset} da contagem para desligamento sempre que houver presença detectada, de forma a inibir acionamentos desnecessários do relé. Outra aplicação para o monitoramento da presença é a descoberta de comportamento anormal. Por exemplo, se o usuário sempre toma café entre 8 e 10 horas, e não apresentar presença na casa até às 15 horas, o sistema pode notificar emails de parentes cadastrados.

\subsection{Módulo de Aquário}

Devido a altos custos de compra, implantação e manutenção de um aquário, que pode ser de água doce ou salgada, e até abrigar espécies raras, é desejável que uma série de riscos sejam mitigados. Dentre tais riscos, destacam-se:

\begin{table}[hbp]
		\centering
		\caption{Riscos para o Aquário}
		\resizebox{\textwidth}{!}{%
		\begin{tabular}{cp{8cm}p{8cm}}
			\toprule
			\textbf{Perigo/Necessidade} 					& \textbf{Origem} 																			& \textbf{Consequência}  \\
			\midrule
			Aquecimento acidental 							& Ajuste errado da temperatura do termostato												& Superaquecimento; risco de mortes (peixes e plantas) \\
			Falta de água 									& Vazamento ou evaporação natural															& Mal funcionamento ou queima da bomba submersa (à longo prazo, falta de oxigenação da água) \\
			Falta de circulação de água 					& Entupimento do tubo de circulação ou mal funcionamento da bomba							& Falta de oxigenação da água, ocasionando em risco de mortes (peixes) \\
			Iluminação adequada								& Existência de plantas e/ou iluminação natural insuficiente no ambiente do aquário			& Ambiente nocivo para os peixes, risco de morte das plantas (principalmente durante períodos de esquecimento/viagens) \\
			\bottomrule
	\end{tabular}}
	\label{table:riscosaquario}
\end{table}

Sobre o caso específico da residência onde foram executados os testes de campo, considere um aquário com 50 litros de água, com uma bomba reserva (não a responsável pela circulação de água, que não muda o volume interno) e um aquecedor, com muitos peixes dentro, que são sensíveis à variação brusca de temperatura.

\begin{figure}[H]
	\centering
	\caption{Aquário de Santo André}
	\includegraphics[width=0.8\textwidth]{aquario}
	\label{fig:aquario}
\end{figure}

Na ocasião de troca de água do aquário, cerca de 20\% do volume total é substituído. A saída de água se dá por um funil (simplificado, para se obter uma vazão de saída constante), e a adição é realizada pela bomba auxiliar, que possui água limpa, sem cloro e com certeza menos amônia e outros compostos nocivos aos peixes (que justificam essa troca periódica de água). O monitoramento do nível da água pode também mitigar o risco de esquecimento --- o que levaria o nível da água a tender a zero.

\begin{figure}[H]
	\centering
	\caption{Diagrama ilustrativo do módulo de Aquário}
	\includegraphics[width=0.8\textwidth]{diagramaAquario}
	\label{fig:diagramaAquario}
\end{figure}

Para mitigar os riscos descritos anteriormente, foi desenvolvido o módulo de aquário, que permite:

\begin{enumerate}
	\item Controle de horários em que a lâmpada fica acesa, fornecendo uma iluminação adequada para as plantas e peixes do aquário em períodos curtos de viagem;
	\item Monitoramento do nível de água do aquário principal, alerta pelo aplicativo quando não estiver no nível esperado (pode ocorrer por muita evaporação, vazamento ou problema com a bomba);
	\item Monitoramento da temperatura do aquário, e bloqueamento do aquecedor caso a água já esteja numa temperatura desejável (evitar superaquecimento devido a mal funcionamento do termostato), feito por meio de ligação em série com o termostato do aquário;
	\item Nos casos de perigo acima descritos, alerta sonoro também localmente (por meio do buzzer).
	
\end{enumerate}

\subsection{Módulo de Interface com Sistema de Alarmes}

O módulo de interface com o sistema de alarmes monitora dois setores específicos, um relativo a presença, que possui sensores no corredor e na sala da residência, e outro relativo à porta da sala (sensor único na porta da residência). O módulo implementado foi instalado em uma residência em Jarinu - SP.

\begin{figure}[H]
	\centering
	\caption{Diagrama ilustrativo do módulo de Interface com Sistema de Alarmes}
	\includegraphics[width=0.8\textwidth]{diagramaAlarme}
	\label{fig:diagramaAlarme}
\end{figure}

Um problema recorrente é a conexão com a internet, que apesar da indicação de estado válido e conectado, não fornecia acesso à sites, tampouco acesso externo por meio de abertura de porta no roteador. Para contornar esse problema, e permitir que o módulo esteja disponível para coleta de dados e persistência em cartão SD, foi instalado também um relê \emph{NF} (normalmente fechado) em série com a alimentação do roteador. Em caso do módulo estar desligado, o roteador ficará ligado.

Já quando há erro persistente (maior que 10 vezes em intervalos de 2 a 3 minutos), ao realizar a operação de \emph{ping} com sites ou servidores conhecidos, o módulo reinicia a conexão, atuando diretamente no roteador. Ocorre a execução deste procedimento em intervalos cada vez mais espaçados, de forma a não executar muitas vezes o reinício do roteador sem que a conexão seja reestabelecida com sucesso (nas primeiras tentativas, tem-se um intervalo de 3 minutos até a nova tentativa, a partir da terceira vez um intervalo de 10 minutos, e assim por diante).

Outra dificuldade encontrada em sua instalação em campo foi o fato de obtenção de endereço IP dinâmico (onde outros dispositivos, tais como celulares e tablets também obtinham endereços IP dinâmicos), gerando indisponibilidades do módulo. Com a mudança do endereço IP para fixo, em outro intervalo de endereçamento, o módulo passou a ter alta disponibilidade, ficando até semanas sem reiniciar.


\section{Controlador local}
Para a intercomunicação entre os módulos e a nuvem, há a presença do servidor local, Morpheus, responsável por introduzir mais uma camada de segurança na troca de mensagens. Para isso, foi desenvolvida uma plataforma, com a utilização de sistemas de mensageria, e definido um protocolo de comunicação entre os serviços de nuvem e os módulos. Assim, quando um usuário realiza determinada operação por meio do cliente web, uma mensagem é enviada, interpretada pelo servidor local e, em seguida, encaminhada para o destino por meio do protocolo \wmqtt{} com o broker Mosquitto. O Morpheus é visto em detalhe na Seção \ref{chap:morpheus}.

\subsection{Raspberry Pi}
O Raspberry Pi é um computador integrado em um único chip, do tamanho de um cartão de crédito. Foi desenvolvido com o objetivo de promover o ensino de computação básica, e possui funcionalidades tais como um Computador Pessoal (PC): navegação na Internet, reprodução de video, processamento de texto, dentre outros. No projeto, será utilizado como servidor local (gerenciador de módulos local da casa), exatamente pelas funcionalidades compatíveis com a de um computador desktop.

A versão 3 possui uma CPU 1.2 Ghz 64-bit quad-core ARMv8, conexão 802.11n Wireless LAN, Bluetooth 4.1, suporte a Bluetooth Low Energy (BLE), 1GB RAM, 4 portas USB, 40 pinos GPIO, porta HDMI, porta Ethernet, interface para câmera, display e cartão SD. Para projetos que necessitem de baixo consumo energético, os modelos mais indicados são Pi Zero ou A+ \cite{raspPi}.

\begin{figure}[H]
	\centering
	\caption{Raspberry Pi 3 Modelo B}
  \includegraphics[width=0.5\textwidth]{Raspberry-Pi-3}
	\caption*{Fonte: \cite{raspPi}}
\label{fig:Raspberry-Pi-3}
\end{figure}


\section{Servidor na Nuvem}
Como foi escolhida uma arquitetura baseada microsserviços para construção do projeto, módulos diferentes podem ser escritos em linguagens de programação diferentes, o que promove uma maior flexibilidade não só durante o desenvolvimento dos módulos de mostrados nesse trabalho, mas também daqueles projetados futuramente como extensões do sistema Hedwig.

Para o desenvolvimento dos módulos definidos na especificação do projeto, utilizamos tecnologias atuais que são utilizadas em grandes empresas de tecnologia do mundo e possuem vasta documentação, referências e fontes de conhecimento como tutoriais e exemplos.
Para o desenvolvimento da parte de software, utilizaremos tecnologias atuais, que são também utilizadas nas maiores empresas de tecnologia do mundo. De acordo com o planejamento, utilizaremos uma arquitetura de microsserviços para construção do projeto. Com esta técnica, módulos diferentes poderiam ser escritos em, inclusive, linguagens de programação diferentes, o que promove uma maior flexibilidade durante o desenvolvimento.
Para o desenvolvimento da API, responsável pelos módulos sendo executados na nuvem e comunicação com banco de dados, utilizaremos Node.js (interpretador de código JavaScript do lado do servidor), com a utilização de alguns frameworks como é o caso do Express. O banco de dados com a qual ela se conecta é do tipo MongoDB (banco de dados orientado a documentos). Tais decisões foram baseadas no fato de bancos de dados em MongoDB serem altamente escaláveis e flexíveis, assim como Node.js, que, por sua arquitetura movida a eventos de E/S que não bloqueiam o servidor, provê ao mesmo uma altíssima escalabilidade, ao permitir milhares de conexões simultâneas, sem impacto na performance do servidor. Além disso, o fato de que os dados provindos do banco já estão organizados em objetos, e dessa forma, podem ser recebidos prontamente como objetos JavaScript no código em Node.Js geram facilidade e fluidez para o desenvolvimento do código.

\subsection{Arquitetura de Microsserviços}
\subsubsection{Características}
A arquitetura de microsserviços é um estilo que compreende a estruturação de uma aplicação em um conjunto de serviços com baixo grau de acoplamento que se comunicam por meio de protocolos de comunicação leves.

Para melhor compreender essa arquitetura, podemos compará-la à arquitetura monolítica. Uma aplicação monolítica está contida em uma única unidade, que geralmente é dividida em camadas de funcionalidade tecnológica como interface web, camada de negócios server-side e camada de persistência de dados. A escalabilidade desse modelo é dada por meio do aumento do número de servidores, máquinas virtuais ou contêineres juntamente a um load balancer - é a chamada escalabilidade horizontal. Uma alteração em uma pequena parte da aplicação significa que toda a aplicação deverá passar por um processo de \textit{build} e \textit{deploy}. Já a arquitetura de microsserviços divide as funcionalidades em serviços autônomos, muitas vezes usando as regras de negócios para realizar essa divisão. Cada serviço tem seu próprio ciclo de desenvolvimento e pode ser atualizado independentemente. A escalabilidade também é tratada serviço a serviço.

\begin{figure}[H]
	\centering
	\caption{Comparação entre uma aplicação monolítica (esquerda) e com microsserviços (direita)}
  \includegraphics[width=0.8\textwidth]{estruturaMicrosservicos}
\label{fig:estruturaMicrosservicos}
\end{figure}

É difícil delimitar uma definição formal para arquitetura de microsserviços, pois não existe consenso a respeito de sua definição formal. Contudo, existe uma série de características que projetos usando essa arquitetura compartilham. Detalhamos a seguir alguns atributos e aspectos dos microsserviços. Nem todos os projetos possuem rigorosamente todas as características, mas a maioria deles possui um perfil similar ao descrito aqui.

\begin{itemize}
\item \textbf{Serviços são processos.}Pode-se fazer um mapeamento de um processo para um serviço, porém isso é apenas uma aproximação, podendo um serviço ser constituído por uma aplicação de múltiplos processos.
\item \textbf{Serviços comunicam-se por protocolos leves.}Geralmente, são usados protocolos como o HTTP.
\item \textbf{Serviços implementam capabilidades do negócio.}Isto é, a divisão de serviços é baseada nas regras de negócio e nas funcionalidades que o produto deverá suprir.
\item \textbf{Serviços são facilmente substituíveis.}
\item \textbf{Cada serviço tem um ciclo de vida independente.}Isso inclui o desenvolvimento e os processos de \textit{deploy}. Um microsserviço pode ser implementado e atualizado independentemente dos outros.
\end{itemize}

As vantagens da arquitetura de microsserviços giram em torno da modularidade e autonomia dos serviços que é natural à sua estrutura. Com isso, pode-se ter uma heterogeneidade de tecnologias, isto é, cada serviço pode ser desenvolvido usando diferentes linguagens, \textit{frameworks} e ferramentas de acordo com seus requisitos. A independência entre serviços também possibilita o deploy automatizado e o uso de práticas de integração contínua. Também há benefícios de aspecto gerencial: como cada serviço tem como escopo uma capabilidade do negócio que envolve interfaces de interação com usuário, código em várias camadas que implementa as funcionalidades necessárias e persistência em bancos de dados, é possível criar pequenas equipes multidisciplinares para cada microsserviço.

Building for failure % TODO

Existem trade-offs que devem ser considerados ao decidir pela arquitetura de microsserviços. A comunicação entre serviços por meio de uma rede possui maior latência e exige maior processamento do que mensagens trocadas a nível de processos. Por isso, é muito importante analisar as fronteiras dos serviços e a alocação de responsabilidades durante do projeto. A descentralização de dados entre microsserviços traz também a necessidade de métodos para manter a consistência das informações. Outro ponto crítico são sistemas com alta granularidade de microsserviços, causando overhead tanto de comunicação como de código além de uma fragmentação lógica que causa mais impactos negativos na complexidade e performance do que benefícios - tal caso de antipadrão foi chamado de nanosserviço \cite{rotem}.

% TODO melhorar paragrafo abaixo
Os microsserviços podem ser vistos como um estilo específico de arquitetura orientada a serviços (\textit{Service-oriented architecture} - SOA), visto que existem várias características compartilhadas entre os dois. Contudo, o termo arquitetura orientada a serviços é muito amplo, e muitas de suas implementações podem não seguir certos pontos apresentados como aspectos dos microsserviços, como por exemplo, o uso de grande inteligência no mecanismo de comunicação de dados ao invés de delegar tal complexidade aos endpoints do serviço \cite{james}. Esse e outros problemas conhecidos das experiências passadas de sistemas estruturados em SOA fazem com que muitos encarem os microsserviços como uma modernização da arquitetura orientada a serviços.

Apesar do termo microsserviço ter surgido por volta de 2011 \cite{james}, as ideias por trás desse estilo arquitetural não são recentes. O aumento da discussão em torno dos microsserviços nos últimos anos pode ser creditada a avanços tecnológicos tais como a disseminação dos serviços de nuvem, o crescimento de ferramentas de automatização de deployment, a consolidação dos conceitos de DevOps, entre outros.

\subsubsection{Casos de uso} % TODO

\subsubsection{Microsserviços e Internet das Coisas} % TODO


\section{Cliente Web}
Para criar aplicações web que demonstrem as funcionalidades dos módulos de automação da casa, foi escolhida a biblioteca de JavaScript React, que permite o fácil desenvolvimento de aplicações single-page, renderizadas do lado do cliente, e que permitem a atualização dinâmica da página, de forma fluida, rápida, o que acaba enriquecendo a experiência do usuário na aplicação (UI e UX). Esse cliente irá se comunicar com a API, por meio do protocolo HTTP, e utilizando autenticação de usuário por meio  de tokens do tipo JSON Web Token. JSON Web Tokens são tokens gerados no cadastro ou login do usuário, e são enviados ao browser, onde são armazenados na Localstorage do mesmo.

A partir desse momento, todas as requisições ao back end conterão tal token no campo de Authentication do cabeçalho dos métodos HTTP (GET, PUT, POST, DELETE). Somente requisições contendo tal token, e cujo token seja válido, são aceitas.

Outro ponto interessante para a utilização da biblioteca React é que, com a biblioteca React Native - uma extensão da biblioteca React - é possível a geração de aplicativos nativos para iOS e Android, que podem vir a ser desenvolvidos no desenrolar do projeto. Isso diminui a necessidade de retrabalho e dispensa a necessidade de estudo aprofundado das linguagens e ambientes de desenvolvimento tradicionais de projeto de aplicativos nativos.


\section{Comunicação}
Conforme explicado anteriormente, neste projeto utilizamos tanto protocolos de comunicação próprios quanto os elaborados comercialmente. A arquitetura desenvolvida aqui busca viabilizar a robustez do sistema, trabalhando em um nível local e outro nível remoto, onde o usuário terá o controle de sua casa por meio do \textit{smartphone} ou computador pessoal.

Nosso serviço em nuvem recebe as requisições do usuário por meio de um cliente web ou nativo. Esse servidor processa as requisições, aplicando os filtros de segurança necessários, de modo a consultar a autenticidade do pedido e verificar se aquele usuário possui as permissões necessárias para o serviço que deseja operar. Os serviços da nuvem se comunicam com o servidor local da casa requisitada, o qual também aplica os filtros de segurança necessários, e realiza a comunicação com os módulos.

A infraestrutura de comunicação entre a nuvem e o servidor local, e o servidor local e os sensores e atuadores utiliza o protocolo de aplicação \wmqtt{}, referência em aplicações \wiot{} no mundo. O protocolo \wmqtt{} é estabelecido em cima dos protocolos TCP/IP (nas camadas inferiores) e é orientado à sessão, diferentemente do protocolo HTTP, de mesma camada.

O protocolo \wmqtt{} é do tipo Pub/Sub (de \textit{publisher/subscriber}) e é estritamente orientado à tópicos. Assim, um \textit{subscriber} se inscreve a um tópico de seu interesse, e recebe todas as publicações que um \textit{publisher} realizar. Os tópicos são organizados com estrutura semelhante a de um sistema de arquivos Unix, com níveis hierárquicos separados por barras, de modo que o subscriber pode se inscrever para tópicos utilizando \textit{wildcards} (* e +, os quais são válidos para mais de um nível e um único nível, respectivamente).

Para interconectar os tópicos, com \textit{publishers} e \textit{subscribers}, é necessário um agente que realiza a transmissão das mensagens, e que garante a segurança e confiabilidade. Esse agente é conhecido como Broker (em versões anteriores) ou Server (na versão atual, V3.1.1). O \textit{broker} irá permitir ou negar a subscrição ou a publicação a determinado tópico.

A segurança da troca de mensagens é realizada por meio do protocolo TLS (\textit{Transport Layer Security}) que encripta os segmentos na camada de transporte. Toda a parte de segurança e criptografia será detalhada no momento oportuno, bem como a organização dos tópicos implementados. % TODO era bom falar em qual seção está isso

Além disso, o protocolo \wmqtt{} oferece três tipos de QoS (\textit{Quality of Service}), possibilitando: diminuir o overhead ao máximo, enviando a mensagem uma única vez, na configuração mais simples; garantir que a mensagem seja entregue no mínimo uma vez, na configuração de segundo nível; garantir que a mensagem seja entregue exatamente uma vez, no terceiro nível, o que aumenta o overhead, consequentemente.

As mensagens são transmitidas em texto puro, e é necessário estabelecer um protocolo para a sua utilização. Utilizaremos aqui o protocolo que define a configurações das mensagens, desenvolvido no projeto HomeSky.

O \textit{broker} Mosquitto\footnote{https://mosquitto.org/} será utilizado, e foi escolhido por ser amplamente adotado em projetos de \wiot{}, além de ser open source e com licença abrangente (MIT). Entretanto, há diversas possibilidades, como o HiveMQ, adotado no projeto HomeSky, e com grande uso em aplicações enterprise.

A arquitetura de comunicação é representada pelo diagrama abaixo, com um alto nível de abstração, cujos detalhes serão vistos no momento oportuno, com granularidade menor.

\begin{figure}[H]
	\centering
	\caption{Visão alto nível da comunicação no Hedwig}
  \includegraphics[width=0.8\textwidth]{arquiteturaHedwig}
\label{fig:diagramaComunicacao}
\end{figure}

\subsection{Entre módulos e controlador local}
\subsection{Entre controlador local e nuvem}
\subsection{Entre cliente web e nuvem}
\subsection{Entre app backup e módulos}


\chapter{Metodologia}

\section{Gerência do projeto}
% TODO Alguns dos itens abaixo podem/serão movidos para os apêndices
Para realizar a gerência do projeto Hedwig, foram usadas as diretrizes do Guia PMBOK \cite{pmi} e da norma ISO/IEC 12207 \cite{iso12207} como referência para coordenar os processos.

Para gerenciar as tarefas, estudos e pesquisas necessárias para a realização do projeto, foi utilizado o Trello\footnote{Pode ser acessado gratuitamente em https://trello.com/} - sistema online para organização de ideias e projetos, que permite listagem e acompanhamento de tarefas a serem realizadas, com deadlines, responsáveis e categorização em diversos tipos de tarefas.

% \subsection{Gerência de Tempo}

\subsection{Gerência de aquisição}

\subsection{Gerência de processos de software}

Para gerenciar o código-fonte e permitir o trabalho da equipe em múltiplas partes do projeto ao mesmo tempo, foi utilizado o Git, um sistema de controle de versão distribuído. Para publicação do código, foi escolhido o GitHub, onde está a organização do projeto Hedwig\footnote{https://github.com/hedwig-project} e os repositórios de código dos módulos associados ao sistema. A preferência pelo GitHub se deu pelas suas funcionalidades de gerenciamento e colaboração como a notificação de bugs, acompanhamento do progresso de tarefas e criação de wikis, além de ser uma plataforma conhecida por abrigar grandes projetos open-source que chegam a ter centenas ou milhares de contribuidores \cite{github}.

Para o fluxo de trabalho nesses repositórios, foi utilizado o fluxo conhecido como \textit{Feature Branch Workflow} \cite{atlassian}, caracterizado pela criação de \textit{branches} (ramificações) para o desenvolvimento de cada nova funcionalidade. Ao final do desenvolvimento de cada funcionalidade, é feito um pedido para mesclar o código desenvolvido em tal ramificação com o da ramificação principal (\textit{master branch}).

% \subsection{Gerência de Partes Interessadas}

\subsection{Gerência de comunicação}

% \subsection{Gerência de Escopo}

\subsection{Gerência de riscos}
% TODO colocar matriz de probabilidade e impacto?

\section{Pesquisa bibliográfica}

O estudo dos tópicos relacionados a aprendizagem de máquina foi realizado com auxílio do curso Aprendizagem Automática do Professor Andrew Ng\footnote{ https://www.coursera.org/learn/machine-learning}, oferecido pela Universidade de Stanford e disponibilizado no Coursera, uma plataforma de MOOCs (\textit{Massive Open Online Courses}) que oferece cursos abertos e especializações.

Os cursos da especialização em \textit{Data Science} da Universidade Johns Hopkins\footnote{ https://www.coursera.org/specializations/jhu-data-science}, também disponíveis no Coursera, foram usados como referência e treinamento para realizar a coleta de dados de maneira metódica. Por esse motivo, foi dada maior atenção ao curso \textit{Getting and Cleaning Data}. Contudo, também foi aproveitado conteúdo do curso \textit{Practical Machine Learning}.

\section{Ferramentas e tecnologias}

Para aprender a utilizar a biblioteca React para o desenvolvimento do front-end, foi usada como referência a documentação oficial\footnote{https://facebook.github.io/react/docs/hello-world.html} oferecida pelo Facebook e o curso \textit{React for Beginners} de Wes Bos\footnote{ https://reactforbeginners.com/}. O aprendizado de Redux foi auxiliado pelo curso \textit{Learn Redux}\footnote{https://learnredux.com}, do mesmo autor.

\chapter{Arquitetura: Tecnologia e Implementação}

\section{Módulos}
\subsection {Rotinas}
\subsubsection{Multiplexação no tempo}
Para tratar indisponibilidade dos módulos devido a tentativas de reconexão e conexão e requisições não gerenciadas, e aumentar a disponibilidade, além do circuito antitravamento e \textit{hard reset}, as diversas rotinas - desde configuração inicial, reconfigurações, coletas de dados, atuar por meio de relés, até conexão, desconexão, reconexão e envio de dados - foram multiplexadas no tempo da seguinte forma:

\begin{figure} [H]
	\centering
	\caption{Rotina de multiplexação de procedimentos no tempo}
  \includegraphics[width=0.8\textwidth]{rotinaMultiplexacao}
\label{fig:rotinaMultiplexacao}
\end{figure}

\subsubsection{Tratamento de indisponibilidade}
Nos casos de indisponibilidade de Internet, servidor ou rede local, o seguinte procedimento foi adotado (observe que a indisponibilidade do próprio módulo é tratada pelo circuito antitravamento):

\begin{figure}[H]
	\centering
	\caption{Tratamento de indisponibilidade de recursos}
  \includegraphics[width=0.8\textwidth]{tratamentoIndisponibilidade}
\label{fig:tratamentoIndisponibilidade}
\end{figure}

Com esse procedimento, as tentativas de reconexão à Internet, servidor e rede local estão segregadas e com tentativas realizadas em intervalos de tempo sucessivamente maiores. Desta forma, conseguimos gerenciar esses procedimentos, já que o nível de processamento é baixo.

\subsubsection{DoS local (\textit{Evil Twin})}
No caso de ataque de \textit{Evil Twin} - no qual uma rede mal intencionada, usualmente aberta, usa o mesmo SSID da rede original, com o objetivo de obter a senha - o sistema pode ficar indisponível até ao nível local. Módulos podem se conectar à rede mal-intencionada e ficarem somente com as funcionalidades offline, como acionamento de lâmpada por botão físico acoplado ao módulo. Outro problema é a queda da rede por interferência de radio frequência ou outro mecanismo utilizado pelo usuário mal intencionado para que os clientes se desconectem, tentem reconexão e forneçam a senha da rede.

Para mitigar esses riscos, os módulos executam o seguinte procedimento:

\begin{figure}[H]
	\centering
	\caption{Tratamento de ataque de DoS Local}
  \includegraphics[width=0.8\textwidth]{tratamentoDoS}
\label{fig:tratamentoDoS}
\end{figure}

\subsubsection{Comunicação por MQTT}
Para o desenvolvimento da comunicação por MQTT com o Morpheus no módulo, houve o uso da biblioteca PubSub para Arduino.

Para cada módulo, temos de fábrica id e senhas próprias, além de configurações de porta e endereço do morpheus locais padrões.

Houve o controle de comunicação da seguinte forma:

\begin{enumerate}
	\item Só há tentativa de conexão MQTT em caso de houver conexão do WiFi e o horário interno estar configurado (em caso contrário, podemos provocar travamento do módulo ou envio de mensagens sem timestamp);
	\item Configuração de callback e servidor MQTT a cada reconexão (este ponto foi crítico para o bom funcionamento da comunicação);
	\item No callback (recebimento de mensagens, tratamento (“parse”) da mensagem recebida, através de obtenção de seu tipo e redirecionamento para rotinas específicas para obtenção dos parâmetros de interesse de cada mensagem;
	\item Envio de mensagens de estado a cada 1 minuto, ou quando houver mudança brusca em um dos sensores (presença, abertura, umidade ou temperatura) ou então requisição de atuação (nesses casos de envio rápido, a mensagem é inserida no início da fila de saída, e enviada logo em seguida, em até um segundo);
	\item Após o tratamento inicial das mensagens (obtenção dos parâmetros de interesse), persistência nas variáveis internas e gravação da EEPROM para que configurações e estados executados no aplicativo backup ou pela dashboard sejam refletidos dos dois lados, tornando transparente ao usuário o uso de qualquer um dos aplicativos, e para que suas funcionalidades estejam integradas;
	\item Ainda, em fase final, após a persistência de variáveis na EEPROM, a inclusão na fila de saída de mensagens de confirmação para o servidor local;
	\item Uso da fila de saída para o envio periódico das mensagens de estado.
\end{enumerate}

\begin{figure}[H]
	\centering
	\caption{Exemplo de estado da fila de saída de mensagens MQTT do módulo}
	\includegraphics[width=0.5\textwidth]{filasaidaMQTT}
	\label{fig:filasaidaMQTT}
\end{figure}

Mensagens de estado periódicas (azuis) são inseridas no final da fila.
No caso de acionamento, uma mensagem de estado (vermelha) é inserida no início da fila, e em caso de configuração, uma mensagem de confirmação também é inserida no início da fila.

Por fim, vale destacar a necessidade de alteração da biblioteca PubSub para suportar mensagens maiores (em caso de impossibilidade de envio de mensagens maiores, todo o protocolo deveria ser alterado para comportar mensagens mais compactas ou então deveria haver o uso de algum mecanismo de codificação/compactação em conjunto com o protocolo desenvolvido).

\subsection{Diagrama}
Abaixo está o diagrama do circuito impresso (PCB).

\begin{figure}[H]
	\centering
	\caption{Diagrama PCB do Módulo Base}
  \includegraphics[width=0.8\textwidth]{diagramaModuloBase}
\label{fig:diagramaModuloBase}
\end{figure}

\begin{enumerate}
\item Wemos D1 mini
\item Astável 555
\item Fonte 5V 3W
\item \textit{Buzzer}
\item Relé 1
\item Relé 2
\item Hard Reset
\item Botões
\item Presença
\item RF-RX
\item RF-TX
\end{enumerate}

\begin{figure}[H]
	\centering
	\caption{Diagrama elétrico do Módulo Base}
  \includegraphics[width=0.8\textwidth]{esquematicoModuloBase}
\label{fig:esquematicoModuloBase}
\end{figure}

\begin{description}
\item [A - Saídas] Circuitos simples de transistor para acionamento de relés (para lâmpadas) e buzzer.
\item [Proteção 3V3 5V] Como o display trabalha com tensão de 5V, há proteções com diodos para não danificar as entradas digitais do Wemos D1 mini, que trabalha com tensão de 3V3.
\item [3 Entradas em A0] O circuito tem como entradas um botão (para acionamento do relé 1), o LDR (para chaveamento do backlight do display), e um outro botão para hard reset do dispositivo, todos numa entrada analógica, cujo mapeamento E/S é da seguinte forma:

\begin{figure}[H]
	\centering
	\caption{Entradas em A0}
  \includegraphics[width=0.4\textwidth]{entradasEmA0}
\label{fig:entradasEmA0}
\end{figure}

\item [Presença ou RF-TX] A entrada digital D6 é usada exclusivamente como entrada do sensor de presença PIR ou receptor RF.
\item [Astável 555 para Hard Reset e Botão] A porta D6 é usada como LED \textit{keep alive} do módulo. Sua demora ao piscar indica que o módulo está travado ou demorando muito para processar algo, o que não deveria acontecer, uma vez que os procedimentos estão multiplexados no tempo, de acordo com seus tempos limite. Dessa forma, conectamos essa saída a um circuito antitravamento, que executa o reset nos casos mencionados, de travamento ou \textit{timeout}.

O primeiro capacitor tem como objetivo desacoplamento DC, de forma que a entrada do circuito envolvendo o astável 555 seja somente AC. Assim, travamentos em 0 ou 1 indicam travamento.

Enquanto o LED pisca em intervalos esperados (regularmente), o transistor conduz e mantém uma saída dente de serra muito próxima de 0. Quando o módulo trava, o transistor não conduz mais, e a saída passa a oscilar entre 1/3 e 2/3 da tensão total (observe que o carregamento é feito pelo resistor de 4M7, muito maior que o resistor de 100k, fazendo com que o tempo de carga seja muito maior que o tempo de descarga, uma vez que esses tempos são diretamente proporcionais à constante de tempo dos circuitos RC, que é dada pelo produto do R*C). Durante a descarga, o reset da placa é realizado. Observe que os tempos foram ajustados pelos valores dos componentes discretos, para que o tempo entre resets sucessivos seja menor que o tempo necessário para o módulo voltar a funcionar após um reset.

Segue abaixo uma ilustração sobre o funcionamento do circuito.

\begin{figure}[H]
	\centering
	\caption{Funcionamento do circuito antitravamento}
  \includegraphics[width=0.8\textwidth]{circuitoAntiTravamento}
\label{fig:circuitoAntiTravamento}
\end{figure}

\item [DHT11] Entrada D3 é ligada a uma montagem básica para leitura de umidade e temperatura através do periférico DHT11.

\end{description}

\subsection {Montagem}

As fotos e comentários seguintes descrevem o processo de montagem física dos quatro módulos usados neste trabalho. Além destes, outras versões também foram construídas anteriormente no decorrer do projeto, instaladas na residência de um dos membros do grupo e usados para coleta de dados (que, por se tratarem de protótipos, não tem sua montagem completamente documentada tampouco, uniformidade como os módulos seguintes).
Ao final, também consta o procedimento utilizado para validação dos módulos após sua construção, que contribuiu para a identificação de ligações não realizadas e outros problemas de montagem.

\begin{figure}[H]
	\centering
	\caption{Evolução do hardware (de fevereiro/17 a setembro/17)}
	\includegraphics[width=0.8\textwidth]{evolHW}
	\label{fig:evolHW}
\end{figure}

\begin{figure}[H]
	\centering
	\caption{Evolução da caixa de proteção e display}
	\includegraphics[width=0.8\textwidth]{evolProtDisplay}
	\label{fig:evolProtDisplay}
\end{figure}

\begin{figure}[H]
	\centering
	\caption{Materiais e preparação da placa padrão}
	\includegraphics[width=0.8\textwidth]{materiaisPrepPlaca}
	\label{fig:materiaisPrepPlaca}
\end{figure}

\begin{enumerate}
	\item Primeiramente, a partir do esquemático em escala real, foram cortados e realizados furos na placa padrão, conforme figura acima.

	\begin{figure}[H]
		\centering
		\caption{Posicionamento dos componentes}
		\includegraphics[width=0.8\textwidth]{PosicionamentoComp}
		\label{fig:PosicionamentoComp}
	\end{figure}

	\item Em seguida, foram posicionados os componentes, conforme mostra a figura acima.

	\begin{figure}[H]
		\centering
		\caption{Preparação dos displays com o I2C e soldagem}
		\includegraphics[width=0.8\textwidth]{PrepI2CSoldagem}
		\label{fig:PrepI2CSoldagem}
	\end{figure}

	\item A próxima etapa é soldar as trilhas por baixo conforme o diagrama. Também é necessário soldar o I2C com o display. Essa é a etapa mais demorada, que poderia ser facilmente operacionalizada ao adotarmos placas de circuito impresso, o que aumentaria muito a capacidade de montagem.

	\begin{figure}[H]
		\centering
		\caption{Preparação das caixas para os módulos}
		\includegraphics[width=0.8\textwidth]{PrepCaixasModulos}
		\label{fig:PrepCaixasModulos}
	\end{figure}

	\item Prossegue-se para a marcação das caixas para uso das furadeiras e lixas (com ferramentas mais adequadas, essa etapa também poderia ser mais rápida e eficiente).

	\begin{figure}[H]
		\centering
		\caption{Caixa protetora, ligação dos botões e cabos de força}
		\includegraphics[width=0.8\textwidth]{BotoesCabosForca}
		\label{fig:BotoesCabosForca}
	\end{figure}

	\item Com a caixa preparada, foi possível inserir as placas padrão com os componentes e trilhas. Resta fixá-los com parafusos, montar a sustentação do display e sensor de presença (também integrado nesse passo), isopor para isolar termicamente o medidor de temperatura e umidade DHT, além das ligações dos botões, cabos de força e fios dos relés.

	\begin{figure}[H]
		\centering
		\caption{Quatro módulos prontos}
		\includegraphics[width=0.8\textwidth]{QuatroModulos}
		\label{fig:QuatroModulos}
	\end{figure}

\end{enumerate}

\subsubsection {Montagem}
\begin{enumerate}
	\item Carregar Programa para testar módulo;
	\item Realizar Setup da Conexão com sua rede WiFi;
	\item Verificar com o multímetro se há curto em algumas ligações principais (terra, VCC);
	\item Checar a alimentação da fonte e sua saída correta;
	\item Realizar o Hard Reset ao apertar o botão atrás do isopor do DHT até ouvir 10 beeps
	\item Fazer o passo 2 novamente (pois o módulo deve ter voltado à versão de fábrica). Logo em seguida fazer o teste de Auto Reset, que é o teste para verificar se o circuito antitravamento está funcionando. Para esse teste, é simulado uma pausa do sinal keep alive que o circuito baseado no astável 555 monitora;
	\item Cobrir o sensor de presença. Verificar a inatividade no aplicativo. Descobrí-lo e verificar atividade;
	\item Executar o passo 7 para o sensor de luminosidade (LDR) também;
	\item Verificar com um medidor externo ou consulta a um site de previsões do tempo se as medidas de temperatura e umidade estão de acordo. Realizar ajuste (offset) no aplicativo se necessário;
	\item Verifique o funcionamento dos botões, acionando-os um por um;
	\item Checar o acionamento dos reles pelo aplicativo web também;
	\item Após gravar o RF de um controle para os dois relés, testar seu funcionamento;
	\item Verifique com um multímetro a saída dos relés (se troca de nível com acionamento pela página web, botões e controle RF).
\end{enumerate}

\section{Controlador Local - Morpheus \label{chap:morpheus}}

\subsection{Descrição}
Morpheus é o servidor local responsável pela interconexão da casa inteligente com os serviços de nuvem. O nome tem sua origem na mitologia grega, cujo Deus dos sonhos, Morpheus, era responsável pelo envio de mensagens entre dois mundos diferentes, o dos deuses e o dos mortais \cite{morpheusName}. A principal atribuição do servidor local é garantir que a troca de mensagem entre os módulos e a nuvem seja realizada com segurança e confiabilidade, munindo-se de soluções robustas para desempenhar o seu papel.

\subsection{Plataforma}
O Morpheus tem seu desenvolvimento realizado em Java. Conforme será detalhado em seguida, tal escolha foi realizada com base na portabilidade que a máquina virtual Java (JVM) oferece, bem como na disponibilidade de bibliotecas e serviços largamente utilizados em aplicações comerciais. O servidor foi construído utilizando-se o Spring Boot Framework.

Para se comunicar com os módulos, o Morpheus utiliza-se da conexão com um \emph{broker} \wmqtt{}. O \emph{broker} Mosquitto foi utilizado por ser uma solução \emph{open-source} largamente utilizada em projetos de \wiot. Conforme detalhado a frente, configurações de segurança específicas para o projeto foram registradas no \emph{broker}. Para a conexão com os serviços na nuvem, é utilizado um canal WebSocket aberto pelo Morpheus (cliente) e aceito pela nuvem (servidor). Esta solução veio a partir de uma discussão em relação à segurança, relativa ao requisito não-funcional RNF-6, conforme documentada na Subseção \ref{sub:websocket}.

\subsection{Tecnologias Utilizadas}
Toda a implementação do Morpheus foi realizada na linguagem Java. Desde o começo do projeto, decidiu-se que a escolha de tecnologias para implementação das diversas camadas deveria ter por base os seus benefícios, e não necessitaria ser rígida ou uniforme. Assim, o principal esforço foi sempre no planejamento das interfaces de comunicação entre as partes, que poderiam ser implementadas em linguagens complementamente diferentes. Os sistemas de nuvem, por exemplo, foram implementados em Node.js. O aplicativo web também em foi feito em JavaScript, com utilização da biblioteca React. Os módulos de hardware foram programados em linguagem \emph{C-like}, própria para Arduino, e o controlador local em Java. Essa flexibilidade permitiu a utilização de recursos e tecnologias que fossem melhor integrados com os requisitos propostos.

Um requisito essencial para o controlador local é a sua robustez (RNF-4). Em um cenário em que este controlador não esteja disponível, a casa passa a funcionar em estado de emergência, no qual os comandos são reduzidos e não permitem acesso remoto. Entretanto, há inúmeras possibilidades e eventos que poderiam causar a queda deste controlador, muitas das quais referem-se a situações fora de nosso alcance. Por exemplo, a falta de energia ou de Internet na residência interrompe o seu funcionamento, não sendo possível ter controle sobre tal situação. O mesmo ocorre no evento de problemas de hardware na plataforma que o sistema estiver rodando. Os planos para contenção dos seus efeitos de tais situações são complexos, custosos e fogem do escopo deste projeto, como seria o caso de implementar duplicações, banco de baterias e tecnologia celular para comunicação secundária.

Há, entretanto, problemas no software que poderiam afetar o funcionamento do controlador. Por meio de testes, muitos desses problemas podem ser evitados ainda em tempo de desenvolvimento. A utilização de tecnologias que facilitam o desenvolvimento seguro da aplicação é uma vantagem para este caso, já que ferramentas estão disponíveis para que haja maior controle sobre o código desenvolvido, e pode-se detectar erros mais facilmente, ainda em tempo de compilação, por exemplo.

O controlador local também precisa lidar com as requisições assincronamente, conforme o requisito não-funcional RNF-10. Parte desta tarefa é facilitada com a utilização do sistema de mensageria \wmqtt{}, operado pelo \emph{broker} Mosquitto. Com sua utilização, mensagens podem ser enviadas mesmo que o controlador não consiga recebê-las, pois elas não serão perdidas. Entretanto, devido às características do sistema proposto, as mensagens precisam ser operadas sem maiores demoras. O controlador deve receber e processar as mensagens paralelamente, e não esperar o processamento de uma mensagem inteira para processar a próxima, de modo que o paralelismo deve ser parte essencial da arquitetura.

Ainda, para a integração com os serviços da nuvem, é necessário a utilização de JSON, para a serialização das mensagens, em um formato que pode ser desserializado posteriormente, independentemente da plataforma. Para a utilização de WebSockets, é necessário o uso de bibliotecas disponíveis, de modo que o desenvolvimento seja facilitado. Por último, é necessário gerenciar eficientemente todas essas dependências. Atualizá-las quando necessário, ou substituí-las, se desejado, deve ser uma tarefa simples.

A arquitetura oferecida pelo Java mostra ser efetiva para as necessidades levantadas acima. Com a utilização de uma IDE avançada, inúmeros recursos estão disponíveis para limpeza, refatoração, organização do código, etc. É uma linguagem utilizada em vasta gama de aplicações, desde complexos softwares comerciais como a IDE Eclipse, até softwares embutidos, como controladores de BlueRay \cite{javaBlueray}. A escolha do Java 8 foi decidida para que o desenvolvimento possa utilizar certos recursos de paradigmas funcionais, como o conceito de Streams de dados e Funções Lambdas.

O \emph{framework} Spring Boot\footnote{https://projects.spring.io/spring-boot/} foi utilizado para o desenvolvimento por oferecer diversos recursos facilitadores, configurações de ambiente e um \emph{container} para inversão de controle (\emph{IoC - Inversion of Control}) e injeção de dependência. Essa técnica diminui o acoplamento entre classes e permite a evolução e implementação de novas funcionalidades de maneira fácil \cite{iocFowler}. Assim, cada módulo recebe em seu construtor todas as dependências que serão utilizadas. A responsabilidade da construção de tais dependências passa, então, a ser responsabilidade do gerenciador de contexto, e não mais do módulo.

Além disso, o gerenciador de dependências Gradle\footnote{https://gradle.org/} também foi utilizado por oferecer um poderoso ambiente para configurar, construir e distribuir aplicações. Gradle faz uso de Groovy\footnote{http://groovy-lang.org/}, tecnologia que também roda na Java Virtual Machine (JVM). Por outro lado, o uso de tais ferramentas e plataformas necessita de hardware mais robusto para que funcione, sendo uma desvantagem. Contudo, frente aos benefícios, ainda é vantajosa a utilização de Java neste caso.

Internamente, o Morpheus é dividido em pacotes, que são responsáveis pela modelagem do problema. Há classes que modelam o domínio, que executam as regras de negócio, que fazem a interface entre outros sistemas (\wmqtt{} Mosquitto Broker e nuvem), e que fazem a execução de tarefas como backup de mensagens, e serviços de conversão.
Assim que uma mensagem chega ao Morpheus, ela é reconhecida e é realizado o seu \emph{parsing} para as estruturas de domínio internas. Caso haja algum erro nas mensagens vindas da nuvem, há o envio de relatório com os problemas encontrados. A partir do reconhecimento, a mensagem é colocada em uma fila interna, onde outro módulo será responsável por capturá-la e realizar o processamento necessário.

\begin{figure}
	\centering
	\caption{Arquitetura do servidor local}
  \includegraphics[width=\textwidth]{diagramaDeComunicacao}
\label{fig:diagramaDeComunicacao}
\end{figure}

\subsection{Características e Recursos}
Para a concepção do servidor local, foram considerados os requisitos funcionais e não-funcionais, discutidos na Seção \ref{sec:requisitos}. As características e recursos da implementação são discutidos em seguida.

\begin{description}

\item \textbf{Configuração dos módulos físicos}

De acordo com as regras e interfaces estabelecidas, os módulos podem ser configurados por meio de mensagens. Os serviços da nuvem enviam os parâmetros de configuração de cada módulo ao Morpheus, que os transmitirá ao módulo.

\item \textbf{Envio de dados para a nuvem}

Dados provenientes de sensores são enviados para a nuvem, para que possam ser tratados de acordo com as regras de \emph{Business Intelligence} e utilizados em algoritmos de aprendizado de máquina.

\item \textbf{Persistência de dados}

Quando não houver conexão, o servidor armazena os dados localmente e, quando solicitado, os envia à nuvem.

\item \textbf{Tentativas de reenvio}

Quando uma mensagem não é enviada com sucesso à nuvem, o Morpheus tenta novamente por um número configurável de vezes em um curto espaço de tempo. Isso ocorre porque, se determinada mensagem não pode ser enviada em uma janela temporal, ela perde o seu sentido (e.g. requisição de abertura de portão).

\item \textbf{Verificação do \emph{timestamp}}

Quando uma nova mensagem chegar, seu \emph{timestamp} é verificado, e a mensagem tomará curso somente se não for obsoleta.

\item \textbf{Tomadas de ação}

Quando o usuário requisitar uma tomada de ação, esta deve ser enviada por meio de uma mensagem ao Morpheus, por onde será transmitida ao módulo.

\item \textbf{Configuração em arquivo}

As configurações básicas do Morpheus devem ser registradas em um arquivo YAML que é lido durante a inicialização.

\item \textbf{Listeners para diferentes tipos de mensagens}

Devem haver \emph{listeners} para todos os tipos de mensagens que são recebidos da nuvem e dos módulos.

\item \textbf{Processamento concorrente}

Toda a infraestrutura do Morpheus permite o processamento concorrente de mensagens. Não é necessário esperar o processamento completo de uma mensagem para que outra comece a ser processada.

\item \textbf{Utilização de criptografia na troca de mensagens com a nuvem}

Os dados que trafegam entre a nuvem e o servidor local são encriptados na camada de transporte por meio de \emph{TLS}.

\item \textbf{Conversão de mensagens}

As mensagens enviadas à nuvem são codificadas em JSON após serem convertidas do formato interno, que se refere apenas à troca de mensagens entre os módulos e o Morpheus.

\item \textbf{Serialização das configurações}

O servidor serializa e persiste as configurações relativas aos módulos que foram configurados para carregá-las em sua inicialização.

\item \textbf{Destruição de pools de threads}

Ao ser desligado, todos os \emph{pools} de \emph{threads} criados são destruídos.

\end{description}

\subsection{Protocolo para Troca de Mensagens com Módulos}

Toda a comunicação entre as partes do projeto é realizada por troca de mensagens. Foi desenvolvido um protocolo específico, leve e expansível, para a codificação dessas mensagens. As subseções seguintes definem e exemplificam o uso do protocolo.

\subsubsection{Tópicos}
Todos os tópicos devem seguir o formato especificado em seguida. Com essa formatação, é possível garantir que:

\begin{enumerate}
\item Somente o Morpheus consegue publicar em qualquer tópico ou ser um \emph{subscriber} de qualquer tópico.
\item Cada módulo somente consigue publicar no tópico determinado para ele, o que é garantido com as credenciais (usuário e senha) fornecidos pelo tópico.
\item Caso um módulo malicioso seja implantado com o roubo das credenciais de um módulo legítimo, o impacto será unicamente concentrado naquele tópico, não atingindo outros módulos.
\end{enumerate}

Têm-se as seguintes regras:

\textbf{hw/\textless ID do módulo\textgreater /s2m}

\emph{Server to Module} --- o módulo deve ser \emph{subscriber} desse tópico. O servidor deve ser \emph{publisher} desse tópico.

\textbf{hw/\textless ID do módulo\textgreater /m2s}

\emph{Module to Server} --- o servidor deve ser \emph{subscriber} desse tópico. O módulo deve ser \emph{publisher} desse tópico.

\subsubsection{Regras de Negócio}
O servidor foi desenvolvido com base nas regras de negócio seguintes. Cada regra de negócio tem sua identificação dada por \emph{mRN}, seguida de um identificador numérico --- e.g. \emph{mRN-1}.
\begin{description}
\item[mRN-1:]Após a compra de cada módulo, o usuário deve registrar online a aquisição. O servidor da nuvem enviará para o servidor local da casa a requisição para configurar o módulo.
\item[mRN-2:]Cada módulo envia mensagens para o servidor local com o seus dados por meio do \wmqtt{}.
\item[mRN-3:]Para a troca de senha do Wi-Fi, o usuário cadastra no aplicativo a nova senha. O servidor na nuvem faz a requisição para o servidor local, o qual enviará um arquivo de configuração com a nova senha para cada um dos módulos registrados. Após a configuração de todos os módulos, o servidor local envia resposta de sucesso para a nuvem, a qual indica ao usuário que a troca de senha já pode ser feita com sucesso.
\item[mRN-4:]Todo módulo sai de fábrica configurado com o tópico que deve se inscrever e publicar com base no seu ID, o qual será o seu usuário, também havendo uma senha para se autenticar junto ao \emph{broker} \wmqtt{}.
\end{description}

\subsubsection{Definição de Interfaces}
\begin{itemize}
\item Há três tipos de mensagens que vão do Morpheus para os módulos:
  \begin{itemize}
  \item Configuração (\texttt{configuration})
  \item Requisição de ação (\texttt{action\_request})
  \item Requisição de dados (\texttt{data\_transmission})
  \end{itemize}
\item Há três tipos de mensagens que chegam dos módulos:
  \begin{itemize}
  \item Confirmação (\texttt{confirmation})
  \item Envio de dados (\texttt{data\_transmission})
  \item Requisição de dados (\texttt{data\_request})
  \end{itemize}
\end{itemize}

\subsubsection{Definição das Mensagens}

\paragraph{Configuração (\texttt{configuration})}
\begin{itemize}
\item Sentido: Morpheus para módulo.
\item Uso: envio de parâmetros para configuração dos módulos.
\end{itemize}

\textbf{Configuração de hora}
\begin{lstlisting}
#configuration
$ts: <timestamp>
$ty: time_config
@
updated_ntp: <segundos desde 00h00 de 1 de Janeiro de 1970, 64 bits>
@
\end{lstlisting}

\textbf{Configuração de nome}
\begin{lstlisting}
#configuration
$ts: <timestamp>
$ty: name_config
@
new_name: <string do nome>
new_rele1name: <string do nome | "">
new_rele2name: <string do nome | "">
@
\end{lstlisting}

\textbf{Configuração de comunicação}
\begin{lstlisting}
#configuration
$ts: <timestamp>
$ty: comunication_config
@
new_ssid: <novo ssid>
new_password: <nova senha>
ip_local: <novo ip local fixo>
ap_mod: <"sempre ativo" | "automatico">
ap_name: <nome do ap para acesso direto>
ap_password: <senha do ap para acesso direto>
@
\end{lstlisting}

\textbf{Configuração de RF}
\begin{lstlisting}
#configuration
$ts: <timestamp>
$ty: rf_config
@
<nome do sensor | controle | funcao>: <"store" | "clear" | "keep">
@
\end{lstlisting}

\textbf{Configuração de display}
\begin{lstlisting}
#configuration
$ts: <timestamp>
$ty: display_config
@
displaytype: <1 | 2 | 3>
backlight: <0 | 1 | 2>
@
\end{lstlisting}
0 = desligado, 1 = ligado, 2 = automático

\paragraph{Requisição de ação (\texttt{action\_request})}
\begin{itemize}
\item Sentido: Morpheus para módulo.
\item Uso: quando um usuário faz a requisição de uma ação por meio do aplicativo. Por exemplo, quando deseja-se acender uma luz, o aplicativo envia uma requisição para o Morpheus, que enviará uma mensagem de \texttt{action\_request} para o módulo correspondente.
\end{itemize}

\textbf{Requisição de acionamento}
\begin{lstlisting}
#action_request
$ts: <timestamp>
$ty: rele1_action
@
rele1: <0 | 1>
@
\end{lstlisting}
Essa mensagem fica sem efeito no caso do Módulo de Acesso, que usa mensagens com senha para o acionamento do relé 1.

Rele1: 0 = desligar; 1 = ligar

\begin{lstlisting}
#action_request
$ts: <timestamp>
$ty: rele2_action
@
rele2: <0 | 1>
@
\end{lstlisting}
Rele2: 0 = desligar; 1 = ligar

\textbf{Requisição de reinício de software}
\begin{lstlisting}
#action_request
$ts: <timestamp>
$ty: sw_reset
@
swreset: <0 | 1>
@
\end{lstlisting}
0 = não; 1 = confirmar reinício

\textbf{Requisição de teste de \emph{auto reset}}
\begin{lstlisting}
#action_request
$ts: <timestamp>
$ty: autoreset_test
@
autoreset: <0 | 1>
@
\end{lstlisting}
0 = não; 1 = confirmar reinício

\paragraph{Confirmação (\texttt{confirmation})}
\begin{itemize}
\item Sentido: do módulo para o servidor.
\item Uso: confirmação de uma configuração ou requisição de ação vindas do servidor.
\end{itemize}

\textbf{Confirmação de hora}
\begin{lstlisting}
#confirmation
$ts: <timestamp>
$ty: time_confirm
@
ntp: <segundos desde 00h00 de 1 de Janeiro de 1970, 64 bits>
@
\end{lstlisting}

\textbf{Confirmação de nome}
\begin{lstlisting}
#confirmation
$ts: <timestamp>
$ty: name_confirm
@
name: <nome>
rele1name: <string do nome | "">
rele2name: <string do nome | "">
@
\end{lstlisting}

\textbf{Confirmação de comunicação}
\begin{lstlisting}
#confirmation
$ts: <timestamp>
$ty: communication_confirm
@
ssid: <novo ssid>
password: <nova senha>
ip local: <novo ip local fixo>
ap mod: <"sempre ativo" | "automatico">
ap name: <nome do ap para acesso direto>
ap password: <senha do ap para acesso direto>
@
\end{lstlisting}

\textbf{Confirmação de configuração de RF}
\begin{lstlisting}
#confirmation
$ts: <timestamp>
$ty: rf_confirm
@
<nome do sensor | nome do controle | nome do funcao>: <valor_gravado>
@
\end{lstlisting}

\textbf{Configuração de configuração de display}
\begin{lstlisting}
#confirmation
$ts: <timestamp>
$ty: display_confirm
@
displaytype: <1 | 2 | 3>
backlight: <0 | 1>
@
\end{lstlisting}

\textbf{Confirmação de reinício de software}
\begin{lstlisting}
#confirmation
$ts: <timestamp>
$ty: sw_reset_confirm
@
swreset: <0 | 1>
@
\end{lstlisting}

\textbf{Confirmação de teste de \emph{auto reset}}
\begin{lstlisting}
#confirmation
$ts: <timestamp>
$ty: autoreset_test_confirm
@
autoreset: <0 | 1>
@
\end{lstlisting}

\paragraph{Transmissão de dados (\texttt{data\_transmission})}
\begin{itemize}
\item Sentido: do módulo para o servidor.
\item Uso: envio de dados de sensores para o servidor.
\end{itemize}

\textbf{Transmissão de umidade, temperatura, presença, relés, sensor de presença e luminosidade}
\begin{lstlisting}
#data_transmission
$ts: <timestamp>
$ty: temp_umi_pres
@
s1: umidade
vl1: <value>
s2: temperatura
vl2: <value>
s3: presenca
vl3: <value>
s4: rl1
vl4: <value>
s5: rl2
vl5: <value>
s6: abertura
vl6: <0|1>
s7: luz
vl7: <int de 0 a 1000>
@
\end{lstlisting}

\paragraph{Requisição de dados (\texttt{data\_request})}
\begin{itemize}
\item Sentido: do módulo para o servidor.
\item Protocolo: \wmqtt{}
\item Uso: requisição de alguma informação do servidor (ex.: atualização de hora).
\end{itemize}


\textbf{Mensagens próprias para o Módulo de Acesso}


\textbf{Configuração de alarme}
\begin{lstlisting}
#configuration
$ts: <timestamp>
$ty: alarm_config
@
alarme: <0|1>
alarme_tempo: <int do tempo em segundos>
@
\end{lstlisting}

\textbf{Configuração de luz automática}
\begin{lstlisting}
#configuration
$ts: <timestamp>
$ty: auto1_config
@
initial_time1: <integer de 0 a 23>
final_time1: <integer de 0 a 23>
time_keepon1: <tempo em minutos>
time_deslmanual1: <tempo em minutos>
@
\end{lstlisting}

\begin{lstlisting}
#configuration
$ts: <timestamp>
$ty: auto2_config
@
initial_time2: <integer de 0 a 23>
final_time2: <integer de 0 a 23>
time_keepon2: <tempo em minutos>
time_deslmanual2: <tempo em minutos>
@
\end{lstlisting}

\textbf{Configuração de senha}
\begin{lstlisting}
#configuration
$ts: <timestamp>
$ty: password_config
@
old_password: <string>
new_password: <string>
@
\end{lstlisting}

\textbf{Abertura de portão}
\begin{lstlisting}
#action_request
$ts: <timestamp>
$ty: abertura_portao
@
password: <string>
@
\end{lstlisting}

\textbf{Confirmação de alarme}
\begin{lstlisting}
#confirmation
$ts: <timestamp>
$ty: alarm_confirm
@
alarme: <0|1>
alarme_tempo: <tempo em minutos>
@
\end{lstlisting}

\textbf{Confirmação de configuracão de luz automática}
\begin{lstlisting}
#confirmation
$ts: <timestamp>
$ty: auto1_confirm
@
initial_time1: <integer de 0 a 23>
final_time1: <integer de 0 a 23>
time_keepon1: <tempo em minutos>
time_deslmanual1: <tempo em minutos>
@
\end{lstlisting}

\begin{lstlisting}
#confirmation
$ts: <timestamp>
$ty: auto2_confirm
@
initial_time2: <integer de 0 a 23>
final_time2: <integer de 0 a 23>
time_keepon2: <tempo em minutos>
time_deslmanual2: <tempo em minutos>
@
\end{lstlisting}

\textbf{Confirmação de senha}
\begin{lstlisting}
#confirmation
$ts: <timestamp>
$ty: password_confirm
@
password: <string>
@
\end{lstlisting}

\textbf{Transmissão de estado de acesso}
\begin{lstlisting}
#data_transmission
$ts: <timestamp>
$ty: acesso_estado
@
s1: abertura
vl: <0 fechado | 1 aberto>
s2: alarme
vl: <0 desligado | 1 ligado>
s3: tempo_alarme
vl: <value>
@
\end{lstlisting}

tempo\_alarme: tempo em minutos em que o sensor de abertura está aberto.

\textbf{Mensagens próprias para o Módulo do Quarto}

\textbf{Configuração de despertador}
\begin{lstlisting}
#configuration
$ts: <timestamp>
$ty: despertador_config
@
despertador: <0 | 1>
despertador_tempo: <tempo em minutos>
@
\end{lstlisting}

\textbf{Configuração de luz automática}
\begin{lstlisting}
#configuration
$ts: <timestamp>
$ty: acesso_config
@
initial_time: <integer de 0 a 23>
final_time: <integer de 0 a 23>
time_keepon: <tempo em minutos>
time_deslmanual: <tempo em minutos>
@
\end{lstlisting}

\textbf{Confirmação de despertador}
\begin{lstlisting}
#confirmation
$ts: <timestamp>
$ty: despertador_config
@
despertador: <0 | 1>
despertador_tempo: <tempo em minutos>
@
\end{lstlisting}

\textbf{Confirmação de luz automática}
\begin{lstlisting}
$ts: <timestamp>
$ty: acesso_config
@
initial_time: <integer de 0 a 23>
final_time: <integer de 0 a 23>
time_keepon: <tempo em minutos>
time_deslmanual: <tempo em minutos>
\end{lstlisting}

\textbf{Mensagens próprias para o Módulo do Externo}

\textbf{Configuração de alerta de 1 (água) e 2 (energia elétrica)}
\begin{lstlisting}
#configuration
$ts: <timestamp>
$ty: offset_config
@
alerta1: <0 | 1>
alerta1_nivel: <vl>
alerta2: <0 | 1>
alerta2_nivel: <vl>
@
\end{lstlisting}

\textbf{Confirmação de configuração de alerta de 1 (água) e 2 (energia elétrica)}
\begin{lstlisting}
#confirmation
$ts: <timestamp>
$ty: offset_config
@
alerta1: <0 | 1>
alerta1_nivel: <vl>
alerta2: <0 | 1>
alerta2_nivel: <vl>
@
\end{lstlisting}

\textbf{Transmissão de estado de Módulo Externo}
\begin{lstlisting}
#data_transmission
$ts: <timestamp>
$ty: externo_estado
@
s1: agua
vl: <integer>
s2: energia_eletrica
vl: <integer>
s1: agua_alerta
vl: <0 | 1>
s2: energia_eletrica_alerta
vl: <0 | 1>
@
\end{lstlisting}

\textbf{Mensagens próprias para o Módulo da Cozinha}

\textbf{Programação para preparo (café ou arroz)}
\begin{lstlisting}
#configuration
$ts: <timestamp>
$ty: offset_config
@
initialtime: <vl>
finaltime: <vl>
cooking_time: <vl>
cook_mode: <"auto" | "manual">
@
\end{lstlisting}

\textbf{Confirmação de programação para preparo (café ou arroz)}
\begin{lstlisting}
#confirmation
$ts: <timestamp>
$ty: offset_config
@
initialtime: <vl>
finaltime: <vl>
cooking_time: <vl>
cook_mode: <"auto" | "manual">
@
\end{lstlisting}

\textbf{Transmissão de estado do Módulo da Cozinha}
\begin{lstlisting}
#data_transmission
$ts: <timestamp>
$ty: cozinha_estado
@
s1: gas
vl: <integer>
s2: cooking
vl: <0 | 1>
@
\end{lstlisting}

\textbf{Requisição de offset para alarme de gás}
\begin{lstlisting}
#data_transmission
$ts: <timestamp>
$ty: gas_offset
@
<vl>
@
\end{lstlisting}

\subsubsection{Testes realizados da Comunicação Morpheus e Módulos}

Para que fosse simulado o envio de mensagens, o aplicativo \wmqtt{} Fx\footnote{http://mqttfx.jensd.de/} foi utilizado. Com o uso deste software, é possível se inscrever em determinado tópico, enviar as credenciais para o \emph{broker}, tanto em forma de usuário e senha, quando em forma de certificados, além de publicar no tópico desejado.

\textbf{Requisição de acionamento 1}
\begin{lstlisting}
#action_request
$ts:<timestamp>
$ty: rele1_action
@
rele1: 0
@
\end{lstlisting}

\emph{Esperado: 0 no serial do Arduino, indicando recebimento.}

\emph{Resultado: de acordo.}

\textbf{Requisição de acionamento 2}
\begin{lstlisting}
#action request
$ts: <timestamp>
$ty: rele2_action
@
rele2: 1
@
\end{lstlisting}

\emph{Esperado: 1 no serial do Arduino, indicando recebimento.}

\emph{Resultado: de acordo.}

\textbf{Requisição e confirmação de reinício de software}
\begin{lstlisting}
#action_request
$ts: <timestamp>
$ty: sw_reset
@
swreset: 1
@
\end{lstlisting}

\emph{Esperado: confirmação de SW Restart no tópico \wmqtt{} m2s.}

\emph{Resultado: de acordo.}

\textbf{Requisição e confirmação de teste de \emph{auto reset}}
\begin{lstlisting}
#action request
$ts: <timestamp>
$ty: autoreset_test
@
autoreset: 1
@
\end{lstlisting}

\emph{Esperado: confirmação no tópico \wmqtt{} m2s.}

\emph{Resultado: de acordo.}

\textbf{Configuração e confirmação de hora}
\begin{lstlisting}
#configuration
$ts: 293029
$ty: time_config
@
updated_ntp: 293029
@
\end{lstlisting}

\emph{Esperado: confirmação no tópico \wmqtt{} m2s.}

\emph{Resultado: de acordo.}

\textbf{Configuração e confirmação de nome}
\begin{lstlisting}
#configuration
$ts: 432524
$ty: name_config
@
new_name: NovoNome
new_rele1name: Portal1
new_rele2name: Portal2
@
\end{lstlisting}

\emph{Esperado: confirmação no tópico \wmqtt{} m2s.}

\emph{Resultado: de acordo.}

\textbf{Configuração e confirmação de comunicação}
\begin{lstlisting}
#configuration
$ts: 5349545
$ty: comunication_config
@
new_ssid: Novossid
new_password: novaSenha
ip_local: 192.168.0.32
ap_mod: automatico
ap_name: AcessoDiretoAP
ap_password: 1234
@
\end{lstlisting}

\emph{Esperado: confirmação no tópico \wmqtt{} m2s.}

\emph{Resultado: de acordo.}

\textbf{Configuração e confirmação de RF}
\begin{lstlisting}
#configuration
$ts: 4839434
$ty: rf_config
@
Janela4: 01234
@
\end{lstlisting}

\emph{Esperado: confirmação no tópico \wmqtt{} m2s.}

\emph{Resultado: de acordo.}

\textbf{Configuração e confirmação de display}
\begin{lstlisting}
#configuration
$ts: 543242
$ty: display_config
@
displaytype: 369
backlight: 1
@
\end{lstlisting}

\emph{Esperado: confirmação no tópico \wmqtt{} m2s.}

\emph{Resultado: de acordo.}

\textbf{Transmissão de umidade, temperatura, presença e relés}
\begin{lstlisting}
messageToSend = UmiTempPresReles(0,80,25,1,1,0);
\end{lstlisting}

\emph{Esperado: mensagem no tópico \wmqtt{} m2s.}

\emph{Resultado: de acordo.}

\subsection{Protocolo para Troca de Mensagens com a Nuvem}
A troca de mensagens entre Morpheus e servidor na nuvem ocorre por meio de WebSockets. Como explicado anteriormente, caso o servidor local provesse uma API REST aberta à conexões externas, seriam necessárias configurações, software e hardware avançados para a proteção, o que seria inviável para o propósito em questão. Além disso, usuários domésticos possuem endereços IP dinâmicos, de maneira que seria necessária a atualização desse endereço na nuvem a cada mudança. Com a utilização do WebSocket, o servidor local é um cliente que solicita a abertura de uma conexão com a nuvem, que é mantida aberta a partir daí.

A comunicação com a nuvem é baseada em eventos, que são recebidos e enviados com os dados relevantes. Os eventos estão descritos a seguir.

\subsubsection{Morpheus}
O Morpheus ouve os seguintes eventos vindos da nuvem:

\begin{itemize}
\item \texttt{configuration}
\item \texttt{action}
\item \texttt{data} (requisitar informações sobre módulo, e.g. se portão está aberto ou não).
\end{itemize}

\subsubsection{Nuvem}
A nuvem ouve os seguintes eventos vindos do Morpheus:

\begin{itemize}
\item \texttt{confirmation}
\item \texttt{configuration}
\item \texttt{data}
\end{itemize}

Definição de mensagens entre Nuvem e Morpheus
\begin{lstlisting}
configuration =
{
    "configurationId": <configurationId> ,
    "timestamp": <timestamp> ,
    "morpheusConfiguration": <morpheusConfiguration> ,
    "modulesConfiguration": <modulesConfiguration>
}
\end{lstlisting}

\begin{lstlisting}
<morpheusConfiguration>  =
{
    "register": [<eachModuleRegistration> ],
    "requestSendingPersistedMessages": <true | false>
}
\end{lstlisting}

\begin{lstlisting}
<eachModuleRegistration>  =
{
    "moduleId": <moduleId> ,
    "moduleName": <moduleName> ,
    "moduleTopic": <moduleTopic> ,
    "receiveMessagesAtMostEvery": <time> ,
    "qos": <qosLevel>
 }
 \end{lstlisting}


\textbf{Requisitos}

O campo receiveMessagesAtMostEvery deve estar no formato“\textless time\textgreater :\textless unit\textgreater ”
A unidade deve ser “s” para segundos, “m” para minutos ou “h” para horas. O valor padrão é 60 segundos.

Ex.: requisição de mensagens persistidas e configuração do Morpheus.

\textbf{Configuração de módulo}

A seção de configuração de módulo é um objeto com duas partes. A primeira identifica o módulo dentro do Morpheus, e a segunda envia as mensagens que serão interpretadas pelo módulo.
\begin{lstlisting}
{
    "moduleId": <moduleId> ,
    "moduleName": <moduleName> ,
    "moduleTopic": <moduleTopic> ,
    "unregister": <true|false>,
    "messages": [<message>]
}
\end{lstlisting}

\begin{lstlisting}
<message> =
"controlParameters":
{
    "parameter": <name> ,
    "value": <value>
},
"payload": {
    [
        <key>: <value>
    ]
}
\end{lstlisting}

\textbf{Requisição de ação}
As mensagens de \texttt{action} seguem o mesmo protocolo estabelecido anteriormente para mensagens de \texttt{action\_request}.

\textbf{Transmissão de dados}
As mensagens de \texttt{data} também seguem o mesmo protocolo estabelecido anteriormente para mensagens de \texttt{data\_transmission}.

\textbf{Exemplo de mensagem}
A seguir, um exemplo de uma mensagem completa de configuração.

\begin{lstlisting}
{
    "configurationId": "1",
    "timestamp": "1499717103422",
    "morpheusConfiguration": {},
    "modulesConfiguration": [
        {
            "moduleId": "00765914",
            "moduleName": "teste_wemos",
            "moduleTopic": "hw/00765914",
            "unregister": true
    	},
    	{
            "moduleId": "000281D0",
            "moduleName": "hugo_basic",
            "moduleTopic": "hw/000281D0",
            "unregister": false,
            "messages": [
                {
                    "controlParameters": [
                        {
                            "parameter": "ts",
                            "value": 1500914158
                        },{
                            "parameter": "ty",
                            "value": "rele1_action"
                        }
                    ],
                    "payload": {
                        "v1": 5,
                        "v2": "auto"
                    }
                }
            ]
    	}
    ]
}
\end{lstlisting}

\subsection{Configurações}

\subsubsection{Configuração do Mosquitto}

Em situações reais, cada casa terá uma instância do \emph{broker} Mosquitto rodando independente de todas as outras e aceitando somente conexões locais. Entretanto, para que fossem realizados testes e simulações, a instalação e execução de uma instância em cada máquina diferente, localmente, seria inviável. Para tanto, foram realizadas instalações em uma máquina remota --- em servidor da Digital Ocean\footnote{https://www.digitalocean.com/} --- com configurações diferentes, uma para cada residência simulada. São executadas instâncias como processos \emph{daemon} vinculados a portas diferentes --- a partir da porta 8883 (conexão com criptografia) para Morpheus e 1883 para módulos. Também foi criado um script em \emph{bash} para iniciar o processo e ativar as portas no \emph{firewall}. São necessárias as seguintes configurações para a máquina remota \cite{MQTTSecurity} \cite{PubSub}:

\begin{enumerate}
\item Habilitar a restrição de tópicos na instância \cite{topicRestriction}. A restrição deve levar em conta as credenciais do dispositivo logado no momento para definição do formato dos tópicos.

\item Os tópicos que finalizam em \emph{s2m} devem ser exclusivamente restritos ao Morpheus. Nenhum outro dispositivo deve conseguir publicar nestes tópicos. O Morpheus pode publicar e ouvir todos os tópicos.

\item Os tópicos que finalizam em \emph{m2s} são exclusivos de cada módulo. O \emph{broker} saberá se um módulo pode se inscrever ou publicar no tópico de acordo com o seu número serial.

\item Para cada casa, os módulos devem se conectar a partir da porta 1883 (e.g. primeira casa \textrightarrow{} 1883; segunda casa \textrightarrow{} 1884). Essas portas não exigem criptografia, mas devem exigir usuário e senha (que estarão vulneráveis).

\item O Morpheus é obrigado a se conectar a partir da porta 8883 (e.g. primeira casa \textrightarrow{} 8883; segunda casa \textrightarrow{} 8884), passando suas credenciais encriptadas.
\end{enumerate}

\subsubsection{Guia de instalação}\label{sec:arquivosCriados}
O passo-a-passo descrito aqui foi testado em um ambiente Ubuntu 16.10 x64.

\begin{enumerate}
\item Utilizar o terminal para fornecer os seguintes comandos.

\lstinline{sudo apt-get update}

\lstinline{sudo apt-get install mosquitto mosquitto-clients}

\lstinline{sudo systemctl enable mosquitto}

\item Criar pastas para cada casa em \lstinline{/etc/mosquitto/conf.d} com os nomes \texttt{home\textless Número\textgreater}. Devem-se criar os arquivos \texttt{acl\_list}, \texttt{m\_home\_\textless Número\textgreater .conf}, \texttt{passwd}. O conteúdo de cada um desses arquivos é mostrado abaixo (relativos à casa de número 1).


\textbf{acl\_list}

\begin{lstlisting}[language=bash]
    # General section

    # User specific section
    ## Morpheus
    user adf654wae84fea5d8ea6
    topic readwrite hw/#

    # Client section

    ## Modules can write only to the topic with their username in the m2s version
    pattern write hw/%u/m2s

    ## Modules can only read to the topic with their username in the s2m version
    pattern read hw/%u/s2m
\end{lstlisting}

\textbf{m\_home\_1.conf}

\begin{lstlisting}[language=bash]
    password_file /etc/mosquitto/conf.d/home1/passwd
    allow_anonymous false
    acl_file /etc/mosquitto/conf.d/home1/acl_list

    # General Listener
    # When running in production, this should bind to localhost
    port 1883
    require_certificate false
    use_username_as_clientid true

    # Morpheus Listener
    # When running in production, this should bind to localhost
    listener 8883
    cafile /etc/mosquitto/ca_certificates/ca.crt
    keyfile /etc/mosquitto/certs/mosquitto.key
    certfile /etc/mosquitto/certs/mosquitto.crt
    require_certificate true
\end{lstlisting}

\textbf{passwd}

\begin{lstlisting}[language=bash]
    0002D3D7:135876
    01344682:374028
    000750A1:524708
    001A1B07:321115
    0014BB3E:147203
    asd561asd5asd984faee:852456987
\end{lstlisting}

\item Para execução do script, basta utilizar o comando seguinte na pasta onde o arquivo se localiza.

\lstinline{. start.sh}

O conteúdo do script é mostrado no Anexo \ref{att:script}.
\end{enumerate}

\subsubsection{Criação dos Certificados}

Por fim, devem-se criar certificados válidos tanto para o \emph{broker} Mosquitto, quanto para as instâncias do Morpheus. Neste projeto, os certificados são gerados e auto-assinados. Entretanto, em um ambiente de produção, deve haver uma autoridade certificadora independente para garantia da validade e segurança.

\begin{enumerate}
\item
Criação da autoridade certificadora (chave e certificado). Para a versão atual, a senha é \texttt{hedwig123}.

\lstinline{openssl req -new -x509 -extensions v3\_ca -keyout ca.key -out ca.crt}
\item
Criação de chave e certificado para o Mosquitto. O \emph{common name} deve ser o IP do servidor.

\begin{lstlisting}[language=bash]
openssl genrsa -out mosquitto.key 2048
openssl req -new -key mosquitto.key -out mosquitto.csr
openssl x509 -req -in mosquitto.csr -CA ../ca.crt -CAkey ../ca.key -CAcreateserial -out mosquitto.crt -days 3650 -sha256
\end{lstlisting}

\item
Criação de chave e certificado para o Morpheus. O \emph{common name} deve ser \texttt{localhost}.

\begin{lstlisting}[language=bash]
openssl genrsa -out morpheus.key 2048
openssl req -new -key morpheus.key -out morpheus.csr
openssl x509 -req -in morpheus.csr -CA ../ca.crt -CAkey ../ca.key -CAcreateserial -out morpheus.crt -days 3650 -sha256 -addtrust clientAuth
openssl x509 -in morpheus.crt -outform der -out morpheus.der
\end{lstlisting}
\end{enumerate}

\subsubsection{Senhas}

Conforme mostrado anteriormente no guia de instalação (item \ref{sec:arquivosCriados}), o arquivo de senha deve ser criado no formato \lstinline{usuario:senha}. Deve-se, então, rodar o seguinte comando para que a senha não fique exposta em formato de texto:

\begin{lstlisting}
  sudo mosquitto_passwd -U passwd
\end{lstlisting}

\subsubsection{Casos de Teste para Controle de Acesso nos Tópicos \wmqtt{} entre Módulos e Nuvem \label{testesTopicos}}
\begin{enumerate}
\item
Conectar na porta 1883 sem usuário e senha.

Esperado: falha de conexão.

Resultado: bem-sucedido.
\item Conectar na porta 1883 com usuário e senha corretos.

Esperado: permissão de conexão.

Resultado: bem-sucedido.
\item
Conectar com credenciais corretas e tentar publicar em tópico que não pertence ao seu usuário.

Esperado: não-publicação.

Resultado: bem-sucedido.
\item
Conectar com credenciais corretas e tentar publicar em tópico que pertence ao seu usuário.

Esperado: publicação.

Resultado: bem-sucedido.
\item
Conectar com credenciais corretas e tentar ouvir um tópico que não pertence ao seu usuário.

Esperado: não receber dados.

Resultado: bem-sucedido.
\item
Conectar com credenciais corretas e tentar ouvir um tópico que pertence ao seu usuário.

Esperado: receber dados.

Resultado: bem-sucedido.

\item
Conectar com credenciais referentes ao Morpheus e tentar publicar ou ouvir qualquer tópico começando com \texttt{hw}.

Esperado: publicação ou subscrição com sucesso.

Resultado: bem-sucedido.
\end{enumerate}

\section{Servidor na nuvem \label{servidorNaNuvem}}

\subsection{Descrição}

O servidor na nuvem é composto por um servidor WebSocket para comunicação entre clientes e casas, um banco de dados em memória para armazenar informações sobre conexões WebSocket ativas, uma API REST para acesso aos dados persistidos, um banco de dados não-relacional para armazenar dados dos sensores, módulos, Morpheus e usuários, um proxy reverso e um \emph{firewall}.

Para fins de prova de conceito, optou-se por prosseguir com uma arquitetura monolítica. A implementação da arquitetura de microsserviços aumentaria consideravelmente a complexidade do projeto, e seus principais benefícios não seriam tão bem aproveitados, visto que o sistema não seria colocado a provas de carga real no momento. Contudo, ressalta-se que o monolito que compõe o servidor na nuvem poderia sim ser implementado como um conjunto de microsserviços, o que seria uma evolução natural à medida que o sistema escala.

\subsection{Características da implementação}

Com base nos requisitos funcionais e não-funcionais, discutidos na Subseção \ref{sec:requisitos}, foi realizada a implementação na nuvem. Suas características são discutidas em seguida.

\begin{description}

\item \textbf{Comunicação}

\begin{itemize}
\item O servidor permite que Morpheus e aplicativos clientes se comuniquem via WebSocket.
\item Morpheus pode enviar as mensagens provenientes dos módulos físicos, que são: \texttt{configuration}, \texttt{confirmation} e \texttt{data}. Também podem receber mensagens destinadas aos módulos físicos, que são: \texttt{action}, \texttt{configuration} e \texttt{data}.
\item Morpheus pode receber mensagens de registro e remoção de módulo para definir quais dispositivos ele gerencia.
\item Morpheus pode enviar mensagens de \texttt{report}.
\item Os aplicativos cliente podem enviar as mensagens correspondentes a interações do usuário, que são: \texttt{action} e \texttt{configuration}. Também podem receber mensagens provenientes dos módulos físicos, que são: \texttt{confirmation} e \texttt{data}.
\item Aplicativos cliente podem receber mensagens de \texttt{report}.
\item Aplicativos cliente podem receber o status de conectividade dos Morpheus e receber notificações quando um Morpheus for desconectado.
\end{itemize}

\item \textbf{Persistência de dados}

\begin{itemize}
\item O servidor persiste dados de usuário, de configurações de Morpheus e módulo e de mensagens de dados (\texttt{data} e \texttt{report}).
\end{itemize}

\item \textbf{Gerenciamento de dados}

\begin{itemize}
\item O servidor deve oferece uma API REST para leitura e escrita de dados de usuário, configurações de Morpheus e de módulos.
\end{itemize}

\item \textbf{Conexões}

\begin{itemize}
\item O servidor permite o estabelecimento de conexões HTTPS seguras e criptografadas.
\item O \emph{firewall} bloqueia conexões em portas que não estão sendo utilizadas.
\end{itemize}

\end{description}

\subsection{Tecnologias usadas}

O servidor foi desenvolvido usando Node.js \footnote{https://nodejs.org}, um ambiente em tempo de execução para código em JavaScript. Sua arquitetura usa um modelo orientado a eventos e realiza a execução de comandos concorrentemente sem bloquear o servidor. Assim, servidores em Node.js conseguem alcançar uma melhor escalabilidade, suportando múltiplas conexões simultâneas sem impactos de performance.

Para a persistência de dados, foi escolhido o MongoDB \footnote{https://www.mongodb.com/}, banco de dados não-relacional baseado em documentos. A facilidade de integração com JavaScript e Node.js, a similaridade dos documentos com objetos JSON e a natureza dos dados de sensores foram as motivações para sua escolha como banco de dados principal.

Contudo, não são apenas informações sobre usuários e dispositivos e dados coletados pelos sensores que precisam ser armazenados. Para gerenciar quais Morpheus estão conectados à nuvem e e a quais aplicativos suas informações em tempo real devem ser enviadas, é usado o Redis \footnote{https://redis.io/}, banco de dados em memória. Redis é popularmente usado para fins como cache, mensageria e implementação de filas. No caso do Hedwig, ele é utilizado para armazenar informações de sessão, que são temporárias e requerem baixas latências para leitura e escrita.

Para implementar a comunicação entre aplicativos e casas, foi escolhida a biblioteca Socket.io \footnote{https://socket.io/}, que fornece uma API de alto nível para troca de informações bidirecional por meio de eventos. Além de abstrair a API de baixo nível do protocolo de WebSockets, o Socket.io já fornece eventos referentes ao status da conexão, facilitando o disparo de notificações caso o controlador de uma casa seja desconectado, e implementa um fallback para clientes que não suportam o protocolo de WebSocket. Por exemplo, se um usuário acessa um aplicativo por meio de um navegador antigo, a troca de dados continua sendo feita por meio de long polling.

A arquitetura possui um proxy reverso que é responsável por enviar as requisições ao servidor em Node.js. Para isso, foi usado o nginx \footnote{https://nginx.org/}, popularmente utilizado como servidor HTTP e proxy genérico para TCP e UDP. Ele permite a configuração de conexões seguras via HTTPS e dispensa a necessidade de delegar privilégios para acessar as portas reservadas 80 e 443 ao processo que roda o servidor Node.js.

Por fim, foi usada a ferramenta padrão do Ubuntu para \emph{firewall}, ufw, que permite criar regras para bloquear tráfego IPv4 e IPv6.

\begin{figure}[H]
	\centering
	\caption{Componentes e implementação na nuvem}
  \includegraphics[width=0.5\textwidth]{componentesNuvem}
\label{fig:componentesNuvem}
\end{figure}

\subsection{Infraestrutura}

Para hospedar o servidor do Hedwig, foi utilizado o serviço de computação na nuvem Digital Ocean \footnote{https://www.digitalocean.com/}. Com ele, foi possível implantar o servidor em uma instância que roda Ubuntu 16.04.3 x64, com 1 CPU, 512MB de memória, 20GB de armazenamento de disco SSD e 1000GB de cota disponível para transferência de dados. O data center que hospeda essa instância fica em Nova Iorque.

\subsection{Segurança}

O servidor na nuvem suporta conexões HTTPS, permitindo que navegadores verifiquem a autenticidade do servidor e garantindo a privacidade e integridade dos dados transmitidos. Para isso, foi usado o Let's Encrypt\footnote{https://letsencrypt.org}, uma autoridade de certificação aberta, gratuita e automatizada. O Let's Encrypt usa o protocolo ACME (\emph{Automatic Certificate Management Environment}) para automatizar a comunicação entre autoridade e candidato para assegurar a autenticidade deste e conceder-lhe certificados de forma rápida e prática. É realizado um teste para verificar que o candidato possui controle sobre o domínio e, então, é gerado um certificado válido por 90 dias que pode ser renovado a qualquer momento. Para usá-lo no servidor, basta acrescentar novas configurações ao nginx.

Além disso, outra medida de segurança foi usar um \emph{firewall} para bloquear conexões nas portas TCP que não estão sendo usadas.

\subsection{Operação}

O serviço Keymetrics\footnote{https://keymetrics.io/} permite verificar se o servidor na nuvem está online, monitorar o uso de CPU e de memória, investigar a ocorrência de erros e realizar ações comuns, como reiniciar o processo do servidor, por meio de uma interface amigável. Investir em um sistema de monitoramento como esse auxilia tanto a manutenção preventiva como a corretiva, o que é essencial para um sistema de casa inteligente que tem a disponibilidade como requisito prioritário.

\begin{figure}[H]
	\centering
	\caption{Monitoramento do servidor na nuvem}
  \includegraphics[width=1.0\textwidth]{keymetrics}
\label{fig:keymetrics}
\end{figure}

Outro ponto abordado é o uso de logs, arquivos que gravam eventos relevantes que acontecem no sistema. Eles podem ser usados para realizar a auditoria de falhas ocorridas e compreender melhor o funcionamento de um programa. Por questões de simplicidade, para classificar os eventos, o servidor na nuvem do Hedwig usa três do sete níveis de severidade definidos pelo padrão syslog \cite{rfc5424}: \emph{error}, \emph{warning} e \emph{informational}. O servidor implementa um esquema de rotação de logs, criando arquivos individuais para cada dia, o que facilita o arquivamento de logs muito antigos e a pesquisa de eventos específicos. Com esse sistema, também é possível filtrar eventos de severidades diferentes em arquivos separados.

\section{Aplicativo de dashboard}

\subsection{Descrição}
O aplicativo desenvolvido é uma dashboard que permite ao morador da casa ver os dados dos sensores em tempo real, enviar requisições para os módulos realizarem alguma ação, receber confirmações de que essas ações foram realizadas e gerenciar seus dispositivos - Morpheus e módulos. Essa dashboard foi implementada com base nas características dos PWAs apresentadas no capítulo de Arquitetura a fim de permitir uma boa experiência de usuário tanto em smartphones como em computadores.

\subsection{Requisitos}

\subsubsection{Requisitos funcionais}
\begin{description}

\item \textbf{Realizar cadastro, login e gerenciamento de informações pessoais}

\begin{itemize}
\item O usuário pode se cadastrar com seu email, nome e data de nascimento, criando também um nome de usuário e senha para acessar a dashboard.
\item O usuário pode realizar login usando seu nome de usuário e senha.
\item O usuário pode alterar as informações pessoais de seu cadastro.
\end{itemize}

\item \textbf{Gerenciar Morpheus}

\begin{itemize}
\item O usuário pode adicionar e remover controladores locais - Morpheus, sendo que para o cadastro basta adicionar o número de série do dispositivo.
\item O usuário pode configurar o Morpheus, especificando se deseja que mensagens que não puderam ser enviadas por problemas de conectividade sejam mandadas assim que a conexão for reestabelecida.
\end{itemize}

\item \textbf{Gerenciar módulos}

\begin{itemize}
\item O usuário pode adicionar e remover módulos, sendo que para o cadastro deve-se adicionar o número de série do dispositivo, relacioná-lo ao Morpheus que ouvirá suas mensagens, dar um nome ao módulo e seus relês e especificar o seu tipo.
\item O usuário pode configurar o módulo, especificando as configurações de conectividade, display e teste de \emph{auto reset}.
\end{itemize}

\item \textbf{Receber dados e enviar ações em tempo real}

\begin{itemize}
\item O usuário pode visualizar em tempo real os dados dos sensores de abertura, presença, temperatura, umidade e luminosidade do módulo básico.
\item O usuário pode visualizar o estado dos relês e enviar ações para ligá-los e desligá-los.
\item O usuário pode visualizar em tempo real os dados do estado do portão e do alarme do módulo de acesso.
\item O usuário pode, no módulo de acesso, enviar uma ação para abrir o portão usando uma senha.
\end{itemize}

\item \textbf{Alterar configurações avançadas dos módulos}

\begin{itemize}
\item O usuário pode receber confirmações de que ações sensíveis foram recebidas pelos módulos corretamente.
\item O usuário pode gerenciar o sensor de radiofrequência.
\item O usuário pode enviar uma ação para sincronizar a hora do módulo.
\item O usuário pode enviar uma ação para reiniciar o módulo (\textit{soft reset}).
\end{itemize}

\item \textbf{Visualizar estado das conexões}

\begin{itemize}
\item O usuário pode ver o status de sua conexão com o servidor da nuvem e da conexão de seus Morpheus com a nuvem.
\end{itemize}

\end{description}

\subsubsection{Requisitos não-funcionais}

\begin{itemize}
\item O aplicativo deve ser responsivo e totalmente funcional nos navegadores mais recentes em suas versões desktop e mobile.
\end{itemize}

\subsection{Tecnologias utilizadas}

Para criar a aplicação web que demonstra a funcionamento do Hedwig, foi escolhido o React\footnote{https://reactjs.org/}, biblioteca \emph{open-source} de JavaScript mantida pelo Facebook. O React é conhecido por facilitar o desenvolvimento de aplicações \textit{single-page}, renderizadas do lado do cliente, que permitem a atualização dinâmica da página de forma fluida e rápida, o que acaba enriquecendo a experiência do usuário. Possui uma linguagem declarativa e baseada em componentes, tornando-a altamente modularizável e reutilizável. Uma de suas características mais reconhecidas é o uso de um DOM virtual e um algoritmo de reconciliação que consegue identificar quais as partes da página que precisam ser renderizadas a cada interação com o usuário \cite{reactdiff}, melhorando a performance e permitindo maior fluidez em animações e mudanças visuais.

Outro ponto interessante para a utilização do React é que, com a biblioteca React Native\footnote{https://facebook.github.io/react-native/} - uma extensão do React - é possível a geração de aplicativos nativos para iOS e Android. Assim, caso surja a necessidade de implementar uma nova funcionalidade que possua algum requisito que não pode ser contemplado por um \textit{Progressive Web App}, mas pode ser atendido por um aplicativo nativo, pode-se usar o React Native. Isso diminui a necessidade de retrabalho e dispensa a necessidade de estudo aprofundado das linguagens e ambientes de desenvolvimento tradicionais de projeto de aplicativos nativos.

A fim de facilitar a implementação das interações com o usuário, usamos juntamente ao React a biblioteca Redux\footnote{https://redux.js.org/}, um gerenciador de estado global. Dessa forma, pretendemos facilitar o compartilhamento de informações entres diferentes componentes. A motivação por trás do Redux é facilitar a leitura e atualização do estado da aplicação, que, em sites \textit{single-page} modernos, pode armazenar respostas do servidor, cache de dados e dados criados localmente que ainda não foram persistidos no servidor. O Redux é baseado em três princípios \cite{redux}:

\begin{itemize}
\item O estado é o ponto único de verdade.
\item O estado permite apenas a leitura - a única forma de alterá-lo é emitindo uma \textit{action}.
\item Alterações no estado devem ser feitas por funções puras - são os chamados \textit{reducers}, que recebem o estado anterior e uma \textit{action} e retornam o novo estado.
\end{itemize}

\begin{figure}
	\centering
	\caption{Diagrama de funcionamento do Redux}
  \includegraphics[width=\textwidth]{reduxFlowchart}
  \caption*{Fonte: \cite{reduxdataflow}}
\label{fig:reduxFlowchart}
\end{figure}

Para receber dados dos módulos em tempo real, foi usado um cliente do socket.io, framework que já foi discutido na seção do servidor na nuvem. O cliente de JavaScript já permite monitorar o status da conexão do aplicativo com o servidor, emitindo eventos em caso de desconexão ou reconexão. Assim, foi necessário criar \textit{listeners} para os eventos customizados que criamos para nossos protocolos.

Uma das vantagens em usar JavaScript para criar a view da aplicação é sua versatilidade, visto que é uma linguagem multi-paradigmas que suporta programação imperativa, declarativa e orientada a objetos \cite{mdnjs}. Outro ponto é que ela é usada tanto para desenvolvimento \emph{front-end} como \emph{back-end}, permitindo que o conhecimento acumulado na execução de um projeto ajude no outro.

Para aproveitar as funcionalidades e facilidades sintáticas das especificações mais recentes de JavaScript, usamos Babel\footnote{https://babeljs.io/}, um compilador capaz de converter as sintaxes novas e substituir as funções ainda não suportadas pelos navegadores com auxílio de \textit{polyfills}. Muitas vezes, é usado o termo transpilador para referir-se ao Babel, visto que é um compilador de JavaScript para JavaScript, não deixando de emitir como saída uma linguagem de alto-nível. Assim, é possível usar ES6 - a versão mais recente de JavaScript - sem se preocupar com a compatibilidade do aplicativo com os navegadores que ainda não implementaram essa especificação completamente.

Para agilizar o desenvolvimento e prevenir falhas, usamos o \textit{linter} dedicado a JavaScript ESLint\footnote{https://eslint.org/}. \textit{Linter} é uma ferramenta para verificar o código e identificar erros de programação ou inconsistências de estilo \cite{linter}, o que possibilita produzir programas mais consistentes e menos suscetíveis a bugs.

Por fim, a pipeline de desenvolvimento do cliente usa o Webpack\footnote{https://webpack.js.org} como \textit{module bundler}. O Webpack cria gráficos de dependência de todos os componentes da aplicação web - imagens, folhas de estilo, scripts - e então os processa transformando-nos em \textit{bundles} ou pacotes, que nada mais são que arquivos estáticos.

\subsection{Interface}

\subsubsection{Identidade visual}

Para facilitar o desenvolvimento, foram usados os componentes do Material Design\footnote{https://material.io/} da \emph{Google}, que satisfazem várias necessidades básicas e casos de uso da criação de interfaces modernas. Para distinguir cada tipo de módulo, foram usados ícones e paletas de cores distintas.

\subsection{Interações}

\subsubsection{Dados}

Dados provenientes dos sensores dos módulos são atualizados em tempo real por meio de eventos do socket.io. Para situar o usuário, o instante de tempo em que a mensagem do módulo foi enviada fica visível no fim na página. Caso seja detectado que a conexão com o Morpheus se perdeu, isso é mostrado explicitamente pela dashboard a fim de evitar que o usuário olhe dados antigos demais.

\subsubsection{Ações}

Ao enviar uma ação, o estado local do aplicativo não muda até que seja recebida uma nova mensagem de dados ou de confirmação. Isto é, se um relê está ligado e o usuário pede para desligá-lo, o aplicativo só vai mostrar que o relê desligou quando o módulo o avisa disso. Isso evita que o aplicativo entre em estados inconsistentes com os módulos.

\subsubsection{Conectividade}

O aplicativo possui um indicador no canto superior direito indicando o status da conexão. Ele pode assumir três estados:

\begin{itemize}
\item Aplicativos e todos os Morpheus do usuário estão devidamente conectados ao servidor na nuvem
\item Aplicativo está devidamente conectado ao servidor na nuvem, mas pelo menos um dos Morpheus não
\item Aplicativo não está conectado ao servidor na nuvem
\end{itemize}

Além disso, caso o aplicativo perca conexão com a nuvem, são emitidas notificações na parte inferior da tela. São realizadas 10 tentativas de reconexão e o resultado - positivo ou não - também aparece como um alerta na tela. Caso não haja sucesso em nenhuma das 10 tentativas, o usuário é aconselhado a atualizar a página.

\subsubsection{Aplicativo na tela inicial do dispositivo móvel}

Usando um arquivo de manifesto, foi possível configurar como o aplicativo aparece na tela inicial de um dispositivo móvel como um smartphone. Diminuindo o número de passos para o usuário acessar o aplicativo o encoraja a usá-lo mais vezes, além de criar uma sensação parecida com a de um aplicativo nativo.

\subsection{Implantação}

O aplicativo foi publicado com o serviço Surge\footnote{https://surge.sh/}, que permite a hospedagem de websites estáticos e oferece um domínio customizado. O Surge possui uma aplicação de linha de comando que permite a publicação de um diretório de arquivos HTML, CSS e JavaScript de maneira rápida e sem extensiva configuração. A dashboard está disponível em \url{https://hedwig.surge.sh}.

\subsection{Segurança}

O Surge usa a comunicação via HTTPS por padrão, oferecendo o suporte básico a SSL. Dessa forma, os dados transmitidos entre navegador e servidor podem ser criptografados, permitindo uma maior segurança em operações como cadastro, login e recebimento dos dados dos sensores.

\subsection{Performance}

Levando em consideração que o aplicativo pode ser acessado pelo celular em situações em que a qualidade da conexão não é a ideal, a otimização da dashboard para obter uma boa performance e baixos tempos de carregamento torna-se um ponto importante. Para isso, algumas medidas foram tomadas:

\begin{itemize}
\item Arquivos estáticos são servidos por uma CDN e podem ser armazenados no cache do navegador, medida que diminui o tempo de carregamento nas visitas subsequentes à dashboard.
\item Arquivos são compactados em formato \texttt{gzip} antes de serem enviados ao cliente, o que diminui seu tempo de download.
\item Arquivos HTML, CSS e JavaScript são minificados, diminuindo consideravelmente seu tamanho.
\item As imagens são redimensionadas e otimizadas, diminuindo consideravelmente seu tamanho.
\item Não há redirecionamentos desnecessários.
\end{itemize}

\section{Aplicativo Backup}

Para lidar com o caso de indisponibilidade do controlador local Morpheus, da rede local (roteador \emph{wireless}) ou da conexão com a Internet, foi desenvolvido um aplicativo de \textit{backup}. Esse aplicativo permite acesso direto ao módulo através do endereço local (supondo que o dispositivo celular esteja na mesma rede) ou por conexão direta com o ponto de acesso do módulo (disponível todo o tempo ou sempre que o módulo não consiga conexão com a Internet, a depender da preferência do usuário).

\subsection{Requisitos}

Destacam-se os requisitos mais críticos do aplicativo backup:

\begin{enumerate}
	\item Permitir acesso direto ao módulo em caso de indisponibilidade da rede Wi-Fi e Internet;
	\item Permitir a visualização de estado de sensores e a atuação em tempo próximo ao apresentado por botão físico;
	\item Acesso ao módulo por meio de endereço local no caso de haver rede local disponível, mas sem acesso à Internet;
	\item Possibilitar ao usuário configurar rede Wi-Fi, nome do módulo, cor do painel, offset de sensores, nome dos relés e regras de atuação (por radiofrequência, com senha ou não, por horário, por eventos dos sensores de presença e abertura, além do tempo em que deve ficar ligado no caso de configurações automáticas);
	\item Permitir configurar múltiplos códigos RF (radiofrequência) para sensor de abertura, atuação do relé 1 ou do relé 2;
	\item Permitir ao usuário configurar controle de acesso e senha para atuação de um relé em específico;
	\item Segregar permissões entre administrador e usuário --- usuários não podem executar configurações, somente visualizar estados e atuar em relés sem senha;
	\item Permitir visualização de vários módulos da residência;
	\item Ter interface de fácil navegação, intuitiva.
\end{enumerate}

\subsection{Tecnologias utilizadas}

Para o desenvolvimento do aplicativo backup, foram utilizados:

\begin{enumerate}
	\item Módulo como servidor, utilizando bibliotecas de comunicação próprias do ESP8266 no caso de comunicação direta com o módulo, e uso do módulo como cliente, usando a mesma biblioteca;
	\item Uso de \emph{ping} para verificação de conexões e execução de rotinas para a correta configuração de estado do ponto de acesso (ligado se não conectado à rede), rotinas de desconexão e reconexão;
	\item CSS para a implementação de interface amigável para o usuário, e segregação de configurações em níveis de navegação maiores para que configurações mais usadas sejam mais facilmente acessíveis a partir do menu principal do módulo;
	\item JavaScript para as rotinas de configuração e atualização do estado no menu principal;
	\item HTML5 para o desenvolvimento da maioria das páginas;
	\item EEPROM do módulo ESP8266, onde todas configurações de conexão, gerais e relés ficam armazenadas. Em caso de mudança desses parâmetros, ocorre sua persistência na EEPROM;
	\item Bibliotecas próprias para interface com sensores e outros periféricos (DHT e I2C, por exemplo), comunicação em geral, além do Wi-Fi e \emph{Access Point} (PubSub para comunicação por MQTT).
\end{enumerate}

\subsection{Navegação}

A partir da tela inicial, um cliente pode verificar todos os módulos presentes em sua casa, visualizar temperatura, umidade, luminosidade, sensores de presença ou abertura e apagar ou acender luzes (para atuadores protegidos com senha, o usuário deve acessar a página do respectivo módulo), numa interface configurável (o usuário pode configurar quais parâmetros observar nesse menu principal). A tela da Figura \ref{fig:telasPrincipaisBackup} é própria para celulares, enquanto a tela da Figura \ref{fig:plantaBackup}, com posicionamento configurável e que simula a planta de uma casa (em um cenário real, poderia ser a própria planta da casa do usuário) é própria para desktops.

\begin{figure}[H]
	\centering
	\caption{Telas principais do Aplicativo Backup}
  \includegraphics[width=0.5\textwidth]{telasPrincipaisBackup}
\label{fig:telasPrincipaisBackup}
\end{figure}

\begin{figure}[H]
  \centering
  \caption{Visão geral da planta de uma casa com o Aplicativo Backup}
  \includegraphics[width=0.8\textwidth]{plantaBackup}
  \label{fig:plantaBackup}
\end{figure}

A partir do menu principal, pode-se acessar o menu do módulo desejado (vide Figura \ref{fig:menuPrincipalBackup}). Nele, encontram-se informações de luminosidade, horário, data, temperatura, umidade, sensor de presença (sensor 1) e sensor de abertura (sensor 2), além de controle de portão, lâmpada ou eletrodoméstico. Também exibe o nível de recepção de Wi-Fi do módulo e a versão do \emph{firmware}.

\begin{figure}[hbp]
    \centering
    \begin{minipage}{.4\linewidth}
        \centering
        \captionof{figure}{Menu principal do Aplicativo Backup}
        \includegraphics[height=6cm]{menuPrincipalBackup}
        \label{fig:menuPrincipalBackup}
    \end{minipage}
    \hfill
    \begin{minipage}{.4\linewidth}
        \centering
        \captionof{figure}{Teclado para digitação da senha}
        \includegraphics[height=6cm]{senhaBackup}
        \label{fig:senhaBackup}
    \end{minipage}
\end{figure}

No caso de proteção de controle por senha, é exibido o painel numérico como na Figura \ref{fig:senhaBackup}. A cada requisição da página, o módulo manda um mapeamento das teclas diferente. Por exemplo, de teclas A, B, C, etc. para (3,1), (9,2), etc. Após o usuário entrar com a senha (sequência de teclas do tipo A, B, C, etc.), o módulo valida a sequência e autoriza o acionamento. Dessa forma, pessoas não conseguem copiar a senha ao visualizar a sequência de teclas do usuário, tampouco um invasor poderia copiar a sequência e utilizá-la para abertura logo em seguida, pois o mapeamento seria outro.

\subsection{Configurações}

A partir do menu principal do módulo, pode-se ter acesso ao seu log para depuração, uma opção disponível apenas para administradores, e um menu de configurações. Do menu de configurações, pode-se alterar o modo do display (dentre 3 opções) e as cores do módulo, conforme opções anteriormente citadas.

Ainda do menu de configurações, podem-se configurar alertas e alarmes (sonoros a partir dos módulos e pela Internet através do provedor gratuito Blynk e email), relés (possibilidade de auto ligar a partir de um sensor específico ou de ser acionado a partir de um sinal de rádio - usualmente um controle remoto ou até um sensor de abertura adicional ou mais de uma opação) e o log (quais parâmetros são persistidos e mandados para a nuvem). Também pode-se acessar o menu de Ferramentas, no qual é possível realizar testes de \emph{auto reset} para verificação do funcionamento do circuito antitravamento, que age em cerca de 30 segundos, reiniciar o módulo, desconectá-lo, voltar à versão de fábrica (versão implementada em software, enquanto a versão em hardware é realizada por meio de botão oculto) e atualizar o \emph{firmware} (apenas disponível para administradores).
Ao acessar a atualização de \emph{firmware}, escolhe-se a opção, que mostra a versão atual e, após escolha do arquivo, a versão a ser inserida. % TODO melhorar essa frase

É possível também acessar as configurações avançadas \ref{fig:firmwareconfigavancadasDHT}. Nela, pode-se configurar um offset para temperatura e umidade, de preferência realizados a partir de um medidor confiável para calibração.

Do menu de configurações avançadas, podem ser configurados os sensores e controladores de radiofrequência (sem fio), aspectos de conectividade do servidor gratuito Blynk e possivelmente trocar a rede Wi-Fi em que o módulo está conectado. Ainda, para administradores, há a opção de configurar os usuários que têm acesso ao módulo.

Ainda pode-se trocar o nome do módulo e ativar/desativar o ponto de acesso (para acesso direto ao módulo), além de configurar o nome (SSID) e senha de sua rede Wi-Fi.

Demais ilustrações referentes às configurações do aplicativo backup estão disponíveis no Anexo \ref{attachmentsImagensBackup}{}.

\subsection{Abertura de porta do roteador}

Para acessar o módulo e o menu remotamente, um solução é usar a abertura de porta (mecanismo NAT ou \emph{virtual servers}), configurável nas páginas de configuração dos roteadores. Maiores informações para essa configuração podem ser encontradas nos manuais dos roteadores. Observe que há uma segurança menor envolvida com essa configuração. Assim, essa alternativa provê a abertura da porta para acesso remoto sem a necessidade de serviços em nuvem, o que é indicado apenas para usuários que podem lidar com um nível de segurança mais baixo.

\begin{figure}[hbp]
    \centering
    \caption{Página de configuração TP Link}
    \includegraphics[width=0.8\textwidth]{tpLinkAppBackup}
    \label{fig:tpLinkAppBackup}
\end{figure}

\emph{Obs.:} Caso sua operadora só forneça o CGNAT, a abertura de porta por parte do usuário não será possível.

\subsection{Controle remoto}

Para que o controle remoto seja corretamente configurado, são necessários os seguintes passos:
\begin{enumerate}
    \item
    A partir da página inicial do módulo, entre em \# \textrightarrow{} Configurações Avançadas \textrightarrow{} RF433.

    \item
    Configure o Sensor de Abertura “Sulton” no modo 2 para mandar sinais diferentes de abertura e fechamento. Consulte o manual do fabricante.

    \item
    Aperte +, espere até a página indicar “Aguardando” e abra o sensor de abertura. É permitido incluir diversos sinais para o controle do mesmo relé, abertura ou fechamento de sensor. Aperte “OK” no canto superior direito da página, para salvar suas configurações.

    \item
    Repita o passo 3 para o fechamento do sensor de abertura. \textbf{Cuidado:} o sensor de abertura repete algumas vezes o mesmo sinal. Aguarde alguns instantes entre gravar a abertura e o fechamento para não gravar o mesmo sinal. Isso pode ser verificado na série numérica que aparece gravada: se estiver igual, há um problema.

    \item
        \begin{enumerate}
            \item
            A partir da página inicial do módulo, entre em \# \textrightarrow{} Configurações Avançadas \textrightarrow{} RF433.

            \item
            Aperte + (ao lado do relé 1 ou relé 2), espere até a página indicar “Aguardando” e abra o sensor de abertura. É permitido incluir diversos sinais para o controle do mesmo relé, abertura ou fechamento de sensor).

            \item
            Aperte "OK" no canto superior direito da página para salvar suas configurações.

        \end{enumerate}
\end{enumerate}

\subsection{Notificações}
Para que as notificações sejam ativadas, siga os próximos passos.

\begin{enumerate}
    \item
    Para que o suporte consiga acessar remotamente o módulo e realize a coleta de dados, acesse \# \textrightarrow{} Configurações Avançadas.\textrightarrow{} Blynk.

    \item
    Insira o \emph{Auth Token} (e.g. aa7a6dc1170640f08e951ed8cd2198a1).

    \item
    Selecione Notificações ao iniciar: Sim, e Wi-Fi: Sim.

    \item
    Aperte em OK, no canto superior direito, para salvar suas alterações.
\end{enumerate}

\subsection{Offset de temperatura e umidade}

\begin{enumerate}
    \item
    Para que o suporte consiga acessar remotamente o módulo e realize a coleta de dados, acesse \# \textrightarrow{} Configurações Avançadas.\textrightarrow{} Temperatura e Umidade.

    \item
    Efetue a calibração do equipamento usando uma referência externa.

    \item
    Aperte em OK, no canto superior direito, para salvar suas alterações.
\end{enumerate}

\subsection{\emph{Hard reset}}
\begin{enumerate}
	\item Ligue na tomada
	\item Abra a tampa do módulo, e retire o isopor que cobre o sensor de temperatura azul;
	\item Localize o botão (“\emph{pushbutton}”), após retirar o isopor;
	\item Pressione o botão até ouvir 6 bipes;
	\item O módulo retornará para a configuração de fábrica. Veja as seções a seguir para sua configuração inicial.
\end{enumerate}

\subsection{Setup inicial}
\begin{enumerate}
	\item Primeiro, conecte à rede Wi-Fi do Módulo, com nome CONFIG (a senha da rede CONFIG é 12345678).
	\item Abra um navegador e vá ao endereço 192.168.4.1 para entrar na página de configuração. Entre com as credenciais (login: admin e senha: 1234).
	\item Espere até que as redes disponíveis apareçam, selecione a rede de sua casa e forneça sua senha para conectar o módulo à Internet.
	\item Observe o endereço local 192.168.0.X que aparecerá no LCD do módulo. Caso não consiga ver, realize o passo 7 e use uma ferramenta como o “Zentri” (aplicativo Android) para descobrir em que endereço o módulo entrou. Para ver na tela novamente, pode-se tirar o módulo da tomada e ligá-lo novamente.
	\item Espere o módulo reiniciar (continue na rede CONFIG) e aguarde até a página recarregar. Insira o nome do módulo e depois aperte “Salvar e Reiniciar”.
	\item Conecte-se novamente na rede Wi-Fi da sua casa.
	\item Entre no endereço 192.168.0.X que foi mostrado no módulo, clique em \# e então em “Reiniciar Busca”. Aguarde até que todos os módulos sejam descobertos.
	\item Retorne para a página anterior (usando o “\textless” no canto superior esquerdo). Você verá o menu principal da casa, e então poderá controlar relés, visualizar dados coletados pelos módulos e entrar em cada módulo. O menu é personalizado na página anterior (ao pressionar o “\#” no canto superior direito da tela).
\end{enumerate}

%  Para o desenvolvimento de funcionalidades de aprendizado de máquina, será utilizada a linguagem Python, que possui diversos pacotes que facilitam sua utilização para implementar algoritmos de aprendizado, e funcionalidades para tratamento de dados. Além disso, é usada em vários outros âmbitos como cursos acadêmicos voltados ao ensino de programação e aplicações web, o que facilita a familiarização com o desenvolvimento nela.


\chapter{Conclusões}

O principal foco deste trabalho foi a criação de uma arquitetura abrangente e robusta para automação residencial, com sua implementação completa --- desde os módulos de hardware à aplicação cliente.

A fim de demonstrar todas as funcionalidades dos módulos, por meio do aplicativo web, optou-se pelo modelo de dashboard, que promove interação com fluidez e concisão. Porém, a arquitetura permite que outros tipos de aplicação possam se comunicar com as casas. O processamento de linguagem natural com um chatbot, por exemplo, ajudaria o morador a interagir com a casa, guiando-o durante o processo de criação de regras para acionamento automático de dispositivos. A área de Business Intelligence também oferece um vasto campo a ser explorado, com a criação de um aplicativo capaz de exibir dados da casa em gráficos parametrizados, com relatórios sobre o comportamento dos usuários. Como demonstrado na Seção \ref{coletaAnaliseDados}, são obtidas informações relevantes sobre a vida dos moradores de uma residência mesmo com um número limitado de sensores e atuadores em uso.

Para a implementação atual, não foram realizados testes de carga para o servidor na nuvem, o qual foi  implementado como um monolito, em uma única instância. O próximo passo seria avaliar os melhores procedimentos de escalabilidade e estudo de sua performance com vários dispositivos e usuários conectados, de modo que seja possível a identificação de gargalos em seus componentes. Como visto na Seção \ref{servidorNaNuvem}, as tecnologias empregadas na implementação do servidor na nuvem --- nginx, Node.js, MongoDB e Redis --- possuem características favoráveis à sua utilização em aplicações altamente escaláveis.  O desafio, a partir daí, se concentraria na redundância para a comunicação com as casas, onde são utilizadas conexões WebSocket.

O monitoramento do servidor de nuvem também pode ser explorado, de maneira que sejam entendidas suas características de uso, peça essencial para o alcance de maior disponibilidade e robustez. A integração realizada durante o projeto foi simplificada e usou um serviço que apenas monitora uso de memória e CPU. Idealmente, deve ser possível também acompanhar eventos e transações importantes que ocorrem na nuvem, configurar alertas de picos de uso de recursos e poder ter uma visão consolidada dos logs, com busca e estatísticas.

Nos aspectos de hardware, e da infraestrutura de comunicação, podem ser explorados outros canais, transparentes ao usuário, e que seriam capazes de oferecer maior robustez à aplicação. Ao ampliar as formas de comunicação disponíveis, os módulos poderiam fazer uso de redes de dados --- 4G e futuramente 5G ---, e não seriam mais dependentes exclusivos da internet local. A integração, e também o desenvolvimento, de dispositivos vestíveis, integráveis com a casa, é também um passo futuro, que melhorará a experiência do usuário e expandirá as possibilidades de aplicação ---como, por exemplo, um relógio que monitora pessoas idosas, e envia notificações aos familiares.

A possibilidade de atualização de firmware remotamente agregou em segurança e flexibilidade ao projeto, já que torna possível o envio de correções diretamente ao usuário, por meio da internet, sem a necessidade do contato direto com o módulo. Assim, frente a uma potencial ameaça, pode-se desenvolver uma correção e enviá-la aos clientes, que manteriam suas casas atualizadas.

Como outras sugestões aos próximos passos e caminhos futuros, indica-se a criação de testes unitários e automatizados, que verificam pequenas porções de código por vez. Isso facilitaria a adição de novas funcionalidades, garantindo a compatibilidade reversa, além de aumentar as chances de identificar falhas antes de liberar novas versões de software ao público. Com a mesma mentalidade, pode-se implementar uma pipeline de integração contínua para identificar erros de integração rapidamente por meio de testes e verificação de código e permitir o lançamento de novas versões de maneira ágil. O fato do código-fonte do Hedwig já estar em um sistema de controle de versão, o GitHub, é um acelerador para o uso de serviços de integração contínua. Outro importante aspecto relacionado às tecnologias de Internet das Coisas, mas que não foi coberto neste trabalho são as questões éticas e relacionadas à privacidade do usuário, as quais são complexas e necessitarão de grandes esforços para a elaboração de uma legislação adequada.



% ========== Referências ==========
% --- IEEE ---
%	http://www.ctan.org/tex-archive/macros/latex/contrib/IEEEtran
%\bibliographystyle{IEEEbib}

% --- ABNT (requer ABNTeX 2) ---
%	http://www.ctan.org/tex-archive/macros/latex/contrib/abntex2
\bibliographystyle{abntex2-alf}

\bibliography{citations.bib}


% ========== Apêndices (opcional) ==========
%\apendice



% ========== Anexos (opcional) ==========
\anexo
\chapter{Códigos das aplicações desenvolvidas}
\label{github}
Todos os códigos das aplicações desenvolvidas neste projeto estão disponíveis em:
\par 
\url{https://github.com/hedwig-project/}.

\chapter{Lista de materiais para montagem dos módulos}
\label{listamateriais}

\begin{table}[hbp]
    \caption{Lista de materiais}
    \begin{tabular}{lc}
        \toprule
        \textbf{Item} & \textbf{Quantidade} \\
        \midrule
        Raspberry Pi 3 Modelo B                                & 1                   \\
        Wemos D1 Mini                                          & 7                   \\
        Cabo fêmea-fêmea 20cm                                  & 1                   \\
        RF 433                                                 & 4                   \\
        DHT 11                                                 & 4                   \\
        Módulo I2C                                             & 4                   \\
        Display LCD 16x2                                       & 4                   \\
        Fonte 5V 3W                                            & 4                   \\
        Sensor de presença embutido                            & 5                   \\
        Sensor de abertura sem-fio                             & 7                   \\
        Controle remoto                                        & 5                   \\
        Chave push button R13-507 sem trava preta              & 15                  \\
        Chave táctil 6x6x5mm 4 terminais                       & 10                  \\
        Rolo de solda Best azul 189 MSX10 60x40 1/2 kg fio 1mm & 1                   \\
        Placa de circuito impresso padrão 10x20cm tipo ilha    & 2                   \\
        Cabo de força                                          & 6                   \\
        Caixa patola PB-114/2 36x97x147                        & 4                   \\
        Relê T73 5V 1 pólo 2 posições 5 terminais 125V 10A     & 8                   \\
        Circuito integrado LM555 (NE555/NE555P)                & 4                   \\
        Buzzer 12mm com oscilador interno                      & 4                   \\
        Borne KF-3000 2 terminais                              & 10                  \\
        Borne KF-3000 3 terminais                              & 10                  \\
        Capacitor de tântalo 10µF                              & 4                   \\
        LDR                                                    & 4                   \\
        Capacitor poliéster 100nF                              & 8                   \\
        Chave gangorra KCD1-102 preta 3 terminais              & 5                   \\
        Resistor 1k                                            & 100                 \\
        Resistor 3k                                            & 100                 \\
        Resistor 3M3                                           & 100                 \\
        \bottomrule
    \end{tabular}
\end{table}

% TODO pegar lista de https://docs.google.com/document/d/1YicwPTyGzLodDYLabBcOs4gSv6esAMr_XmCm1hCHBmc/edit


\end{document}
