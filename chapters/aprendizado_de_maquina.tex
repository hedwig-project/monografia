\chapter{Aprendizado de Máquina e Análise de Dados}

Para o desenvolvimento de funcionalidades de aprendizado de máquina, será utilizada a linguagem Python, que possui diversos pacotes que facilitam sua utilização para implementar algoritmos de aprendizado, e funcionalidades para tratamento de dados. Além disso, é usada em vários outros âmbitos como cursos acadêmicos voltados ao ensino de programação e aplicações web, o que facilita a familiarização com o desenvolvimento nela.

\section{Coleta e Análise de Dados}
Um dos processos mais críticos para o sucesso de um projeto é a obtenção de resultados e dados, de onde se obterá conhecimento sobre o comportamento do sistema em atividade.

\subsection{Conexões}
Para melhor diagnóstico do estado de disponibilidade de conexão dos módulos, o seguinte vetor de parâmetros é monitorado ao longo do tempo, para cada módulo instalado:

\begin{description}
	\item [bit 0:] 1 se houve reinicialização do módulo, 0 se não;
	\item [bit 1:] 1 se houve reconexão de \wwifi, 0 se não;
	\item [bit 2:] 1 em caso de reconexão ao broker \wmqtt{}, 0 se não;
	\item [bit 3:] 1 em caso de reconexão a servidor para persistência de dados, 0 se não;
\end{description}

Desta forma, podemos acompanhar a relação entre problemas de conexão para posterior análise e tratamento.

\subsection{Uso}
Para monitorar o uso das funcionalidades dos produtos, há o seu monitoramento. Dessa forma, funcionalidades mais usadas podem ser melhoradas e funcionalidades não utilizadas podem ser excluídas, gerando um melhor retorno aos usuários.

Por exemplo, para um uso de automação da iluminação, temos:

\begin{description}
	\item [bit 0:] acionamento manual por botão físico no módulo;
	\item [bit 1:] acionamento manual por aplicativo de celular;
	\item [bit 2:] acionamento manual por página web;
	\item [bit 3:] acionamento automático.
\end{description}

\subsection{Coleta}

Para a coleta de dados, foram usados um total de onze módulos, distribuidos em locais e residências diferentes, de modo a simular maior diversidade de utilização.

\begin{enumerate}
	\item Aquário
	\item Corredor
	\item Lavanderia
	\item Sala/Cozinha
	\item Entrada
	\item Caixa d’água
	\item Victor
	\item Jarinu
	\item Daniela
	\item Hugo
	\item Gabriela
\end{enumerate}

Os módulos de 1 a 7 estão localizados na mesma residência, em Santo André. Já os módulos 9, 10 e 11 estão em casas diferentes na Grande São Paulo, e o módulo 8 está localizado na cidade de Jarinú-SP.

Para os módulos em Santo André, foi utilizado um dispositivo com cartão SD para persistência de dados. Sua análise tem como objetivo principal acompanhar parâmetros relacionados à disponibilidade, e sua coleta é local. O módulo de Jarinú possui uma interface com o sistema de alarmes (já instalado no local) e tem como objetivo obter dados de presença e abertura de portas (teste de conceito para validar possível integração futura com parceiros estratégicos).

Os demais módulos possuem coleta de dados a partir de persistência na nuvem e, passando pelo controlador local Morpheus, e possuem como principal objetivo prover dados para o Machine Learning.

Conforme destacado, os dados a seguir foram coletados localmente e possuem informações sobre disponibilidade e uso das funções pelo aplicativo backup (também com escopo local).

\begin{table}[hbp]
    \caption{Validação e Análise Final - de 10/09/2017 a 13/11/2017 - Aquário}
    \begin{tabular}{lc}
        \toprule
        \textbf{Descrição} & \textbf{Valor} \\
        \midrule
        Data Início                               				& 1                  \\
    \end{tabular}
\end{table}
