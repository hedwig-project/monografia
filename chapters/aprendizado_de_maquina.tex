\chapter{Aprendizado de Máquina}

\section{O Aprendizado de Máquina}
% TODO colocar referência ao urso do Andrew Ng
O termo Aprendizado de Máquina (\emph{Machine Learning}) surgiu em 1959, sendo utilizado pela primeira vez  pelo cientista americano Arthur Samuel. O termo foi por ele definido como "o campo de estudos que dá a um computador a habilidade de aprender sem ser explicitamente programado" (tradução livre dos autores). Em 1998, Tom Mitchel - outro cientista da computação americano - propôs uma explicação menos abstrata do termo da seguinte maneira: "um programa de computador aprende com a expriência E com respeito a uma classe de tarefas T e medida de performance P se sua performace nas tarefas T, sendo medida por P, aumenta com o aumento da experiência E".

% TODO colocar referencia https://hbr.org/2017/07/whats-driving-the-machine-learning-explosion
Apesar do conceito de aprendizado de máquina existir desde a década de 50, seu crescimento e relevância aumentaram apenas na última decada. Alguns fatores levaram a esse crescimento acelerado. Os principais foram o aumento da quantidade de dados - e aplicações de \emph{IoT} têm propulsionado esse aumento -, a melhora nos algoritmos, e o \emph{hardware} cada vez mais poderoso dos computadores. Esses últimos dois fatos permitem que a vasta quantidade de dados que possuímos possam ser analisados em tempo viável.

\section{Relação com o Projeto Hedwig}
No projeto Hedwig, a utilização de aprendizado de máquina tem o intuito de trazer melhorias de usabilidade do sistema ao usuário, além de poder trazer medidas de segurança à casa do mesmo. As melhorias de usabilidade podem ocorrer, por exemplo, quando o sistema aprende alguma rotina do usuário e se torna então capaz de prever quando determinada ação seria solicitada pelo usuário. A partir disso, ele então poderia passar a realizar tal ação automaticamente, ou sugerir a realização de tal ação para o usuário, sem que seja necessário que o usuário proativamente aja para solicitar a ação. No quesito de segurança, o sistema pode, a partir do aprendizado de rotinas, detectar ações suspeitas na casa, e opcionalmente agir para intervir quando tal tipo de ação é detectada.

% TODO possivelmente falar sobre a utilização de sistemas prontos em nuvem para ML

\section{Implementação}

\subsection{Linguagem, ferramentas e bibliotecas}

Uma das vantagens da popularização que se vê atualmente do uso de aprendizado de máquina é o surgimento de ambiente extremamente propício para a utilização de aprendizado de máquina, devido à facilidade de advém de grande comunidade trabalhando com o tema. A 

\subsection{Etapas}
\subsection{Algoritmos}
\subsection{Tratamento de Dados}
\subsection{Seleção de Características}