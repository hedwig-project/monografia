\chapter{Aprendizado de Máqina}

Para o desenvolvimento de funcionalidades de aprendizado de máquina, será utilizada a linguagem Python, que possui diversos pacotes que facilitam sua utilização para implementar algoritmos de aprendizado, e funcionalidades para tratamento de dados. Além disso, é usada em vários outros âmbitos como cursos acadêmicos voltados ao ensino de programação e aplicações web, o que facilita a familiarização com o desenvolvimento nela.

\section{Coleta de Dados}
Um dos processos mais críticos para o sucesso do diferencial do projeto (aplicações de Machine Learning) e para monitoramento da disponibilidade é a coleta de dados. Para cada módulo, os seguintes parâmetros serão monitorados:

\subsection{Conexões}
Para melhor diagnóstico do estado de disponibilidade de conexão dos módulos, o seguinte vetor de parâmetros é monitorado ao longo do tempo, para cada módulo instalado:

bit 0: 1 se houve reinicialização do módulo, 0 se não;
bit 1: 1 se houve reconexão de \wwifi, 0 se não;
bit 2: 1 em caso de reconexão ao broker \wmqtt{}, 0 se não;
bit 3: 1 em caso de reconexão a servidor para persistência de dados, 0 se não;

Desta forma, podemos acompanhar a relação entre problemas de conexão para posterior análise e tratamento.

\subsection{Uso}
Para monitorar o uso das funcionalidades dos produtos, há o seu monitoramento. Dessa forma, funcionalidades mais usadas podem ser melhoradas e funcionalidades não utilizadas podem ser excluídas, gerando um melhor retorno aos usuários.

	Por exemplo, para um uso de automação da iluminação, temos:
bit 0: acionamento manual por botão físico no módulo;
bit 1: acionamento manual por aplicativo de celular;
bit 2: acionamento manual por página web;
bit 3: acionamento automático.
