\chapter{Aprendizado de máquina}

	O termo Aprendizado de Máquina (\emph{Machine Learning}) surgiu em 1959, sendo utilizado pela primeira vez  pelo cientista americano Arthur Samuel. O termo foi por ele definido como "o campo de estudos que dá a um computador a habilidade de aprender sem ser explicitamente programado" (tradução livre dos autores). Em 1998, Tom Mitchel - outro cientista da computação americano - propôs uma explicação menos abstrata do termo da seguinte maneira: "um programa de computador aprende com a experiência E com respeito a uma classe de tarefas T e medida de performance P se sua performace nas tarefas T, sendo medida por P, aumenta com o aumento da experiência E" \cite{Coursera}.

	Apesar do conceito de aprendizado de máquina existir desde a década de 50, seu crescimento e relevância aumentaram apenas na última decada. Alguns fatores levaram a esse crescimento acelerado. Os principais foram o aumento da quantidade de dados coletados e disponíveis - e aplicações de \emph{IoT} têm propulsionado esse aumento -, a melhora nos algoritmos, e o \emph{hardware} cada vez mais poderoso dos computadores. Esses últimos dois fatores permitem que a vasta quantidade de dados que possuímos possam ser analisados em tempo viável \cite{hbrMlExplosion}.

	\section{Aprendizado de máquina no projeto Hedwig}

		No projeto Hedwig, a utilização de aprendizado de máquina tem o intuito de trazer melhorias de usabilidade do sistema ao usuário, além de poder trazer medidas de segurança à casa do mesmo.

		As melhorias de usabilidade podem ocorrer, por exemplo, quando o sistema aprende alguma rotina do usuário e se torna então capaz de prever quando determinada ação seria solicitada pelo usuário. A partir disso, ele então poderia passar a realizar tal ação automaticamente, ou sugerir a realização dela para o usuário, sem que seja necessário que o usuário proativamente aja para solicitar a mesma.

		No quesito de segurança, o sistema pode, a partir do aprendizado de rotinas, detectar ações suspeitas na casa (como por exemplo a abertura da porta de entrada em horário não usual), e opcionalmente agir para intervir quando tal tipo de ação é detectada.

		No Hedwig, foi desenvolvida uma análise baseada em dados coletados durante os meses de setembro e outubro, em um dos módulos que mantivemos instalados - o módulo de corredor, que já foi citado no capítulo anterior. A análise aqui descrita focou-se em gerar um algoritmo capaz de prever quando o usuário iria solicitar ao sistema a ativação do relé 1, que, no caso do módulo em análise está conectado à lampada do corredor.

		% TODO adicionar aqui caso mais análises tenham sido feitas

	\section{Implementação}
		% TODO possivelmente falar sobre a possiblidade de utilização de sistemas prontos em nuvem para ML
		% (Google Cloud, Tensor Flow, AWS ML, Microsoft (Luis?), IBM (Bluemix?))

		\subsection{Linguagem, ferramentas e bibliotecas}

			Uma das vantagens da popularização que se vê atualmente do uso de aprendizado de máquina é o surgimento de muitas facilidades para o desenvolvimento de aplicações que envolvam esse tema, devido às vantagens que advém de existir grande comunidade trabalhando com isso.

			% TODO maybe get some references for paragraph below
			Atualmente as duas principais linguagens sendo utilizadas para ciência de dados e aprendizado de máquina são Python e R. As duas são linguagens interpretadas, e que possuem funcionalidade REPL (Read-Eval-Print-Loop), possibilitando desenvolvimento altamente interativo e de fácil visualização de dados e gráficos enquanto se programa.

			Para o desenvolvimento de funcionalidades de aprendizado de máquina no Hedwig, foi escolhida a linguagem Python. Apesar de R ser uma linguagem que já nasceu voltada à análise de dados e tratamento estatístico, Python possui diversos pacotes e bibliotecas que conseguem adicionar tais funcionalidades à mesma. Esses pacotes estão altamente popularizados, e portanto é muito fácil percorrer suas documentações e procurar ajuda em fóruns online para acelerar o aprendizado de suas funcionalidades. Python, por sua vez, é uma linguagem de uso geral (\emph{general purpose language}), o que também tem seus lados positivos. Além de ser uma linguagem extremamente bem estabelecida, o fato de ser de uso geral implica que existem também pacotes e bibliotecas que possibilitam o uso de Python para desenvolvimento de aplicativos com funcionalidades de \emph{back-end} para serviços \emph{web}. Isso se torna altamente interessante por nos possibilitar facilmente transformar essa aplicação em um microsserviço, que interaja em tempo real com o servidor em nuvem do Hedwig, permitindo que o serviço de aprendizado de máquina aja em tempo real com toda a aplicação.

			Os principais pacotes sendo utilizados para auxiliar nas funcionalidades de análise de dados e aprendizado de máquina são:

			\begin{description}
				\item [Numpy] Pacote que dá à linguagem Python algumas facilidades para se lidar com estruturas numéricas como matrizes.
				\item [Pandas] Pacote para manuseio de dados. Dá facilidades para a importação de dados de fontes externas, e a utilzação dos mesmos a partir disso.
				\item [Scikit] Pacote com funcionalidades de análise de dados, e implementação dos principais algoritmos de aprendizado de máquina.
				\item [Pylab] Pacote que implementa funcionalidades para geração de gráficos
				% \item [Matplotlib]
				% \item [Seaborn]
			\end{description}

			A principal ferramenta utilizada durante o desenvolvimento de código de aprendizado de máquina chama-se Jupyter Notebook \footnote{http://jupyter.org/}. Essa ferramenta é uma aplicação \emph{web} de distribuição gratuita, que permite a criação e compartilhamento de documentos com código que pode ser interpretado interativamente, conforme se vai programando, e que, junto dos trechos de código executado coloca suas saídas, bem como gráficos gerados pelo trecho, além de permitir ao usuário colocar explicações ou textos gerais em liguagem Markdown, em meio ao código, de forma a explicá-lo. É uma ferramenta altamente utilizada para aplicações de análise de dados atualmente.

		\subsection{Etapas para o desenvolvimento de aplicação de aprendizado de máquina}

			% https://www.ibm.com/developerworks/community/blogs/jfp/entry/What_Is_Machine_Learning?lang=en
			% https://machinelearningmastery.com/machine-learning-in-python-step-by-step/
			% https://medium.com/dunder-data/how-to-learn-pandas-108905ab4955
			% https://www.digitalocean.com/community/tutorials/
			% how-to-build-a-machine-learning-classifier-in-python-with-scikit-learn  → fala sobre como funciona a divisao em pedaços do data set
			% TODO colocar referencia à seção direito

			O desenvolvimento de aplicações de aprendizado de máquina é caracterizado por algumas etapas.

			Inicialmente, é necessário explicitar qual o problema a ser resolvido. Na seção "Aprendizado de máquina no projeto Hedwig" foi exposto o problema aqui explorado.

			O próximo passo é a coleta de dados, para que os modelos possam ser treinados. Expusemos essa etapa no capítulo anterior.

			Em seguida, é necessário realizar algumas análises de dados e tratamentos dos mesmos, para que se possa sanitizá-los e deixá-los em formato propício ao uso pelos algoritmos. Também é necessário separar uma parcela dos dados para servir para treinamento dos modelos, e outra parcela para servir de dados de teste.

			Tendo isso pronto, pode-se então passar a uma etapa de treinamento de modelos. Para isso, é necessário fazer uma análise de quais algoritmos poderiam se encaixar bem para o problema em questão.

			A partir do treinamento dos modelos, e da separação de alguns dados para teste, é possível então avaliar quão bem os algoritmos estão performando para a previsão de resultados.

			O último passo costuma ser a elaboração do produto final: normalmente a integração sistema de aprendizado com outras aplicações ou então a elaboração de relatórios para que os modelos treinados para o problema sejam levados adiante.

			Nas próximas seções exploraremos o desenvolvimento desses últimos passos no projeto Hedwig.

		\subsection{Algoritmos}

				Existem duas classificações gerais para algoritmos de aprendizado de máquina: os de aprendizado supervisionado e os de aprendizado não supervisionado. Os algoritmos de aprendizado supervisionado são caracterizados por tratarem de problemas nos quais se espera conseguir prever uma determinada saída baseada em um ou mais dados de entrada. Já os algoritmos de aprendizado não supervisionado se caracterizam por tratarem de casos onde não há uma saída supervisionada. Nesses casos o objetivo é apenas de encontrar algum relacionamento ou estrutura nos dados de entrada \cite{islr}. O caso aqui estudado encaixa-se na definição de aprendizado supervisionado, já que temos nos dados de treinamento as saídas que desejamos que o modelo passe a conseguir prever.

				Outro plano no qual os algoritmos são divididos tem relação ao tipo de resposta que se espera deles. Há dois tipos principais de resposta: qualitativa e quantitativa  \cite{islr}. Os problemas para os quais a saída é uma variável quantitativa são chamados de problemas de regressão. Um dos métodos mais comuns para se tratar esse tipo de problema é o da regressão linear com utilização do método dos mínimos quadrados.

				Para os problemas nos quais a saída esperada é uma variável quantitativa, o nome dado é de problemas de classificação - já que o que se deseja obter como repsosta é a qual classe a resposta pertence. Um tipo de algortimo que costuma ser utilizado para problemas desse tipo é o de regressão logística.

				Adicionalmente, vale citar que há alguns tipos de algoritmos que podem ser utilizados tanto para problemas quantitativos quanto para problemas qualitativos, dentre eles: K vizinhos mais próximos (\emph{K-nearest neighbors}) e \emph{boosting} \cite{islr}.

				O algoritmo aqui em desenvolvimento então pode ser classificado como aprendizado supervisionado, de classificação (já que a saída esperada é do tipo "Sim, o usuário provavelmente gostaria de acender a luz nesse momento" ou "Não, o usuário provavelmente não gostaria de acender a luz").

		\subsection{Tratamento de dados}
			% http://scott.fortmann-roe.com/docs/BiasVariance.html
			% Falar sobre tratamento dos dados: undersampling/oversampling (fato de que a maioria dos dados de ativação é falso)

			Neste nosso primeiro estudo, o objetivo é o desenvolvimento de um modelo que aprenda a prever quando ocorrerá uma ativação do relé pelo usuário. No problema em análise, os dados que possuímos são:

			\begin{itemize}
				\item \emph{Timestamp} com dia e hora
				\item Luminosidade
				\item Temperatura
				\item Umidade
				\item Presença
				\item Ativação do relé (para o relé 1 e para o relé 2)
				\item Tipo de ativação (via botão, app backup, controle remoto, ativação agendada)
				\item Status dos relés
			\end{itemize}

			Para que possamos utilizar tais dados, alguns tratamentos são necessários.

			A primeira coisa que é necessária de ser levada em consideração é que a ativação pode ter ocorrido em decorrência de um agendamento. Nesse caso, devemos desconsiderar essa ativação, já que não se trata de uma ativação do usuário e não nos ajuda a prever quando o usuário irá deseajar ativar o relé.

			% TODO tratar timestamp
			% TODO separar ligar de desligar

			Outra questão a ser levada em consideração é a presença de uma disparidade entre a quantidade de amostras de treino que estão em uma classificação, em relação à quantidade que está em outra. No caso em análise, existem muito mais dados nos quais não ocorreu ativação do relé pelo usuário (71) do que os que ocorreram (6851). Ou seja, aproximadamente 99\% dos dados estão em uma classificação, e apenas 1\% na outra. Esse problema é conhecido como \emph{Class Imbalance Problem}. Para resolvê-lo, há duas táticas principais: \emph{oversampling} e \emph{undersampling}. Essa disparidade é um problema devido ao fato de que os algoritmos classificadores têm como objetivo a maximização da sua taxa de acerto. Dessa forma, teria alta tendência a sempre classificar os dados como pertencentes à classe majoritária. Isso é um problema principalmente em casos onde as situações de interesse são exatamente as que menos ocorre, como é o caso do problema aqui em exploração.

			% loans.bad_loans.value_counts()

			As duas táticas procuram balancear a quantidade de amostras em cada uma das classes. Pela tática de \emph{undersampling} isso é feito retirando-se algumas amostras da classe majoritária. Pela tática de \emph{oversampling}, isso é feito adicionando-se amostras da classe minoritária. Também existe a tática híbrida, que combina as duas anteriores.

			% The disadvantage with undersampling is that it discards potentially useful data. The main disadvantage with oversampling, from our perspective, is that by making exact copies of existing examples, it makes overfitting likely.
			% https://beckernick.github.io/oversampling-modeling/

			% https://pdfs.semanticscholar.org/9908/404807bf6b63e05e5345f02bcb23cc739ebd.pdf

			% http://www.chioka.in/class-imbalance-problem/

% SMOTE technique

% https://github.com/scikit-learn-contrib/imbalanced-learn
% 			@article{JMLR:v18:16-365,
% author  = {Guillaume  Lema{{\^i}}tre and Fernando Nogueira and Christos K. Aridas},
% title   = {Imbalanced-learn: A Python Toolbox to Tackle the Curse of Imbalanced Datasets in Machine Learning},
% journal = {Journal of Machine Learning Research},
% year    = {2017},
% volume  = {18},
% number  = {17},
% pages   = {1-5},
% url     = {http://jmlr.org/papers/v18/16-365}
% }


		\subsection{Treinamento dos modelos}

		In this chapter we discuss three of the most
widely-used classifiers: logistic regression, linear discriminant analysis, and
K-nearest neighbors. We discuss more computer-intensive methods in later
chapters, such as generalized additive models (Chapter 7), trees, random
forests, and boosting (Chapter 8), and support vector machines (Chap-
ter 9).

		\subsection{Resultados obtidos}

		There
		is no free lunch in statistics: no one method dominates all others over all
		possible data sets. On a particular data set, one specific method may work
		best, but some other method may work better on a similar but different
		data set. Hence it is an important task to decide for any given set of data
		which method produces the best results. Selecting the best approach can
		be one of the most challenging parts of performing statistical learning in
		practice.


		\subsection{Otimização dos modelos}

% citar que conforme fosse sendo usado o sistema iria aprendendo cada vez mais
