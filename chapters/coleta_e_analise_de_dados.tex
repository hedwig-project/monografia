\chapter{Coleta e Análise de Dados}

Um dos processos mais críticos para o sucesso de um projeto é a obtenção de resultados e dados, de onde se obterá conhecimento sobre o comportamento do sistema em atividade.

\section{Coleta}

Para a coleta de dados, foram usados um total de onze módulos, distribuidos em locais e residências diferentes, de modo a simular maior diversidade de utilização.

\begin{enumerate}
	\item Aquário
	\item Corredor
	\item Lavanderia
	\item Sala/Cozinha
	\item Entrada
	\item Caixa d’água
	\item Victor
	\item Jarinu
	\item Daniela
	\item Hugo
	\item Gabriela
\end{enumerate}

Os módulos de 1 a 7 estão localizados na mesma residência, em Santo André. Já os módulos 9, 10 e 11 estão em casas diferentes na Grande São Paulo, e o módulo 8 está localizado na cidade de Jarinu-SP.

\begin{figure}[H]
	\centering
	\caption{Log na página do Aplicativo Backup}
	\includegraphics[width=0.5\textwidth]{logAppBackup}
	\label{fig:logAppBackup}
\end{figure}

\begin{figure}[H]
	\centering
	\caption{Módulo com cartão SD para coleta local de dados}
	\includegraphics[width=0.3\textwidth]{SDColetaDados}
	\label{fig:SDColetaDados}
\end{figure}

Para os módulos em Santo André, foi utilizado um dispositivo com cartão SD para persistência de dados. Sua análise tem como objetivo principal acompanhar parâmetros relacionados à disponibilidade, e sua coleta é local.

\begin{figure}[H]
	\centering
	\caption{Módulos instalados em Santo André}
	\includegraphics[width=0.8\textwidth]{ModulosStoAndre}
	\label{fig:ModulosStoAndre}
\end{figure}

O módulo de Jarinu possui uma interface com o sistema de alarmes (já instalado no local) e tem como objetivo obter dados de presença e abertura de portas (teste de conceito para validar possível integração futura com parceiros estratégicos).

\begin{figure}[H]
	\centering
	\caption{Controle do Sistema de Alarme e Módulo de Interface de Jarinu}
	\includegraphics[width=0.8\textwidth]{ModuloSistAlarme}
	\label{fig:ModuloSistAlarme}
\end{figure}

\begin{figure}[H]
	\centering
	\caption{Módulo Básico e sensor de presença no corredor de Jarinu}
	\includegraphics[width=0.5\textwidth]{BasicSensorPresJarinu}
	\label{fig:BasicSensorPresJarinu}
\end{figure}

\begin{figure}[H]
	\centering
	\caption{Sensor de abertura das portas da cozinha (à esquerda) e da sala (à direita)}
	\includegraphics[width=0.5\textwidth]{SensorPortasJarinu}
	\label{fig:SensorPortasJarinu}
\end{figure}

Os demais módulos possuem coleta de dados a partir de persistência na nuvem e, passando pelo controlador local Morpheus, e possuem como principal objetivo prover dados para o Machine Learning.

\begin{figure}[H]
	\centering
	\caption{Módulo instalado para coleta de dados do nível da caixa d’água}
	\includegraphics[width=0.8\textwidth]{ModuloCxAgua}
	\label{fig:ModuloCxAgua}
\end{figure}

Conforme destacado, os dados a seguir foram coletados localmente e possuem informações sobre disponibilidade e uso das funções pelo aplicativo backup (também com escopo local).

De modo geral, temos a coleta dos seguintes dados:

\begin{figure}[H]
	\centering
	\caption{Relação dos módulos e dados coletados}
	\includegraphics[width=1.0\textwidth]{DiagramaModulosDados}
	\label{fig:DiagramaModulosDados}
\end{figure}

Foram coletados dados desde o dia 10 de setembro de 2017 até o dia 13 de novembro de 2017, com processo semanal de backup. Os dados foram segregados, suas duplicatas foram removidas e dados dos diversos arquivos de log foram consolidados.

\begin{itemize}
	\item \textbf{Dados do aquário}: temperatura da água, estado do aquecedor e estado da lâmpada usada como fonte de iluminação artificial;
	\item \textbf{Dados de disponibilidade}: memória livre disponível (“free heap”, em casos de pouca memória disponível, o espaço de dados invade o espaço de programa e ocorre travamento do ESP); “loops” (monitoramento do contador de loops de 1 segundo, mas que devido a espera ou demasiado processamento demoraram mais que 2 segundos para ocorrerem) e monitoramento de desconexões, reconexões, updates do firmware dos módulos e reinícios;
	\item \textbf{Luminosidade}: monitoramento a partir de dado analógico de um LDR;
	\item \textbf{Temperatura e Umidade}: sensor DHT;
	\item \textbf{Presença}: Sensor de Presença PIR;
	\item \textbf{Uso dos Reles}: pelo botão físico presente no módulo, pela página web do aplicativo backup, automaticamente por regras configuradas pelo usuário, e por controle de radiofrequência (rf) (nos módulos usados para coleta, não havia a dashboard disponível);
	\item \textbf{Abertura do Portão e Estado do Portão}: monitoramento do estado do portão por sensor eletromagnético com fio, conectado diretamente ao módulo, e acompanhamento das aberturas por pessoa (fabio, victor ou admin - alguma das outras duas pessoas da casa), uso do sistema (aberturas pelo sensor x aberturas pelo sistema) e monitoramento de uma porta intermediária da escada, por meio de sensor de abertura sem fio (comunicação por radiofrequência);
	\item \textbf{Nível máximo da caixa d’água}: sensor de boia próximo ao nível máximo;
	\item \textbf{Porta da Sala (Estado) e Presença (Corredor e Cozinha) de Jarinu}: equipamentos próprios do sistema de alarmes.

\end{itemize}

\section{Análise}

\begin{table}[H]
    \caption{Análise consolidada de disponibilidade}
    \setlength\tabcolsep{1.5pt}
    \centering
    \footnotesize
    \begin{tabular}{cccccc}
        \textbf{Item} & \textbf{Aquário} & \textbf{Corredor} & \textbf{Lavanderia} & \textbf{SalaCozinha} & \textbf{Entrada} \\
        \midrule
        memória livre [bytes] &
        30278 &
        27218 &
        26990 &
        26728 &
        26237 \\
        \# (loops) &
        1251 &
        771 &
        76 &
        925 &
        1140 \\
        \# (reinícios) &
        34 &
        18 &
        35 &
        20 &
        91 \\
        \# (desconexões do WiFi) &
        15 &
        26 &
        24 &
        27 &
        77 \\
    \end{tabular}
\end{table}

Ao analisar o aspecto de disponibilidade a partir da média da memória disponível (quanto maior, melhor), o número de loops de 1s que ocorreram em 2s ou mais (indicando a existência de procedimentos que estão atrasando a correta execução de rotinas periódicas, como aquelas de “keep alive” de comunicação), o número de reinícios e desconexões da rede WiFi (isentos de updates de firmware, isto é, eventos de reinício e desconexões desconsiderando aqueles causados por updates de firmware dos módulos).

Como o firmware do Corredor, Lavanderia e SalaCozinha é o mesmo (basic), conclui-se que:

\begin{itemize}
	\item A versão aquário possui mais memória disponível, possui um menor número de desconexões da rede WiFi e número perto da média de reinícios. Contudo, possui maior ocorrência de loops de 1s maiores que 2s, indicando algum procedimento que está consumindo tempo do loop;
	\item A versão basic possui ocorrências de desconexões, reinícios e memória livre em valores médios;
	\item A versão entrada possui o maior número de reinícios e desconexões da rede WiFi (o que já era esperado, pois é o módulo mais longe do roteador da residência, que está em outro andar);
	\item A memória livre média das versões basic e entrada possuem valores próximos.
\end{itemize}

\begin{table}[H]
	\caption{Análise consolidada de uso dos relês}
	\setlength\tabcolsep{1.5pt}
	\centering
	\footnotesize
	\begin{tabular}{ccccccc}
		\textbf{Nome} &
		\textbf{Corredor} &
		\textbf{Sabrina} &
		\textbf{Varanda} &
		\textbf{Escada} &
		\textbf{Sala} &
		\textbf{Cozinha} \\
		\midrule
		\% Ligado &
		35.08\% &
		22.91\% &
		41.62\% &
		6.89\% &
		80.06\% &
		60.99\% \\
		\# Acionamentos &
		238 &
		202 &
		516 &
		382 &
		190 &
		233 \\
		\% Botão Físico &
		11.61\% &
		87.96\% &
		54.14\% &
		13.57\% &
		26.21\% &
		20.33\% \\
		\% Web &
		21.43\% &
		2.55\% &
		14.79\% &
		1.36\% &
		44.66\% &
		40.66\% \\
		\% RF &
		37.50\% &
		9.49\% &
		0.00\% &
		0.00\% &
		29.13\% &
		39.00\% \\
		\% Auto &
		29.46\% &
		0.00\% &
		31.08\% &
		85.07\% &
		0.00\% &
		0.00\% \\
	\end{tabular}
\end{table}

Quanto ao uso dos acionamentos possíveis (botão físico presente no módulo, pelo aplicativo backup - web, por controle de radiofrequência - rf e por regra automática configurada pelo usuário - auto), número de acionamentos e porcentagem do tempo ligado, observa-se que:

\begin{itemize}
	\item No caso de áreas comuns, temos as maiores porcentagens de tempo de lâmpadas acesas, com maiores números de acionamentos pela web, nenhum acionamento por regra automática, dobro de acionamentos por rf do que botão físico para a lâmpada da cozinha e números de acionamentos por rf e botão físico em valores próximos para a Sala;
	\item Para a escada, temos o melhor uso do acionamento automático (indicando que a regra de acender a lâmpada da escada quando a porta de entrada abrir em horário noturno é bem empregada);
	\item Para a lâmpada da varanda, ainda que com um bom uso automático, temos cerca de metade dos acionamentos ocorrendo por meio do botão físico;
	\item Quarto da Sabrina possui nenhum uso de acionamento automático, e poucos acionamentos pela web (celular) e controle rf, indicando que o sistema não agrega muito valor;
	\item Lâmpada do corredor possui uma distribuição aproximadamente igualitária entre os acionamentos, contrariando o comportamento esperado de que as regras automáticas de acendimento de luz quando da presença detectada seriam suficientes para o controle dessa lâmpada;
	\item O uso de acionamentos pelo aplicativo (web) somente são consideráveis nas áreas comuns (provavelmente quando os usuários desligam as luzes da casa toda antes de dormir). Pode-se considerar que, apesar de ser um meio adequado de apresentar o funcionamento do sistema, no uso cotidiano os usuários preferem usar controles rf, acionamento manual por botão físico ou deixar uma regra para atuação automática.
\end{itemize}

\begin{figure}[H]
	\centering
	\caption{Curva diária Santo André -- terça-feira}
	\includegraphics[width=0.8\textwidth]{diaStoAndreTerca}
	\label{fig:diaStoAndreTerca}
\end{figure}

Tomando o dia de terça como exemplo dos gráficos consolidados de Santo André como exemplo, observa-se que:

\begin{itemize}
	\item Há um pico de presença no corredor às 3:00 a.m. Seria o horário em que os gatos da
	residência estão mais ativos;
	\item A curva de presença indica maiores movimentos entre 9hs e 13hs, e entre 17hs e 01hs (horários em que as pessoas entram ou saem da residência);
	\item Considerando que a segmentação de usuários da entrada está em Fabio, Victor e Admin (Nair e Sabrina):
	\begin{itemize}
		\item Das 4hs às 7hs, Nair/Sabrina sai da casa;
		\item Das 9hs às 13hs, Fabio sai de casa, algumas vezes usando seu usuário, outras vezes manualmente;
		\item Às 10hs, Victor sai de casa;
		\item Das 17hs às 19hs, Sabrina/Nair voltam para casa;
		\item Entre 19hs e 22hs, Victor volta para casa;
		\item Entre 21hs e 23hs, Fabio volta para casa;
	\end{itemize}
	\item Os picos de porta escada indicam os horários de maior entrada e saída da residência.
\end{itemize}

\begin{figure}[H]
	\centering
	\caption{Temperatura e aquecedor -- Módulo do Aquário}
	\includegraphics[width=1.0\textwidth]{TempAquecedorAqua}
	\label{fig:TempAquecedorAqua}
\end{figure}

Considerando a curva diária e histórico do período dos sensores e atuador do aquário, observa-se que:

\begin{itemize}
	\item Às 3hs da manhã, temos lâmpada acesa e aquecedor aceso. Isso pode ter ocorrido em um dia em particular (25/11/2017, dia atípico, como observado no gráfico do período);
	\item Das 9hs às 23hs, há uma regra para que a lâmpada do aquário fique acesa;
	\item A temperatura é menor às 10hs, e das 14hs às 18hs. O comportamento do aquecedor é no sentido de ligar nesses períodos.

\end{itemize}

\begin{figure}[H]
	\centering
	\caption{Consolidado Diário -- Módulo do Aquário}
	\includegraphics[width=1.0\textwidth]{AquaDia}
	\label{fig:AquaDia}
\end{figure}

\begin{figure}[H]
	\centering
	\caption{Memória e loops -- Módulo do Corredor}
	\includegraphics[width=1.0\textwidth]{MemLivreCorredor}
	\label{fig:MemLivreCorredor}
\end{figure}

Observando sua curva diária de memória livre e loops de 1s que foram executados em mais de 2s, temos que:

\begin{itemize}
	\item Os períodos de maior disponibilidade são das 11hs às 16hs, pois temos maior memória livre (parâmetro mais crítico para possíveis reinícios e travamentos);
	\item Os períodos de menor disponibilidade são das 21hs às 22hs, onde tivemos picos de loops demorados e pouquíssima memória livre.
\end{itemize}

\begin{figure}[H]
	\centering
	\caption{Consolidado no período -- Módulo do Corredor}
	\includegraphics[width=1.0\textwidth]{periodoCorredor}
	\label{fig:periodoCorredor}
\end{figure}

Observamos, em verde, no gráfico acima, os updates de firmware (possibilidade de acompanhamento de melhoria de disponibilidade após a implantação de novas versões), a quantidade de reinícios e ocorrências de desconexão do WiFi. Entre 24/10 e 28/10, tivemos grande número de desconexões do WiFi, indicando indisponibilidade da rede nesses dias, que causaram certa quantidade de reinícios no mesmo período.

\begin{figure}[H]
	\centering
	\caption{Consolidado diário dos sensores -- Módulo do Corredor}
	\includegraphics[width=1.0\textwidth]{sensoresdiaCorredor}
	\label{fig:sensoresdiaCorredor}
\end{figure}

No gráfico acima, observamos inatividade às 4hs da manhã e entre as 14hs e 16hs. A temperatura possui pouca variação, assim como a umidade. Já a curva de luminosidade acompanha a de presença (apenas das 14hs às 16hs, considere que temos luz natural, por isso não é um valor próximo àquele apresentado de madrugada, entre 0hs e 4hs.

\begin{figure}[H]
	\centering
	\caption{Uso dos acionamentos no período para a lâmpada -- Módulo do Corredor}
	\includegraphics[width=1.0\textwidth]{UsoRele1periodoCorredor}
	\label{fig:UsoRele1periodoCorredor}
\end{figure}

No primeiro gráfico, observamos a evolução do uso dos acionamentos para o rele 1 (lâmpada do corredor) no período. Observam-se dias com maiores e menores quantidades de acionamentos, e a participação das regras automáticas em relação aos outros tipos de acionamento. No segundo, verifica-se que, no dia, os acionamentos automáticos ocorrem entre 17hs e 22hs. Possivelmente, podemos aplicar regras no período das 23hs às 2hs.

\begin{figure}[H]
	\centering
	\caption{Uso dos acionamentos por dia para a lâmpada -- Módulo do Corredor}
	\includegraphics[width=0.8\textwidth]{UsoRele1DiaCorredor}
	\label{fig:UsoRele1DiaCorredor}
\end{figure}

\begin{figure}[H]
	\centering
	\caption{Curva diária -- Módulo de Acesso}
	\includegraphics[width=1.0\textwidth]{EntradaConsolidadoDia}
	\label{fig:EntradaConsolidadoDia}
\end{figure}

Com a curva diária de comportamento para o módulo de entrada, observa-se que:

\begin{itemize}
	\item Os horários com maior número de acionamentos pelo celular são 10hs, 12hs, 19hs, 21hs. 22hs e 23hs;
	\item Usuários que entram ou saem de casa às 17hs não usam o celular para abrir o portão, tampouco aqueles que entram ou saem às 5hs.
\end{itemize}

\begin{figure}[H]
	\centering
	\caption{Nível da caixa d'água -- consolidado diário}
	\includegraphics[width=1.0\textwidth]{CxAguaDia}
	\label{fig:CxAguaDia}
\end{figure}

Na curva diária do módulo da caixa d’água, nota-se que:

\begin{itemize}
	\item Há maior uso de água (provavelmente para banho) nos períodos das 9hs às 12hs, e entre 16hs e 23hs;
	\item Os horários em que temos, provavelmente, menor disponibilidade de água da rua (para preencher a caixa e fazer o nível voltar a ser alto) são das 5hs às 6hs da manhã e às 11hs da manhã.

\end{itemize}

Observando o gráfico a seguir, observam-se os dias em que houve falta de água (picos de demora para preencher a caixa de água após algum uso perceptível pelo sensor do tipo boia).

\begin{figure}[H]
	\centering
	\caption{Tempo para encher a caixa d'água -- período}
	\includegraphics[width=1.0\textwidth]{tempoPeriodocxAgua}
	\label{fig:tempoPeriodocxAgua}
\end{figure}

\begin{figure}[H]
	\centering
	\caption{Atividade da porta -- Sala de Jarinu}
	\includegraphics[width=1.0\textwidth]{AtivPortaSalaJarinu}
	\label{fig:AtivPortaSalaJarinu}
\end{figure}

Em Jarinu, temos um idoso, com boa saúde física, que mora sozinho em sua residência. Ao analisar as curvas diárias de atividade da porta da sala e presença, temos:

\begin{itemize}
	\item Atividade de presença ente 4hs e 5hs da manhã; entre 10hs e 12hs, e entre 16hs e 20hs.
	\item Atividade de presença maior em domingos, segundas e terças;
	\item Grande atividade da porta da sala às 10hs e 14hs. O pico das 10hs ocorre principalmente aos domingos.
\end{itemize}

\begin{figure}[H]
	\centering
	\caption{Atividade da presença -- Sala de Jarinu}
	\includegraphics[width=1.0\textwidth]{AtivPresencaJarinu}
	\label{fig:AtivPresencaJarinu}
\end{figure}

\begin{figure}[H]
	\centering
	\caption{Consolidado no período -- Jarinu}
	\includegraphics[width=1.0\textwidth]{JarinuPeriodo}
	\label{fig:JarinuPeriodo}
\end{figure}

Ao analisar por período, podemos verificar o aumento ou diminuição de atividade, o que pode ser um indicativo de bem estar. Por exemplo, no gráfico acima verifica-se que o dia 04/11 foi um dia atípico, com menores presença e atividade da porta da sala. Em geral, não houve tendência de crescimento ou diminuição do nível de atividade geral no período observado.
