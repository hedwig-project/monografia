\section{Comunicação}
Conforme explicado anteriormente, neste projeto serão utilizados tanto protocolos de comunicação próprios quanto os criados por terceiros. A arquitetura desenvolvida aqui busca viabilizar a robustez do sistema, trabalhando em um nível local e outro nível remoto, onde o usuário terá o controle de sua casa por meio do \textit{smartphone} ou computador pessoal.

O serviço em nuvem recebe as requisições do usuário por meio de um cliente web ou nativo. Esse servidor processa as requisições, aplicando os filtros de segurança necessários, de modo a consultar a autenticidade do pedido e verificar se aquele usuário possui as permissões necessárias para o serviço que deseja operar. Os serviços da nuvem se comunicam com o servidor local da casa requisitada, o qual também aplica os filtros de segurança necessários, e realiza a comunicação com os módulos.

A infraestrutura de comunicação entre a nuvem e o servidor local, e o servidor local e os sensores e atuadores utiliza o protocolo de aplicação \wmqtt{}, referência em aplicações \wiot{} no mundo. O protocolo \wmqtt{} é estabelecido em cima dos protocolos TCP/IP (nas camadas inferiores) e é orientado à sessão, diferentemente do protocolo HTTP, de mesma camada \cite{ibmMqtt}.

O protocolo \wmqtt{} é do tipo Pub/Sub (de \textit{publisher/subscriber}) e é estritamente orientado à tópicos. Assim, um \textit{subscriber} se inscreve a um tópico de seu interesse, e recebe todas as publicações que um \textit{publisher} realizar. Os tópicos são organizados com estrutura semelhante a de um sistema de arquivos Unix, com níveis hierárquicos separados por barras, de modo que o subscriber pode se inscrever para tópicos utilizando os \textit{wildcards} * e +, os quais são válidos para mais de um nível e um único nível, respectivamente \cite{mqttDocumentation}.

Para interconectar os tópicos, com \textit{publishers} e \textit{subscribers}, é necessário um agente que realiza a transmissão das mensagens, e que garante a segurança e confiabilidade. Esse agente é conhecido como Broker (em versões anteriores) ou Server (na versão atual, V3.1.1). O \textit{broker} irá permitir ou negar a subscrição ou a publicação a determinado tópico. A segurança da troca de mensagens é realizada por meio do protocolo TLS (\textit{Transport Layer Security}) que encripta os segmentos na camada de transporte.

O protocolo \wmqtt{} também oferece três tipos de QoS (\textit{Quality of Service}), possibilitando: diminuir o overhead ao máximo, enviando a mensagem uma única vez, na configuração mais simples; garantir que a mensagem seja entregue no mínimo uma vez, na configuração de segundo nível; garantir que a mensagem seja entregue exatamente uma vez, no terceiro nível, o que aumenta o overhead, consequentemente.

As mensagens são transmitidas em texto puro, e é necessário estabelecer um protocolo para a sua utilização. Foi desenvolvido um protocolo de fácil utilização e com baixo overhead, mas que pudesse ser expansível e flexível aos casos de uso desejados.

O \textit{broker} Mosquitto\footnote{https://mosquitto.org/} será utilizado, e foi escolhido por ser amplamente adotado em projetos de \wiot{}, além de ser open source e com licença abrangente (MIT). Entretanto, há diversas possibilidades, como o HiveMQ, adotado no projeto HomeSky, e com grande uso em aplicações enterprise \cite{hiveMq}.

A arquitetura de comunicação é representada pela Figura \ref{fig:diagramaComunicacao}, com um alto nível de abstração, cujos detalhes serão vistos, com granularidade menor na Seção \ref{chap:morpheus}.

\begin{figure}[H]
	\centering
	\caption{Visão alto nível da comunicação no Hedwig}
  \includegraphics[width=0.8\textwidth]{arquiteturaHedwig}
\label{fig:diagramaComunicacao}
\end{figure}

%\subsection{Entre módulos e controlador local}
%\subsection{Entre controlador local e nuvem}
%\subsection{Entre cliente web e nuvem}
%\subsection{Entre app backup e módulos}
