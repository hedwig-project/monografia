\section{Cliente Web}
Para criar aplicações web que demonstrem as funcionalidades dos módulos de automação da casa, foi escolhida a biblioteca de JavaScript React, que permite o fácil desenvolvimento de aplicações single-page, renderizadas do lado do cliente, e que permitem a atualização dinâmica da página, de forma fluida, rápida, o que acaba enriquecendo a experiência do usuário na aplicação (UI e UX). Esse cliente irá se comunicar com a API, por meio do protocolo HTTP, e utilizando autenticação de usuário por meio  de tokens do tipo JSON Web Token. JSON Web Tokens são tokens gerados no cadastro ou login do usuário, e são enviados ao browser, onde são armazenados na Localstorage do mesmo.

A partir desse momento, todas as requisições ao back end conterão tal token no campo de Authentication do cabeçalho dos métodos HTTP (GET, PUT, POST, DELETE). Somente requisições contendo tal token, e cujo token seja válido, são aceitas.

Outro ponto interessante para a utilização da biblioteca React é que, com a biblioteca React Native - uma extensão da biblioteca React - é possível a geração de aplicativos nativos para iOS e Android, que podem vir a ser desenvolvidos no desenrolar do projeto. Isso diminui a necessidade de retrabalho e dispensa a necessidade de estudo aprofundado das linguagens e ambientes de desenvolvimento tradicionais de projeto de aplicativos nativos.
