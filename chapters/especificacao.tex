\chapter{Especificação}

\section{Stakeholders}
Um dos passos iniciais na elaboração de um projeto é a determinação das partes interessadas. Com esse conhecimento, pode-se entender as necessidades dos diferentes perfis de clientes e as expectativas desses grupos em relação ao uso do produto. Por meio da análise de aplicação, tanto do projeto desenvolvido quanto dos dispositivos existentes, pode-se destacar os seguintes grupos dentre os potenciais consumidores:

\begin{itemize}
\item Pessoas que moram sozinhas e seus familiares;
\item Pessoas que buscam comodidade no uso e controle de dispositivos domésticos;
\item Pessoas preocupadas com o consumo de água e energia elétrica.
\end{itemize}

Considerando o Censo de 2010 \cite{ibge}, pode-se estimar as classes de consumidores para a cidade de São Paulo:

\begin{itemize}
\item Considerando que 1/10 da população com mais de 60 anos more sozinha e que 1/4 deles adquiriria o produto, temos uma estimativa de 33 mil consumidores. Como essa população está envelhecendo em taxas cada vez maiores (8,96\% em 2000 contra 13,6\% em 2016) \cite{bibliotecaVirtual}, a tendência é que essa classe aumente;
\item Considerando que 1/100 dos domicílios ocupados tenha uma pessoa com esse perfil, temos uma estimativa de 35 mil consumidores em potencial;
\item Considerando que cerca de 70\% das residências reduziram o consumo com campanhas de redução de uso de água em 2015 \cite{g1}, supondo que 5\% ficariam preocupados/interessados ao nível de se tornarem consumidores, temos uma estimativa de 71 mil consumidores em potencial.
\end{itemize}

\section{Requisitos}

\subsection{Requisitos Funcionais}
\begin{itemize}
\item O sistema deve permitir o monitoramento de aparelhos do dia a dia, dentro de uma residência, em módulos independentes;
\item O sistema deve ser capaz de enviar notificações aos usuários, por meio de um aplicativo web ou \textit{mobile};
\item O sistema deve poder ser personalizável pelo usuário, o qual pode adquirir novos módulos ou retirar algum já existente;
\item O sistema deve ser capaz de aprender a respeito de cada usuário, utilizando conceitos de Machine Learning. O aprendizado de máquina é responsável por detectar padrões no comportamento do usuário, e a partir disso, sugerir ao usuário ações a serem tomadas automicamente pelo sistema;
\item O sistema deve manter backup de dados do controlador local na nuvem;
\item O sistema deve permitir a usuários se cadastrarem no mesmo, pela plataforma que melhor lhe convier;
\item O usuário poderá cadastrar sua casa na plataforma, podendo ter uma ou mais casas cadastradas;
\item O usuário poderá cadastrar os módulos dentro de uma casa, sendo que uma casa pode ter vários módulos, e cada módulo só poderá existir em uma casa;
\item O usuário pode efetuar as operações de remoção e modificação nos seus módulos e casas;
% TODO rever item abaixo, acho que talvez essa funcao de reset nao tenha ficado muito claro. Tipo, o que ela faz, qual parte do sistema ela afeta, etc. Dado que nesse contexto, o "sistema" é tudo, ou seja, API, etc tb. E esse reset nao afetaria API. Entao acho que vale a reescrita desse requisito
\item O sistema deve possuir uma função de reset de fácil utilização.
\end{itemize}

\subsection{Requisitos Não-Funcionais}
O levantamento de requisitos não-funcionais foi realizado com base na norma ISO25010:2011 \cite{iso25010}.

\begin{itemize}
\item Os módulos que compõem o sistema dentro de uma residência devem ser independentes entre si, devendo obedecer a uma interface comum de integração com o core do projeto, para que seja facilitada a ampliação e a inserção de novos módulos, com outras funcionalidades. Haverá validação com o desligamento de um módulo e verificação do comportamento dos demais;
\item O sistema deve garantir segurança dos dados por meio de protocolo de comunicação seguro, tanto para o controle de acesso à API por usuários autenticados quanto para impedir que dados sejam interceptados em sua transmissão;
\item O banco de dados deve possuir acesso restrito e estar hospedado em servidor de alta segurança;
% TODO como assim "em menos de 10 minutos"?
\item O sistema deve ser robusto, de modo a continuar operando, mesmo com menor nível de funcionalidades, quando da ocorrência de falhas na comunicação com a nuvem (indisponibilidade parcial devido a problemas com os servidores remotos, ou total com perda da conexão com a Internet) ou falhas na rede local (indisponibilidade da conexão com a rede local). Também deve se recuperar em caso de travamento total do módulo e continuar funcionando em caso de DoS Local. Para validação, haverão testes de indisponibilidade de servidor, conexão com a internet, rede local e DoS local, e observação da continuidade de serviço de atuação na iluminação da casa e abertura do portão em menos de 10 minutos;
% TODO era bom falar algo sobre os dois itens abaixo (tipo, se a gente fez algo pra tentar seguir essa meta)
\item O sistema deve apresentar disponibilidade de 99,9\% - cerca de 8 horas de indisponibilidade por ano -, não levando em consideração problemas com a conexão de internet da residência;
\item O sistema deve ser escalável a até 10 mil usuários, sem perdas de desempenho consideráveis, ou aumento na latência para as requisições serem atendidas;
\item O sistema deve possuir instalação intuitiva e simplificada.
\end{itemize}

\subsection{Requisitos por Nível de Conectividade}

Nas Tabela \ref{table:descricaoRequisitos} são descritos os requisitos, bem como a técnica de avaliação utilizada para demonstrá-lo. A Tabela \ref{table:niveisFuncionalidades} mostra tais requisitos de acordo com o nível de funcionalidade.

\begin{table}[hbp]
	\centering
	\caption{Descrição de requisitos}
	\resizebox{\textwidth}{!}{%
    \begin{tabular}{clp{8cm}p{8cm}}
        \toprule
        \textbf{ID} 			& \textbf{Requisito} 					& \textbf{Avaliação} 																					& \textbf{Descrição} \\
        \midrule
		1 						& Automação Residencial					& Demonstração das funcionalidades listadas															& Controle por App ou no módulo \\
		2 						& Monitoramento Residencial				& Demonstração das funcionalidades listadas 														& Monitoramento no App e displays dos módulos \\
		3 						& Modularidade							& Funcionamento de módulo ``stand-alone'' 															& Soluções modulares \\
		4 						& Escalabilidade para n residências		& Testes de carga, acessos simultâneos 																& Suporta carga, acessos simultâneos \\
		5 						& Machine Learning						& Tratamento de séries de dados reais 																& Análise dos dados para levantamento de rotinas e reconhecimento de padrões \\
		6 						& Backup de dados						& Funcionalidade de Backup de dados no controlador local 											& - \\
		7 						& Tolerância a falhas					& Estatísticas de keep alive, ping local, ping internet de módulo base com e sem as melhorias 		& Manter funcionalidade Online, Local e Offline \\
		8 						& Segurança da informação				& Protocolos seguros, controle de acesso 															& Protocolos seguros, controle de acesso \\
        \bottomrule
	\end{tabular}}
	\label{table:descricaoRequisitos}
\end{table}

\newcommand{\specialcell}[2][c]{%
 	\begin{tabular}[#1]{@{}l@{}}#2\end{tabular}}

\begin{table}[hbp]
	\centering
	\caption{Níveis de funcionalidades}
	\resizebox{\textwidth}{!}{%
    \begin{tabular}{cp{7cm}p{8cm}p{8cm}}
        \toprule
        \textbf{ID} 			& \textbf{Online} 												& \textbf{Local} 																					& \textbf{Offline} \\
        \midrule
		1 						& Usar parâmetros para controle inteligente de lâmpada e despertador (Quarto)												& \specialcell[c]{Configurar lâmpada automática (Quarto) \\ Configurar despertador (Quarto)}								& Ligar e desligar lâmpada (Quarto) - Módulo 									 				\\
		2 						& \specialcell[c]{Estado do portão (Acesso) \\ Temperatura, Umidade (Todos) \\ Presença (Quarto)}							& \specialcell[c]{Estado do portão (Acesso) \\ Temperatura, Umidade (Todos) \\ Presença (Quarto)}							& \specialcell[c]{Desativar despertador (Quarto) - Módulo \\ Destravar porta (Acesso) - App} 	\\
		3 						& -																															& Inserção automática\slash reconhecimento (Todos)																			& \specialcell[c]{Temperatura, Umidade (Todos) - Módulo \\ Presença (Quarto) - Módulo} 			\\
		4 						& -																															& -																															& - 																							\\
		5 						& \specialcell[c]{Derivações de Regras \\ Perfil Esperado de Comportamento}													& Alerta de Portão Aberto (Acesso)																							& - 																							\\
		6 						& Redundância na nuvem																										& Armazenamento de dados temporariamente num cartão SD, no controlador local 												& Alerta de Portão Aberto (Acesso) - Módulo 													\\
		7 						& -																															& Notificações e alertas no App (Monitoramento estado módulos) 																& Circuito ``Keep Alive'' - Hard Reset e Soft Reset 											\\
		8 						& Autenticação para as funcionalidades descritas nesta coluna																& Autenticação para as funcionalidades descritas nesta coluna 																& Autenticação para as funcionalidades descritas nesta coluna - Módulo  						\\
        \bottomrule
	\end{tabular}}
	\label{table:niveisFuncionalidades}
\end{table}
