\chapter{Conclusões}

O principal foco deste trabalho foi a criação de uma arquitetura abrangente e robusta para automação residencial, com sua implementação completa --- desde os módulos de hardware à aplicação cliente.

A fim de demonstrar todas as funcionalidades dos módulos, por meio do aplicativo web, optou-se pelo modelo de \textit{dashboard}, que promove interação com fluidez e concisão. Porém, a arquitetura permite que outros tipos de aplicação possam se comunicar com as casas. O processamento de linguagem natural com um \textit{chatbot}, por exemplo, ajudaria o morador a interagir com a casa, guiando-o durante o processo de criação de regras para acionamento automático de dispositivos. A área de \textit{Business Intelligence} também oferece um vasto campo a ser explorado, com a criação de um aplicativo capaz de exibir dados da casa em gráficos parametrizados, com relatórios sobre o comportamento dos usuários. Como demonstrado na Seção \ref{coletaAnaliseDados} e Anexo \ref{DataCollected}, são obtidas informações relevantes sobre a vida dos moradores de uma residência mesmo com um número limitado de sensores e atuadores em uso.

Para a implementação atual, não foram realizados testes de carga para o servidor na nuvem, o qual foi  implementado como um monolito, em uma única instância. O próximo passo seria avaliar os melhores procedimentos de escalabilidade e estudo de sua performance com vários dispositivos e usuários conectados, de modo que seja possível a identificação de gargalos em seus componentes. Como visto na Seção \ref{servidorNaNuvem}, as tecnologias empregadas na implementação do servidor na nuvem --- nginx, Node.js, MongoDB e Redis --- possuem características favoráveis à sua utilização em aplicações altamente escaláveis.  O desafio, a partir daí, se concentraria na redundância para a comunicação com as casas, onde são utilizadas conexões WebSocket.

O monitoramento do servidor de nuvem também pode ser explorado, de maneira que sejam entendidas suas características de uso, peça essencial para o alcance de maior disponibilidade e robustez. A integração realizada durante o projeto foi simplificada e usou um serviço que apenas monitora uso de memória e CPU. Idealmente, deve ser possível também acompanhar eventos e transações importantes que ocorrem na nuvem, configurar alertas de picos de uso de recursos e poder ter uma visão consolidada dos logs, com busca e estatísticas.

Nos aspectos de hardware, e da infraestrutura de comunicação, podem ser explorados outros canais, transparentes ao usuário, e que seriam capazes de oferecer maior robustez à aplicação. Ao ampliar as formas de comunicação disponíveis, os módulos poderiam fazer uso de redes de dados --- 4G e futuramente 5G --- e não seriam mais dependentes exclusivamente da internet local. A integração, e também o desenvolvimento, de dispositivos vestíveis, integráveis com a casa, é também um passo futuro, que melhorará a experiência do usuário e expandirá as possibilidades de aplicação ---como, por exemplo, um relógio que monitora pessoas idosas, e envia notificações e alertas aos familiares.

A possibilidade de atualização de \textit{firmware} remotamente agregou em segurança e flexibilidade ao projeto, já que torna possível o envio de correções diretamente ao usuário, por meio da internet, sem a necessidade do contato direto com o módulo. Assim, frente a uma potencial ameaça, pode-se desenvolver uma correção e enviá-la aos clientes, que manteriam suas casas atualizadas.

Como outras sugestões aos próximos passos e caminhos futuros, indica-se a criação de testes unitários e automatizados, que verificam pequenas porções de código por vez. Isso facilitaria a adição de novas funcionalidades, garantindo a compatibilidade reversa, além de aumentar as chances de identificar falhas antes de liberar novas versões de software ao público. Com a mesma mentalidade, pode-se implementar uma \textit{pipeline} de integração contínua para identificar erros de integração rapidamente por meio de testes e verificação de código e permitir o lançamento de novas versões de maneira ágil. O fato do código-fonte do Hedwig já estar em um sistema de controle de versão, o GitHub, é um acelerador para o uso de serviços de integração contínua. Outro importante aspecto relacionado às tecnologias de Internet das Coisas, mas que não foi coberto neste trabalho são as questões éticas e relacionadas à privacidade do usuário, as quais são complexas e necessitarão de grandes esforços para a elaboração de uma legislação adequada.
