\chapter{Metodologia}

\section{Gerência do projeto}
% TODO Alguns dos itens abaixo podem/serão movidos para os apêndices
Para realizar a gerência do projeto Hedwig, foram usadas as diretrizes do Guia PMBOK \cite{pmi} e da norma ISO/IEC 12207 \cite{iso12207} como referência para coordenar os processos.

Para gerenciar as tarefas, estudos e pesquisas necessárias para a realização do projeto, foi utilizado o Trello\footnote{Pode ser acessado gratuitamente em https://trello.com/} - sistema online para organização de ideias e projetos, que permite listagem e acompanhamento de tarefas a serem realizadas, com deadlines, responsáveis e categorização em diversos tipos de tarefas.

\subsection{Gerência de Escopo Tempo}
\subsection{Gerência de Partes Interessadas Aquisição}
\subsection{Gerência de Processos de Software}
Para gerenciar o código-fonte e permitir o trabalho da equipe em múltiplas partes do projeto ao mesmo tempo, foi utilizado o Git, um sistema de controle de versão distribuído. Para publicação do código, foi escolhido o GitHub, onde está a organização do projeto Hedwig\footnote{https://github.com/hedwig-project} e os repositórios de código dos módulos associados ao sistema. A preferência pelo GitHub se deu pelas suas funcionalidades de gerenciamento e colaboração como a notificação de bugs, acompanhamento do progresso de tarefas e criação de wikis, além de ser uma plataforma conhecida por abrigar grandes projetos open-source que chegam a ter centenas ou milhares de contribuidores \cite{github}.

Para o fluxo de trabalho nesses repositórios, foi utilizado o fluxo conhecido como Feature Branch Workflow \cite{atlassian}, caracterizado pela criação de branches (ramificações) para o desenvolvimento de cada nova funcionalidade. Ao final do desenvolvimento de cada funcionalidade, é feito um pedido para mesclar o código desenvolvido em tal ramificação com o da ramificação principal (master branch).

\subsection{Gerência de Partes Interessadas}
\subsection{Gerência de Comunicação}
\subsection{Gerência de Escopo}
\subsection{Gerência de Riscos}
% TODO colocar matriz de probabilidade e impacto?

\section{Pesquisa bibliográfica}
O estudo dos tópicos relacionados a aprendizagem de máquina foi realizado com auxílio do curso Aprendizagem Automática do Professor Andrew Ng\footnote{ https://www.coursera.org/learn/machine-learning}, oferecido pela Universidade de Stanford e disponibilizado no Coursera, uma plataforma de MOOCs (Massive Open Online Courses) que oferece cursos abertos e especializações.

Os cursos da especialização em Data Science da Universidade Johns Hopkins\footnote{ https://www.coursera.org/specializations/jhu-data-science}, também disponíveis no Coursera, foram usados como referência e treinamento para realizar a coleta de dados de maneira metódica. Por esse motivo, foi dada maior atenção ao curso Getting and Cleaning Data. Contudo, também foi aproveitado conteúdo do curso Practical Machine Learning.

\section{Ferramentas e tecnologias}
Para aprender a utilizar a biblioteca React para o desenvolvimento do front-end, foi usada como referência a documentação oficial\footnote{https://facebook.github.io/react/docs/hello-world.html} oferecida pelo Facebook e o curso React for Beginners de Wes Bos\footnote{ https://reactforbeginners.com/}. O aprendizado de Redux foi auxiliado pelo curso Learn Redux\footnote{https://learnredux.com}, do mesmo autor.
